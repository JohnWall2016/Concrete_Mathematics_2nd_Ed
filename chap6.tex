% Chapter 6 of Concrete Mathematics
% (c) Addison-Wesley, all rights reserved.
\input gkpmac
\refin bib

\pageno=257 % Please begin on right-hand page (we want tables on facing pages)
\refin chap2
\refin chap4
\refin chap5
\refin chap7
\refin chap9

\beginchapter 6 Special Numbers

SOME SEQUENCES of "numbers" arise so often in mathematics that we recognize them
instantly and give them special names.
For example, everybody who learns arithmetic knows the sequence
of square numbers $\langle 1,4,9,16,\ldots\,\rangle$.
In Chapter~1 we encountered
the triangular numbers $\langle 1,3,6,10,\ldots\,\rangle$;
in Chapter~4 we studied
the prime numbers $\langle 2,3,5,7,\ldots\,\rangle$;
in Chapter~5 we looked briefly at
the Catalan numbers $\langle 1,2,5,14,\ldots\,\rangle$.

In the present chapter we'll get to know a few other important sequences.
First on our agenda will be the Stirling numbers $n\brace k$ and
$@n@\brack k$, and the Eulerian numbers $n\euler k$; these form triangular
patterns of coefficients analogous to the binomial coefficients $n\choose k$
in Pascal's triangle.
Then we'll take a good look at the harmonic numbers~$H_n$, and
%a quick glance at
the Bernoulli numbers~$B_n$; these differ from the other sequences we've been
studying because they're fractions, not integers.
Finally, we'll examine the fascinating Fibonacci numbers~$F_n$ and some
of their important generalizations.

\beginsection 6.1 Stirling Numbers

We begin with some close relatives of the binomial coefficients, the
"Stirling numbers", named after James "Stirling" (1692--1770). These numbers
come in two flavors, traditionally called by the no-frills names
``Stirling numbers of the first and second kind.\qback''
Although they have a venerable history and numerous applications, they
still lack a standard "notation". Following Jovan "Karamata",
\g\noindent\llap{``}\dots\ par cette notation, les formules deviennent plus sym\'etriques.''
\par\hfill\hskip0pt minus5pt\dash---J.~"Karamata"~[|karamata|]\g
we will write $n\brace k$ for Stirling
numbers of the second kind and $@n@\brack k$ for Stirling numbers
of the first kind; these symbols turn out to be more
user-friendly than the many other notations that people have tried.

Tables |stirl2-triangle| and |stirl1-triangle| show what
$n\brace k$ and~$@n@\brack k$ look like when $n$ and~$k$ are small.
A problem that involves the numbers 
``$1$, $7$, $6$,~$1$'' is likely to be related to $n\brace k$,
 and a problem that involves
``$6$, $11$, $6$,~$1$'' is
likely to be related to $@n@\brack k$,
just as we assume that a problem involving ``$1$, $4$, $6$, $4$, $1$''
is likely to be related to $n\choose k$;
these are the trademark sequences that appear when $n=4$.

\topinsert
\setbox0=\hbox{$\biggr\}$}
\def\\#1{\displaystyle{n\brace#1}\kern-\wd0}
\table Stirling's triangle for subsets.\tabref|stirl2-triangle|
\offinterlineskip
\halign to\hsize{\strut$\hfil#$\quad&\vrule#\kern-5pt\tabskip10pt plus 100pt&
 \hfil$#$&
 \hfil$#$&
 \hfil$#$&
 \hfil$#$&
 \hfil$#$&
 \hfil$#$&
 \hfil$#$&
 \hfil$#$&
 \hfil$#$&
 \hfil$#$\tabskip\wd0\cr
\omit&height 3pt\cr
n &&\\0&\\1&\\2&\\3&\\4&\\5&\\6&\\7&\\8&\\9\cr
\omit&height 2pt\cr
\noalign{\hrule width\hsize}
\omit&height 3pt\cr
0 && 1 \cr
1 && 0 & 1 \cr
2 && 0 & 1 & 1 \cr
3 && 0 & 1 & 3 & 1 \cr
4 && 0 & 1 & 7 & 6 & 1 \cr
5 && 0 & 1 & 15 & 25 & 10 & 1 \cr
6 && 0 & 1 & 31 & 90 & 65 & 15 & 1 \cr
7 && 0 & 1 & 63 & 301 & 350 & 140 & 21 & 1 \cr
8 && 0 & 1 & 127 & 966 & 1701 & 1050 & 266 & 28 & 1 \cr
9 && 0 & 1 & \ 255 & 3025 & 7770 & 6951 & 2646 & 462 & 36 & 1 \ \cr
\omit&height 2pt\cr}
\hrule width\hsize height.5pt
\kern4pt
\endinsert

Stirling numbers of the second kind show up more often than those of
the other variety, so let's consider last things first. The symbol $n\brace k$
\g("Stirling" himself considered this kind first in his book~[|stirling-method|].)\g
\tabref|nn:brace|%
stands for the number of ways to "partition a set" of $n$~things into
$k$~nonempty subsets. For example, there are seven ways to split
a four-element set into two parts:
\begindisplay
&\{1,2,3\}\cup\{4\}\,,\qquad
\{1,2,4\}\cup\{3\}\,,\qquad
\{1,3,4\}\cup\{2\}\,,\qquad
\{2,3,4\}\cup\{1\}\,,\cr
&\{1,2\}\cup\{3,4\}\,,\qquad
\{1,3\}\cup\{2,4\}\,,\qquad
\{1,4\}\cup\{2,3\}\,;
\eqno\eqref|stirl2-42|
\enddisplay
thus ${4\brace2}=7$. Notice that curly braces are used to denote sets
as well as the numbers $n\brace k$. This notational
kinship helps us remember the meaning
of $n\brace k$, which can be read ``$n$~subset~$k$.\qback''

Let's look at small~$k$.
There's just one way to put $n$ elements into a single nonempty set;
hence ${n\brace1}=1$, for all $n>0$. On the other hand ${0\brace1}=0$,
because a $0$-element set is empty.
%In general we have ${n\brace k}=0$ whenever $n<k$.

The case $k=0$ is a bit tricky. Things work out best if we agree that there's
just one way to partition an empty set into zero nonempty parts; hence
${0\brace0}=1$. But a nonempty set needs at least one part,
 so ${n\brace0}=0$ for $n>0$.

What happens when $k=2$? Certainly ${0\brace2}
=0$. If a set of $n>0$ objects is divided into two nonempty
parts, one of those parts contains the last object and some subset
of the first $n-1$~objects. There are $2^{n-1}$ ways to choose the
latter subset,
since each of the first $n-1$~objects is either in it or out of~it;
but we mustn't put all of those objects in~it, because we want to
end up with two nonempty parts. Therefore we subtract~$1$:
\begindisplay
{n\brace2}=2^{n-1}-1\,,\qquad\hbox{integer $n>0$}.
\eqno
\enddisplay
(This tallies with our enumeration of ${4\brace2}=7=2^3-1$ ways above.)

\topinsert
\setbox0=\hbox{$\biggr]$}
\def\\#1{\displaystyle{n\brack#1}\kern-\wd0}
\table Stirling's triangle for cycles.\tabref|stirl1-triangle|
\offinterlineskip
\halign to\hsize{\strut$\hfil#$\quad&\vrule#\kern-5pt\tabskip10pt plus 100pt&
 \hfil$#$&
 \hfil$#$&
 \hfil$#$&
 \hfil$#$&
 \hfil$#$&
 \hfil$#$&
 \hfil$#$&
 \hfil$#$&
 \hfil$#$&
 \hfil$#$\tabskip\wd0\cr
\omit&height 3pt\cr
n &&\\0&\\1&\\2&\\3&\\4&\\5&\\6&\\7&\\8&\\9\cr
\omit&height 2pt\cr
\noalign{\hrule width\hsize}
\omit&height 3pt\cr
0 && 1 \cr
1 && 0 & 1 \cr
2 && 0 & 1 & 1 \cr
3 && 0 & 2 & 3 & 1 \cr
4 && 0 & 6 & 11 & 6 & 1 \cr
5 && 0 & 24 & 50 & 35 & 10 & 1 \cr
6 && 0 & 120 & 274 & 225 & 85 & 15 & 1 \cr
7 && 0 & 720 & 1764 & 1624 & 735 & 175 & 21 & 1 \cr
8 && 0 & 5040 & 13068 & 13132 & 6769 & 1960 & 322 & 28 & 1 \cr
9 && 0 & 40320 & 109584 & 118124 & 67284 & 22449 & 4536 & 546 & \ 36 & 1 \ \cr
\omit&height 2pt\cr}
\hrule width\hsize height.5pt
\kern4pt
\endinsert

\goodbreak
A modification of this argument leads to a recurrence by which we can
compute $n\brace k$ for all~$k@$: Given a set of $n>0$ objects to be
partitioned into $k$ nonempty parts, we either put the last object
into a class by itself (in $n-1\brace k-1$ ways), or we put it together
with some nonempty subset of the first $n-1$ objects. There are $k{n-1\brace k}$
possibilities in the latter case, because each of the $n-1\brace k$ ways
to distribute the first $n-1$ objects into $k$~nonempty parts gives $k$~subsets
that the $n$th object can join. Hence
\begindisplay
{n\brace k}=k@{n-1\brace k}+{n-1\brace k-1}\,,
\qquad\hbox{integer $n>0$}.
\eqno\eqref|stirl2-rec|
\enddisplay
This is the law that generates Table |stirl2-triangle|; without the factor
of~$k$ it would reduce to the addition formula \equ(5.|bc-addition|)
that generates Pascal's triangle.

\smallbreak
And now, "Stirling numbers of the first kind". These are somewhat
like the others, but $n\brack k$ counts the number of ways to arrange $n$~objects
\tabref|nn:brack|%
into $k$ {\it"cycles"\/} instead of subsets. We verbalize
`$n\brack k$' by saying ``$n$~cycle~$k$.\qback''

Cycles are cyclic arrangements, like the "necklace"s we considered in
Chapter~4. The "cycle"
\begindisplay \advance\belowdisplayskip -3pt
\unitlength=1pt
\def\necklace#1#2#3#4{\beginpicture(30,30)(-15,-15)%
 \ovaltlfalse\ovaltrfalse\ovalblfalse\ovalbrfalse
 {\ovaltrtrue\put(5,6){\oval(12,12)}}%
 {\ovaltltrue\put(-5,6){\oval(12,12)}}%
 {\ovalbltrue\put(-5,-6){\oval(12,12)}}%
 {\ovalbrtrue\put(5,-6){\oval(12,12)}}%
 \put(0,12){\makebox(0,0){$#1$}}%
 \put(11,0){\makebox(0,0){$#2$}}%
 \put(0,-12){\makebox(0,0){$#3$}}%
 \put(-11,0){\makebox(0,0){$#4$}}%
 \endpicture}
\kern2pt\necklace ABC{D\kern2pt}
\enddisplay
can be written more compactly as `$[A,B,C,D]$', with the understanding that
\begindisplay
[A,B,C,D]=[B,C,D,A]=[C,D,A,B]=[D,A,B,C]\,;
\enddisplay
a cycle ``wraps around'' because its end is joined to its beginning.
On the other hand, the cycle
$[A,B,C,D]$ is not the same as $[A,B,D,C]$ or $[D,C,B,A]$.

There are eleven different ways
\g\noindent\llap{``}There are nine and~sixty ways\par of constructing tribal lays,\par
And-every-single-one-of-them-is-right.''\par\hfill\dash---Rudyard "Kipling"\g
to make two cycles from four elements:
\begindisplay
&[1,2,3]\,[4]\,,\qquad
[1,2,4]\,[3]\,,\qquad
[1,3,4]\,[2]\,,\qquad
[2,3,4]\,[1]\,,\cr
&[1,3,2]\,[4]\,,\qquad
[1,4,2]\,[3]\,,\qquad
[1,4,3]\,[2]\,,\qquad
[2,4,3]\,[1]\,,\cr
&[1,2]\,[3,4]\,,\qquad
[1,3]\,[2,4]\,,\qquad
[1,4]\,[2,3]\,;
\eqno\eqref|stirl1-42|
\enddisplay
hence ${4\brack2}=11$.

A singleton cycle (that is, a cycle with only one element) is essentially
the same as a singleton set (a set with only one element). Similarly,
a $2$-cycle "!bicycling (goes under sports too)"
is like a $2$-set, because we have $[A,B]=[B,A]$ just as $\{A,B\}=
\{B,A\}$. But there are two {\it different\/} $3$-cycles, $[A,B,C]$ and $[A,C,B]$.
Notice, for example, that the eleven cycle pairs in \thiseq\ can be obtained
from the seven set pairs in \eq(|stirl2-42|) by making two cycles from each
of the $3$-element sets.

In general, $n!/n=(n-1)!$ different $n$-cycles
 can be made from any $n$-element set, whenever
$n>0$. (There are $n!$ permutations, and each $n$-cycle corresponds to~$n$
of them because any one of its elements can be listed first.)
Therefore we have
\begindisplay
{n\brack1}=(n-1)!\,,\qquad\hbox{integer $n>0$}.
\eqno
\enddisplay
This is much larger than the value ${n\brace1}=1$ we had for Stirling
subset numbers. In fact, it is easy to see that the
cycle numbers must be at least as large as the subset numbers,
\begindisplay
{n\brack k}\ge{n\brace k}\,,\qquad\hbox{integers $n,k\ge0$},
\eqno
\enddisplay
because every partition into nonempty subsets leads to at least one
arrangement of cycles.

Equality holds in \thiseq\ when all the cycles are necessarily singletons
or doubletons, because cycles are equivalent to subsets in such cases.
This happens when $k=n$ and when $k=n-1$; hence
\begindisplay
{n\brack n}={n\brace n}\,;\qquad{n\brack n-1}={n\brace n-1}\,.
\enddisplay
In fact, it is easy to see that
\begindisplay
{n\brack n}={n\brace n}=1\,;\qquad{n\brack n-1}={n\brace n-1}={n\choose2}\,.
\eqno\eqref|stirl-nn-1|
\enddisplay
(The number of ways to arrange $n$ objects into $n-1$ cycles or subsets is
the number of ways to choose the two objects that will be in the same
cycle or subset.) The triangular numbers ${n\choose2}=1$, $3$, $6$,~$10$,
\dots\ are conspicuously present in both Table |stirl2-triangle| and Table
|stirl1-triangle|.

We can derive a recurrence for $n\brack k$ by modifying the argument we
used for~$n\brace k$. Every arrangement of $n$~objects in $k$~cycles
either puts the last object into a cycle by itself (in $n-1\brack k-1$ ways)
or inserts that object into one of the $n-1\brack k$ cycle arrangements
of the first $n-1$ objects. In the latter case, there are $n-1$ different
ways to do the insertion. (This takes some thought, but it's not hard
to verify that there are $j$ ways to put a new element into a
$j$-cycle in order to make a $(j+1)$-cycle. When $j=3$, for example,
the cycle $[A,B,C]$ leads to
\begindisplay
[A,B,C,D]\,,\qquad
[A,B,D,C]\,,\qquad\hbox{or}\qquad
[A,D,B,C]
\enddisplay
when we insert a new element $D$, and there are no other possibilities.
Summing over all $j$ gives a total of $n-1$ ways to insert an $n$th object
into a cycle decomposition of $n-1$ objects.)
The desired recurrence is therefore
\begindisplay
{n\brack k}=(n-1){n-1\brack k}+{n-1\brack k-1}\,,
\qquad\hbox{integer $n>0$}.
\eqno\eqref|stirl1-rec|
\enddisplay
This is the addition-formula analog that generates Table |stirl1-triangle|.

Comparison of \thiseq\ and \eq(|stirl2-rec|) shows that the first term
on the right side is multiplied by its upper index $(n-1)$ in the
case of Stirling cycle numbers, but by its lower index~$k$
in the case of Stirling subset numbers. We can therefore
perform ``"absorption"'' in terms like $n{n\brack k}$ and $k{n\brace k}$,
when we do proofs by mathematical induction.

Every permutation is equivalent to a set of cycles. For example, consider the
permutation that takes $123456789$ into $384729156$. We can conveniently
represent it in two rows,
\begindisplay
&1\,2\,3\,4\,5\,6\,7\,8\,9\cr
&3\,8\,4\,7\,2\,9\,1\,5\,6\,,\cr
\enddisplay
showing that $1$ becomes $3$ and $2$ becomes $8$, etc. The cycle structure
comes about because $1$ becomes $3$, which becomes~$4$, which becomes~$7$,
which becomes the original element~$1$; that's
the cycle $[1,3,4,7]$. Another cycle in this permutation is $[2,8,5]$;
still another is $[6,9]$. Therefore the permutation $384729156$ is
equivalent to the cycle arrangement
\begindisplay
[1,3,4,7]\,[2,8,5]\,[6,9]\,.
\enddisplay
If we have any permutation $\pi_1\pi_2\ldots\pi_n$ of $\{1,2,\ldots,n\}$,
every element is in a unique cycle. For if we start with $m_0=m$ and
look at $m_1=\pi_{m_0}$, $m_2=\pi_{m_1}$, etc., we must eventually
come back to $m_k=m_0$. (The numbers must repeat sooner or later,
and the first number to reappear must be $m_0$ because we know the
unique predecessors of the other numbers $m_1$, $m_2$, \dots,~$m_{k-1}$.)
Therefore every permutation defines a cycle arrangement. Conversely, every
cycle arrangement obviously defines a permutation if we reverse the
construction, and this one-to-one
correspondence shows that permutations and cycle arrangements are
essentially the same thing.

Therefore $n\brack k$ is the number of permutations of $n$ objects
that contain exactly $k$ cycles. If we sum $n\brack k$ over all~$k$, we
must get the total number of permutations:
\begindisplay
\sum_{k=0}^n{n\brack k}=n!\,,\qquad\hbox{integer $n\ge0$}.
\eqno\eqref|sum-to-factorial|
\enddisplay
For example, $6+11+6+1=24=4!$.

Stirling numbers are useful because the recurrence relations \eq(|stirl2-rec|)
and \eq(|stirl1-rec|) arise in a variety of problems. For example, if we
want to represent ordinary powers $x^n$ by falling powers $x\_n$, we
find that the first few cases are
\begindisplay
x^0&=x\_0\,;\cr
x^1&=x\_1\,;\cr
x^2&=x\_2+x\_1\,;\cr
x^3&=x\_3+3x\_2+x\_1\,;\cr
x^4&=x\_4+6x\_3+7x\_2+x\_1\,.\cr
\enddisplay
These coefficients look suspiciously
like the numbers in Table |stirl2-triangle|,
reflected between left and right; therefore we can be pretty confident
that the general formula is\g\vskip20pt We'd better define\smallskip
${n\brace k}={n\brack k}=0$\smallskip
when $k<0@$ and $n\ge0$.\g
\begindisplay
x^n=\sum_k{n\brace k}x\_k\,,\qquad\hbox{integer $n\ge0$.}
\eqno\eqref|expand-ord-to-falling|
\enddisplay
And sure enough,
a simple proof by induction clinches the argument: We have $x\cdt x\_k=
x\_{k+1}+kx\_k$, because $x\_{k+1}=x\_k@(x-k)$; hence $x\cdt x^{n-1}$ is
\begindisplay \openup3pt
x\sum_k{n-1\brace k}x\_k&=\sum_k{n-1\brace k}x\_{k+1}+\sum_k{n-1\brace k}kx\_k\cr
&=\sum_k{n-1\brace k-1}x\_k+\sum_k{n-1\brace k}kx\_k\cr
&=\sum_k\biggl(k{n-1\brace k}+{n-1\brace k-1}\biggr)x\_k
 =\sum_k{n\brace k}x\_k\,.\cr
\enddisplay
In other words, Stirling subset numbers are the coefficients of
"factorial powers" that yield ordinary powers.

We can go the other way too,
because Stirling cycle numbers are the coefficients of ordinary
powers that yield factorial powers:
\begindisplay
x\_^0&=x^0\,;\cr
x\_^1&=x^1\,;\cr
x\_^2&=x^2+x^1\,;\cr
x\_^3&=x^3+3x^2+2x^1\,;\cr
x\_^4&=x^4+6x^3+11x^2+6x^1\,.\cr
\enddisplay
We have $(x+n-1)\cdt x^k=x^{k+1}+(n-1)x^k$, so a proof like the one just
given shows that
\begindisplay
(x+n-1)x\_^{n-1}=(x+n-1)\sum_k{n-1\brack k}x^k=\sum_k{n\brack k}x^k\,.
\enddisplay
This leads to a proof by induction of the general formula
\begindisplay
x\_^n=\sum_k{n\brack k}x^k\,,\qquad\hbox{integer $n\ge0$.}
\eqno\eqref|expand-rising-to-ord|
\enddisplay
(Setting $x=1$ gives \eq(|sum-to-factorial|) again.)

But wait, you say. This equation involves rising factorial powers $x\_^n$,
while \eq(|expand-ord-to-falling|) involves falling factorials $x\_n$.
What if we want to express $x\_n$ in terms of ordinary powers, or if we
want to express $x^n$
in terms of rising powers? Easy; we just throw in some minus signs and get
\begindisplay \openup4pt
x^n&=\sum_k{n\brace k}(-1)^{n-k}x\_^k\,,\qquad\hbox{integer $n\ge0$};
\eqno\eqref|expand-ord-to-rising|\cr
x\_n&=\sum_k{@n@\brack k}(-1)^{n-k}x^k\,,\qquad\hbox{integer $n\ge0$}.
\eqno\eqref|expand-falling-to-ord|\cr
\enddisplay
This works because, for example, the formula
\begindisplay
x\_4=x(x-1)(x-2)(x-3)=x^4-6x^3+11x^2-6x
\enddisplay
is just like the formula
\begindisplay
x\_^4=x(x+1)(x+2)(x+3)=x^4+6x^3+11x^2+6x
\enddisplay
but with alternating signs. The general identity
\begindisplay \postdisplaypenalty=10000
x\_n=(-1)^n(-x)\_^n
\eqno
\enddisplay
of exercise 2.|rising-and-falling|
converts \eq(|expand-ord-to-falling|) to \eq(|expand-ord-to-rising|) and
\eq(|expand-rising-to-ord|) to \eq(|expand-falling-to-ord|) if we negate~$x$.

\pageinsert % have to place this carefully so that \eqnos don't get out of order!
\table Basic "Stirling number identities", for integer $n\ge0$.\tabref|stirling-id1|
\begindisplay\abovedisplayskip=-2pt \belowdisplayskip=5pt \openup4pt%
 \advance\baselineskip0pt plus .5pt  \advance\lineskip0pt plus .5pt
\noalign{\hbox{Recurrences:}}
{n\brace k}&=k{n-1\brace k}+{n-1\brace k-1}\,.\cr
{@n@\brack k}&=(n-1){n-1\brack k}+{n-1\brack k-1}\,.\cr
\noalign{\smallskip\hbox{Special values:}}
{n\brace 0}&={@n@\brack 0}=\[n=0]\,.\cr
{n\brace 1}&=\[n>0]\,;\quad}\hfill{
{@n@\brack 1}&=(n-1)!\,\[n>0]\,.\cr
{n\brace 2}&=(2^{n-1}-1)\[n>0]\,;\qquad}\hfill{
{@n@\brack 2}&=(n-1)!\,H_{n-1}\,\[n>0]\,.\cr
{n\brace n-1}&={n\brack@n-1@}={n\choose 2}\,.\cr
{n\brace n}&={@n@\brack n}={n\choose n}=1\,.\cr
{n\brace k}&={@n@\brack k}={n\choose k}=0\,,\qquad\hbox{if $k>n$}.}\hidewidth{\cr
\noalign{\smallskip\hbox{Converting between powers:}}
x^n&=\sum_k{n\brace k}x\_k=\sum_k{n\brace k}(-1)^{n-k}x\_^k\,.}\hidewidth{\cr
x\_n&=\sum_k{@n@\brack k}(-1)^{n-k}x^k\,;\cr
x\_^n&=\sum_k{@n@\brack k}x^k\,.\cr
\noalign{\smallskip\hbox{Inversion formulas:}}
\sum_k{@n@\brack k}{k\brace m}(-1)^{n-k}=\[m=n]\,;}\hidewidth{\cr
\sum_k{n\brace k}{k\brack m}(-1)^{n-k}=\[m=n]\,.}\hidewidth{\cr
\enddisplay
\hrule width\hsize height.5pt
\kern4pt
\endinsert
\pageinsert
\table Additional Stirling number identities, for integers $l,m,n\ge0$.\tabref|stirling-id2|
\begindisplay\abovedisplayskip=-1pt \belowdisplayskip=5pt \openup3pt%
 \advance\displayindent-\parindent\advance\displaywidth\parindent%
 \advance\baselineskip0pt plus .5pt  \advance\lineskip0pt plus .5pt
{n+1\brace m+1}&=\sum_k{n\choose k}{k\brace m}\,.\eqno\eqref|sid-2|\cr
{n+1\brack m+1}&=\sum_k{@n@\brack k}{k\choose m}\,.\eqno\eqref|sid-1|\cr
{n\brace m}&=\sum_k{n\choose k}{k+1\brace m+1}(-1)^{n-k}\,.\eqno\eqref|sid-4|\cr
\noalign{\kern\prevdepth\g\vskip10pt
$n^m(-1)^{n-m}{n\brack m}$\par
\vskip6pt
$\quad=\sum\limits_k\!{\,n\,\brack\, k\,}{-m\choose k-m}n^k\,.$\kern-12pt
\null\g}
{n\brack@m@}&=\sum_k{n+1\brack k+1}{k\choose m}(-1)^{m-k}\,.\eqno\eqref|sid-3|\cr
m!\,{n\brace m}&=\sum_k{m\choose k}k^n(-1)^{m-k}\,.\eqno\eqref|sid-5|\cr
{n+1\brace m+1}&=\sum_{k=0}^n{k\brace m}(m+1)^{n-k}\,.\eqno\eqref|sid-7|\cr
{n+1\brack m+1}&=\sum_{k=0}^n{k\brack m}n\_{n-k}=n!\sum_{k=0}^n{k\brack m}\big/k!\,.\eqno\eqref|sid-6|\cr
{m+n+1\brace m}&=\sum_{k=0}^m\,k\,{n+k\brace k}\,.\eqno\eqref|sid-9|\cr
{m+n+1\brack m}&=\sum_{k=0}^m\,(n+k){n+k\brack k}\,.\eqno\eqref|sid-8|\cr
{n\choose m}&=\sum_k{n+1\brace k+1}{k\brack m}(-1)^{m-k}\,.\eqno\eqref|sid-10|\cr
\noalign{\kern\prevdepth\g\vskip17pt Also,\par\vskip6pt
${n\choose m}(n-1)\_{n-m}$\par\vskip6pt
$\quad=\sum_k{\,n\,\brack\,k\,}{k\brace m}\,,$\par\vskip6pt
a generalization of~\eq(|sum-to-factorial|).\g}
n\_{n-m}\,\[n\ge m]&=
\sum_k{n+1\brack k+1}{k\brace m}(-1)^{m-k}\,.\eqno\eqref|sid-11|\cr
{n\brace n-m}&=\sum_k{m-n\choose m+k}{m+n\choose n+k}{m+k\brack k}\,.\eqno\eqref|sid-13|\cr
{n\brack@n-m@}&=\sum_k{m-n\choose m+k}{m+n\choose n+k}{m+k\brace k}\,.\eqno\eqref|sid-12|\cr
\indent{n\brace l+m}{l+m\choose l}&=\sum_k{k\brace l}{n-k\brace m}{n\choose k}\,.\eqno\eqref|sid-15|\cr
{n\brack@l+m@}{l+m\choose l}&=\sum_k{k\brack l}{n-k\brack m}{n\choose k}\,.\eqno\eqref|sid-14|\cr
\enddisplay
\hrule width\hsize height.5pt
\kern4pt
\endinsert

We can remember when to stick the $(-1)^{n-k}$ factor into a formula
like \eq(|expand-ord-to-rising|) because there's a
natural ordering of powers when $x$ is large:
\begindisplay
x\_^n>x^n>x\_n\,,\qquad\hbox{for all $x>n>1$}.\eqno
\enddisplay
The Stirling numbers $n\brack k$ and $n\brace k$ are nonnegative, so we have
to use minus signs when expanding a ``small'' power in terms of ``large'' ones.

We can plug \eq(|expand-rising-to-ord|) into \eq(|expand-ord-to-rising|) and
get a double sum:
\begindisplay
x^n=\sum_k{n\brace k}(-1)^{n-k}x\_^k
 =\sum_{k,m}{n\brace k}{k\brack m}(-1)^{n-k}x^m\,.
\enddisplay
This holds for all $x$, so the coefficients of $x^0$, $x^1$, \dots,~$x^{n-1}$,
$x^{n+1}$, $x^{n+2}$,~\dots\
on the right must all be zero and we must have the identity
\begindisplay
\sum_k{n\brace k}{k\brack m}(-1)^{n-k}=\[m=n]\,,
\qquad\hbox{integers $m,n\ge0$}.
\eqno\eqref|stirl-inv|
\enddisplay

Stirling numbers, like binomial coefficients, satisfy many surprising
identities. But these identities aren't as versatile as the ones we had
in Chapter~5, so they aren't applied nearly as often. Therefore it's
best for us just to list the simplest ones, for future reference when
a tough Stirling nut needs to be cracked. Tables |stirling-id1|
%\g Or when a crusty Stirling problem needs to be polished off, like
%Stirling silver.\g
and |stirling-id2| contain the formulas that are most frequently useful;
the principal identities we have already derived are repeated there.

When we studied binomial coefficients in Chapter~5, we found that it was
advantageous to define $n\choose k$ for negative~$n$ in such a way that
the identity ${n\choose k}={n-1\choose k}+{n-1\choose k-1}$ is valid without
any restrictions. Using that identity to extend the
$n\choose k$'s beyond those with combinatorial significance,
we discovered (in Table~|pascal-triangle-up|) that Pascal's
triangle essentially reproduces itself in a rotated form when we
extend it upward. Let's try the same thing with "Stirling's triangles": What
happens if we decide that the basic recurrences
\begindisplay \openup5pt \advance\abovedisplayskip-3pt
{n\brace k}&=k@{n-1\brace k}+{n-1\brace k-1}\cr
{n\brack k}&=(n-1){n-1\brack k}+{n-1\brack k-1}\cr
\enddisplay
are valid for all integers $n$ and $k$? The solution becomes unique if we
make the reasonable additional stipulations that
\begindisplay
{0\brace k}={0\brack k}=\[k=0]\And {n\brace0}={n\brack0}=\[n=0]\,.
\eqno
\enddisplay
In fact, a surprisingly pretty pattern emerges:
Stirling's triangle for cycles appears above Stirling's triangle for subsets,
and vice versa! The two kinds of Stirling numbers are
related by an extremely simple law [|knuth-tnn|, |knuth-cp|]:
\begindisplay\advance\abovedisplayskip-3pt\advance\belowdisplayskip-3pt
{@n@\brack k}={-k\brace-n}\,,\qquad\hbox{integers $k,n$}.
\eqno\eqref|stirl-negation|
\enddisplay
We have ``"duality",\qback'' something like the relations between min and max,
between $\lfloor x \rfloor$ and~$\lceil x\rceil$, between $x\_n$
and~$x\_^n$, between gcd and lcm. It's easy to check that both of the recurrences
${n\brack k}=(n-1){n-1\brack k}+{n-1\brack k-1}$ and
${n\brace k}=k{n-1\brace k}+{n-1\brace k-1}$ amount to the same thing, under
this correspondence.

\topinsert
\setbox0=\hbox{$\biggr\}$}
\def\\#1{\displaystyle{n\brace#1}\kern-\wd0}
\table Stirling's triangles in tandem.\tabref|stirl-triangles|
\offinterlineskip
\halign to\hsize{\strut$\hfil#$\quad&\vrule#\kern-5pt\tabskip10pt plus 100pt&
 \hfil$#$&
 \hfil$#$&
 \hfil$#$&
 \hfil$#$&
 \hfil$#$&
 \hfil$#$&
 \hfil$#$&
 \hfil$#$&
 \hfil$#$&
 \hfil$#$&
 \hfil$#$\tabskip\wd0\cr
\omit&height 3pt\cr
n &&\\{-5}&\\{-4}&\\{-3}&\\{-2}&\\{-1}&\\0&\\1&\\2&\\3&\\4&\\5\cr
\omit&height 2pt\cr
\noalign{\hrule width\hsize}
\omit&height 3pt\cr
-5&& 1 \cr
-4&& 10 & 1 \cr
-3&& 35 & 6 & 1 \cr
-2&& 50 & 11 & 3 & 1 \cr
-1&& 24 & 6 & 2 & 1 & 1 \cr
0 && 0 & 0 & 0 & 0 & 0 & 1 \cr
1 && 0 & 0 & 0 & 0 & 0 & 0 & 1 \cr
2 && 0 & 0 & 0 & 0 & 0 & 0 & 1 & 1 \cr
3 && 0 & 0 & 0 & 0 & 0 & 0 & 1 & 3 & 1 \cr
4 && 0 & 0 & 0 & 0 & 0 & 0 & 1 & 7 & 6 & 1 \cr
5 && 0 & 0 & 0 & 0 & 0 & 0 & 1 & 15 & 25 & 10 & 1 \ \cr
\omit&height 2pt\cr}
\hrule width\hsize height.5pt
\kern4pt
\endinsert

\beginsection 6.2 Eulerian Numbers

Another triangle of values pops up now and again, this one due to "Euler"
[|euler-eulerian|, \S13; |euler-diff-calc|, page 485],
 and we denote its
elements by $n\euler k$. The angle brackets in this case suggest
``less than'' and ``greater than'' signs; $n\euler k$ is the number of
permutations $\pi_1\pi_2\ldots\pi_n$ of $\{1,2,\ldots,n\}$ that
have $k$ {\it"ascents"}, namely, $k$~places where $\pi_j<\pi_{j+1}$.
"!descents, \string\see ascents"
"!notation, new"
(Caution: This notation is less standard than our notations
\g(Knuth [|knuth3|, first edition] used\smallskip
 $n\euler k+1$ for $n\euler k$.)\g
\tabref|nn:euler|%
$n\brack k$, $n\brace k$ for Stirling numbers. But we'll see that it makes good
sense.)

For example, eleven permutations of $\{1,2,3,4\}$ have two ascents:
\begindisplay\advance\abovedisplayskip-2pt\advance\belowdisplayskip-2pt%
\openup-2pt
&1324\,,\quad 
1423\,,\quad 
2314\,,\quad
2413\,,\quad 
3412\,;\cr
&1243\,,\quad 
1342\,,\quad 
2341\,;\qquad 
2134\,,\quad
3124\,,\quad 
4123\,.
\enddisplay
(The first row lists the permutations with $\pi_1<\pi_2>\pi_3<\pi_4$; the
second row lists those with $\pi_1<\pi_2<\pi_3>\pi_4$ and
$\pi_1>\pi_2<\pi_3<\pi_4$.) Hence ${4\euler2}=11$. Table~|euler-triangle|
lists the smallest Eulerian numbers; notice that the trademark sequence
is $1$,~$11$,~$11$,~$1$ this time. There can be at most $n-1$~ascents,
when $n>0$, so we have ${n\euler n}=\[n=0]$ on the diagonal of the triangle.

\topinsert
\setbox0=\hbox{$\biggr>$}
\def\\#1{\displaystyle{n\euler#1}\kern-\wd0}
\table Euler's triangle.\tabref|euler-triangle|
\offinterlineskip
\halign to\hsize{\strut$\hfil#$\quad&\vrule#\kern-5pt\tabskip10pt plus 100pt&
 \hfil$#$&
 \hfil$#$&
 \hfil$#$&
 \hfil$#$&
 \hfil$#$&
 \hfil$#$&
 \hfil$#$&
 \hfil$#$&
 \hfil$#$&
 \hfil$#$\tabskip\wd0\cr
\omit&height 3pt\cr
n &&\\0&\\1&\\2&\\3&\\4&\\5&\\6&\\7&\\8&\\9\cr
\omit&height 2pt\cr
\noalign{\hrule width\hsize}
\omit&height 3pt\cr
0 && 1 \cr
1 && 1 & 0 \cr
2 && 1 & 1 & 0 \cr
3 && 1 & 4 & 1 & 0 \cr
4 && 1 & 11 & 11 & 1 & 0 \cr
5 && 1 & 26 & 66 & 26 & 1 & 0 \cr
6 && 1 & 57 & 302 & 302 & 57 & 1 & 0 \cr
7 && 1 & 120 & 1191 & 2416 & 1191 & 120 & 1 & 0\kern-.5em \cr
8 && 1 & 247 & 4293 & 15619 & 15619 & 4293 & 247 & 1\kern-.5em & 0 \cr
9 && 1 & 502 & 14608 & 88234 & 156190 & 88234 & 14608 & \kern.5em502\kern-.5em
 & \ 1 & 0 \ \cr
\omit&height 2pt\cr}
\hrule width\hsize height.5pt
\kern4pt
\endinsert

"Euler's triangle", like Pascal's, is symmetric between left and right.
But in this case the symmetry law is slightly different:
\begindisplay
{n\euler k}={n\euler n-1-k}\,,\qquad\hbox{integer $n>0$};
\eqno\eqref|eulerian-sym|
\enddisplay
The permutation $\pi_1\pi_2\ldots\pi_n$ has~$n-1-k$ ascents if and only if
its ``reflection'' $\pi_n\ldots\pi_2\pi_1$ has $k$ ascents.

Let's try to find a recurrence for $n\euler k$. Each permutation
$\rho=\rho_1\ldots\rho_{n-1}$ of $\{1,\ldots,n-1\}$ leads to $n$ permutations
of $\{1,2,\ldots,n\}$ if we insert the new element~$n$ in all possible ways.
Suppose we put $n$ in position~$j$, obtaining the permutation
$\pi=\rho_1\ldots\rho_{j-1}\,n\,\rho_j\ldots\rho_{n-1}$. The number of
ascents in $\pi$ is the same as the number in $\rho$,
if $j=1$ or if $\rho_{j-1}<\rho_j$; it's
one greater than the number in~$\rho$, if
$\rho_{j-1}>\rho_j$ or if $j=n$.
Therefore $\pi$ has $k$ ascents
in a total of $(k+1){n-1\euler k}$ ways from permutations $\rho$ that
have $k$~ascents, plus a total of $\bigl((n-2)-(k-1)+1\bigr){n-1\euler k-1}$ ways
from permutations $\rho$ that have $k-1$ ascents. The desired recurrence is
\begindisplay
{n\euler k}=(k+1){n-1\euler k}+(n-k){n-1\euler k-1}\,,\quad
\hbox{integer $n>0$}.
\eqno\eqref|eulerian-rec|
\enddisplay
Once again we start the recurrence off by setting
\begindisplay \postdisplaypenalty=10000
{0\euler k}=\[k=0]\,,\qquad\hbox{integer $k$},
\eqno
\enddisplay
and we will assume that ${n\euler k}=0$ when $k<0$.

"Eulerian numbers" are useful primarily because they provide an unusual
connection between ordinary powers and consecutive binomial coefficients:
\begindisplay
x^n=\sum_k{n\euler k}{x+k\choose n}\,,\qquad\hbox{integer $n\ge0$}.
\eqno\eqref|expand-ord-to-consec|
\enddisplay
(This is called ``"Worpitzky"'s identity'' [|worpitzky|].)
\g Western scholars have recently learned of a significant Chinese
\looseness=-1
book by "Li" Shan-Lan~[|li-shan-lan|; |martzloff|, pages\kern-.5em\null\par
320--325],
published in 1867, which contains
the first known appearance of formula~\eq(|expand-ord-to-consec|).\g
For example, we have
\begindisplay \openup5pt \tightplus
x^2&={x\choose2}+{x+1\choose2}\,,\cr
x^3&={x\choose3}+4{x+1\choose3}+{x+2\choose3}\,,\cr
x^4&={x\choose4}+11{x+1\choose4}+11{x+2\choose4}+{x+3\choose4}\,,\cr
\enddisplay
and so on. It's easy to prove \eq(|expand-ord-to-consec|) by
induction (exercise |prove-eulerian-expansion|).

Incidentally, \thiseq\ gives us yet another way to obtain the sum of the
"!sum of consecutive squares"
first~$n$~squares: We have $k^2={2\euler0}{k\choose2}+{2\euler1}{k+1\choose2}
 ={k\choose2}+{k+1\choose2}$, hence
\begindisplay \let\displaystyle=\textstyle \openup7pt
1^2+2^2+\cdots+n^2&=\bigl({1\choose2}+{\2\choose2}+\cdots+{n\choose2}\bigr)
+\bigl({\2\choose2}+{3\choose2}+\cdots+{n+1\choose2}\bigr)\cr
&={n+1\choose3}+{n+\2\choose3}={1\over6}(n+1)n\bigl((n-1)+(n+2)\bigr)\,.
\enddisplay

The Eulerian recurrence \eq(|eulerian-rec|) is a bit more complicated than
the Stirling recurrences \eq(|stirl2-rec|) and \eq(|stirl1-rec|), so
we don't expect the numbers $n\euler k$ to satisfy as many simple identities.
Still, there are a few:
\begindisplay \openup5pt
{n\euler m}&=\sum_{k=0}^m{n+1\choose k}(m+1-k)^n(-1)^k\,;
\eqno\eqref|eulerian-expansion|\cr
m!\,{n\brace m}&=\sum_k{n\euler k}{k\choose n-m}\,;
\eqno\eqref|expand-stirling-to-eulerian|\cr
{n\euler m}&=\sum_k{n\brace k}{n-k\choose m}(-1)^{n-k-m}\,k!\,.
\eqno\eqref|expand-eulerian-to-stirling|\cr
\enddisplay
If we multiply \eq(|expand-stirling-to-eulerian|) by $z^{n-m}$ and sum on~$m$,
we get $\sum_m{n\brace m}m!\,z^{n-m}
=\sum_k{n\euler k}(z+1)^k$.
Replacing $z$ by $z-1$ and equating coefficients of $z^k$ gives
\eq(|expand-eulerian-to-stirling|). Thus the last two of these identities
are essentially equivalent. The first identity,
\eq(|eulerian-expansion|), gives us special values
when $m$ is small:
\begindisplay \postdisplaypenalty=-150 \tightplus
{n\euler0}=1\,;\quad{n\euler1}=2^n-n-1\,;
\quad{n\euler2}=3^n-(n+1)2^n+{n+1\choose2}\,.
\enddisplay

\topinsert
\setbox0=\hbox{$\biggr>\mskip-7mu\biggr>$}
\def\\#1{\displaystyle{\Euler n#1}\kern-\wd0}
\table Second-order Eulerian triangle.\tabref|euler2-triangle|
\offinterlineskip
\halign to\hsize{\strut$\hfil#$\quad&\vrule#\kern-5pt\tabskip10pt plus 100pt&
 \hfil$#$&
 \hfil$#$&
 \hfil$#$&
 \hfil$#$&
 \hfil$#$&
 \hfil$#$&
 \hfil$#$&
 \hfil$#$&
 \hfil$#$\tabskip\wd0\cr
\omit&height 3pt\cr
n &&\\0&\ \\1&\\2&\\3&\\4&\\5&\kern-.3em\\6\kern.3em&\\7&\ \\8\cr
\omit&height 2pt\cr
\noalign{\hrule width\hsize}
\omit&height 3pt\cr
0 && 1 \cr
1 && 1 & 0 \cr
2 && 1 & 2 & 0 \cr
3 && 1 & 8 & 6 & 0 \cr
4 && 1 & 22 & 58 & 24 & 0 \cr
5 && 1 & 52 & 328 & 444 & 120 & 0 \cr
6 && 1 & 114 & 1452 & 4400 & 3708 & 720 & 0 \cr
7 && 1 & 240 & 5610 & 32120 & 58140 & 33984 & 5040 & 0\kern-1em \cr
8 && 1 & 494 & \!19950 & \!195800 & \!644020 & \!785304 & \!341136 &
 \!40320\kern-1em & 0 \cr
\omit&height 2pt\cr}
\hrule width\hsize height.5pt
\kern4pt
\endinsert

We needn't dwell further on Eulerian numbers here; it's usually sufficient
simply to know that they exist, and to have a list of basic identities
to fall back on when the need arises. However, before we leave this topic,
we should take note of yet another triangular pattern of coefficients,
shown in Table~|euler2-triangle|.
We call these ``second-order Eulerian numbers'' $\Euler nk$, because
"!Eulerian numbers, second order" \tabref|nn:euler2|%
they satisfy a recurrence similar to \eq(|eulerian-rec|) but with
$n$ replaced by $2n-1$ in one place:
\begindisplay
\Euler nk=(k+1)\Euler{n-1}k+(2n-1-k)\Euler{n-1}{k-1}\,.
\eqno\eqref|eulerian2-rec|
\enddisplay
These numbers have a curious combinatorial interpretation, first noticed
by "Gessel" and "Stanley"~[|gessel-stanley|]:
If we form permutations of the multiset $\{1,1,\allowbreak
2,2,\allowbreak\ldots,n,n\}$ with
the special property that all numbers between the two occurrences of~$m$
are greater than~$m$, for $1\le m\le n$, then $\Euler nk$ is the
number of such permutations that have $k$ ascents.
For example, there are eight suitable
single-ascent permutations of $\{1,1,2,2,3,3\}$:
\begindisplay
113322,\;
133221,\;
221331,\;
221133,\;
223311,\;
233211,\;
331122,\;
331221.
\enddisplay
Thus $\Euler31=8$. The multiset $\{1,1,2,2,\ldots,n,n\}$ has a total of
\begindisplay
\sum_k\Euler nk=(2n-1)(2n-3)\ldots(1)={(2n)\_n\over2^n}
\eqno
\enddisplay
suitable permutations,
"!product of odd numbers"
because the two appearances of $n$ must be adjacent and there are $2n-1$
places to insert them within a permutation for $n-1$.
For example, when $n=3$ the permutation $1221$ has five insertion points,
yielding $331221$, $133221$, $123321$, $122331$, and $122133$.
Recurrence \eq(|eulerian2-rec|) can be
proved by extending the argument we used for ordinary Eulerian numbers.

Second-order Eulerian numbers are important chiefly because of their connection
with Stirling numbers~[|ginsburg|]: We have, by induction on~$n$,
\begindisplay \openup4pt
{x\brace x-n}&=\sum_k\Euler nk{x+n-1-k\choose2n}\,,
}\hfill{&\qquad\hbox{integer $n\ge0$;}\eqno\eqref|gen-st2|\cr
{x\brack@x-n@}&=\sum_k\Euler nk{x+k\choose2n}\,,
}\hfill{&\qquad\hbox{integer $n\ge0$.}\eqno\eqref|gen-st1|\cr
\enddisplay
For example,
\begindisplay \openup4pt
&{x\brace x{-}1}={x\choose2}\,,}\hfill{\qquad
&{x\brack x{-}1}={x\choose2}\,;\cr
&{x\brace x{-}2}={x{+}1\choose4}+2{x\choose4}\,,}\hfill{\qquad
&{x\brack x{-}2}={x\choose4}+2{x{+}1\choose4}\,;\cr
&{x\brace x{-}3}={x{+2}\choose6}+8{x{+}1\choose6}+6{x\choose6}\,,}\hidewidth{\cr
&&{x\brack x{-}3}={x\choose6}+8{x{+}1\choose6}+6{x{+}2\choose6}\,.
}\hidewidth{\cr
\enddisplay
(We already encountered the case $n=1$ in \eq(|stirl-nn-1|).)
These identities hold whenever $x$ is an integer and $n$ is a nonnegative
integer. Since the right-hand sides are polynomials in~$x$, we can use
\eq(|gen-st2|) and \eq(|gen-st1|)
to define Stirling numbers $x\brace x-n$ and $x\brack x-n$
"!Stirling numbers, generalized"
for arbitrary real (or complex) values of~$x$.

\smallskip
If $n>0$, these polynomials $x\brace x-n$ and $x\brack x-n$ are zero when
$x=0$, $x=1$, \dots, and~$x=n$; therefore they are divisible by
$(x-0)$, $(x-1)$, \dots, and~$(x-n)$. It's interesting to look at what's left
after these known factors are divided out. We define
"!Bernoulli numbers, generalized, \string\see Stirling polynomials"
the {\it"Stirling polynomials"\/} $\sigma_n(x)$ by the rule
\begindisplay
\sigma_n(x)={x\brack x-n}\,\big/\,\bigl(x(x-1)\ldots(x-n)\bigr)\,.
\eqno\eqref|stirl-poly-def|
\enddisplay
(The degree of $\sigma_n(x)$ is $n-1$.) The first few cases are
\g\hbadness=2000\vskip18pt So $1/x$ is a \hbox{polynomial}?
\medskip(Sorry about that.)\g
\begindisplay \openup1pt
\sigma_0(x)&=1/x\,;\cr
\sigma_1(x)&=1/2\,;\cr
\sigma_2(x)&=(3x-1)/24\,;\cr
\sigma_3(x)&=(x^2-x)/48\,;\cr
\sigma_4(x)&=(15x^3-30x^2+5x+2)/5760\,.\cr
\enddisplay
They can be computed via the second-order Eulerian numbers; for example,
\begindisplay
\sigma_3(x)&=\bigl((x{-}4)(x{-}5)+8(x{-}4)(x{+}1)+6(x{+}2)(x{+}1)\bigr)/6!\,.
\enddisplay

\topinsert
\table Stirling convolution formulas.\tabref|stirling-convolutions|
\begindisplay\abovedisplayskip=-2pt \belowdisplayskip=5pt \openup2pt%
 \advance\baselineskip0pt plus .5pt  \advance\lineskip0pt plus .5pt
rs\sum_{k=0}^n\sigma_k(r+tk)\,\sigma_{n-k}(s+t(n-k))&=(r+s)@\sigma_n(r+s+tn)
 \eqno\eqref|st-conv-2a|\cr
s\sum_{k=0}^n \,k@\sigma_k(r+tk)\,\sigma_{n-k}(s+t(n-k))&=n@\sigma_n(r+s+tn)
 \eqno\cr
\noalign{\vskip1pt}
{n\brace m}\hskip4em&\hskip-4em
 =(-1)^{n-m+1}{n!\over(m-1)!}\sigma_{n-m}(-m)\eqno\eqref|s2-to-sigma|\cr
\noalign{\vskip3pt}
{n\brack@m@}\hskip4em&\hskip-4em
 ={n!\over(m-1)!}\sigma_{n-m}(n)\eqno\eqref|s1-to-sigma|\cr
\enddisplay
\kern2pt
\hrule width\hsize height.5pt
\kern4pt
\endinsert

It turns out that these polynomials satisfy two very pretty identities:
\begindisplay
\left(ze^z\over e^z-1\right)^{\!x}&=x\sum_{n\ge0}\sigma_n(x)\,z^n\,;
\eqno\eqref|stirl-poly-gf|\cr
\left({1\over z}\ln{1\over1-z}\right)^{\!x}&=x\sum_{n\ge0}\sigma_n(x+n)\,z^n\,.
\eqno\cr
\enddisplay
And in general, if $\Sscr_t(z)$ is the power series that satisfies
\begindisplay
\ln\bigl(1-z@\Sscr_t(z)^{t-1}\bigr)=-z@\Sscr_t(z)^t\,,
\eqno\eqref|t-series-stirl|
\enddisplay
then
\begindisplay
\Sscr_t(z)^x=x\sum_{n\ge0}\sigma_n(x+tn)\,z^n\,.
\eqno
\enddisplay
Therefore we can obtain general "convolution formulas" for Stirling
"!Stirling numbers, convolutions"
numbers, as we did for binomial coefficients in Table~|conv-table|;
the results appear in Table~|stirling-convolutions|.
When a sum of Stirling numbers doesn't
fit the identities of Table |stirling-id1| or~|stirling-id2|,
Table~|stirling-convolutions| may be just the ticket. (An example appears
later in this chapter, following equation \eq(|stirl-bern|).
Exercise 7.|general-conv-principles| discusses the
general principles of convolutions based on identities like 
\eq(|stirl-poly-gf|) and \thiseq.)

\beginsection 6.3 Harmonic Numbers

It's time now to take a closer look at "harmonic numbers", which we first met
back in Chapter~2:
\begindisplay
H_n=1+{1\over2}+{1\over3}+\cdots+{1\over n}=\sum_{k=1}^n{1\over k}\,,
\quad\hbox{integer $n\ge0$}.
\eqno
\enddisplay
These numbers appear
so often in the analysis of algorithms that computer
scientists need a special notation for them. We use $H_n$, the `$H$' standing
for ``harmonic,\qback'' since a tone of wavelength $1/n$ is called the
$n$th harmonic of a tone whose wavelength is~$1$. The first few values
look like this:
\begindisplay \let\preamble=\tablepreamble \let\strut=\bigstrut%
 \postdisplaypenalty=10000
n&&0&1&2&3&4&5&6&7&8&9&10\cr
\noalign{\hrule}
H_n&&0&1&{3\over2}&{11\over6}&{25\over12}&{137\over60}&{49\over20}&{363\over140}
&{761\over280}&{7129\over2520}&{7381\over2520}\cr
\enddisplay
%The numerators and denominators don't obey any especially
% memorable patterns, so it is pointless to list further values.
Exercise~|prove-h-not-int| shows that $H_n$ is never an integer when $n>1$.

Here's a card trick, based on an idea by R.\thinspace T. Sharp~[|sharp-stack|],
that illustrates how the harmonic numbers arise naturally
in simple situations. Given $n$ cards and a table, we'd like to create
"!card stacking" "!stacking cards"
the largest possible overhang by stacking the cards up over the table's
edge, subject to the laws of "gravity":
\g\vskip.5in \tabref|table-joke| This must be \-Table~|table-joke|.\g
\begindisplay
\unitlength=4pt
\beginpicture(70,18)(-73,-5)
 \put(0,10){\line(-1,0){20.5}}
\put(0,10){\line(0,-1)1}
 \put(0,9){\line(-1,0){30.5}}
\put(-20.5,10){\line(0,-1)1}
  \put(-26.5,11.5){\vector(4,-1)5}
  \put(-30,13){\makebox(0,0){card $1$}}
\put(-10,9){\line(0,-1)1}
 \put(-10,8){\line(-1,0){25.5}}
\put(-30.5,9){\line(0,-1)1}
  \put(-36.5,10.5){\vector(4,-1)5}
  \put(-40,12){\makebox(0,0){card $2$}}
\put(-15,8){\line(0,-1)1}
 \put(-15,7){\line(-1,0){23.833}}
\put(-35.5,8){\line(0,-1)1}
\put(-18.333,7){\line(0,-1)1}
 \put(-18.333,6){\line(-1,0){23.000}}
\put(-38.833,7){\line(0,-1)1}
\put(-20.833,6){\line(0,-1)1}
 \put(-20.833,5){\line(-1,0){22.500}}
\put(-41.333,6){\line(0,-1)1}
\put(-22.833,5){\line(0,-1)1}
 \put(-22.833,4){\line(-1,0){22.167}}
\put(-43.333,5){\line(0,-1)1}
\put(-24.500,4){\line(0,-1)1}
 \put(-24.500,3){\line(-1,0){21.929}}
\put(-45.000,4){\line(0,-1)1}
\put(-25.929,3){\line(0,-1)1}
 \put(-25.929,2){\line(-1,0){21.750}}
\put(-46.429,3){\line(0,-1)1}
\put(-27.179,2){\line(0,-1)1}
 \put(-27.179,1){\line(-1,0){21.611}}
\put(-47.679,2){\line(0,-1)1}
\put(-28.290,1){\line(0,-1)1}
 \put(-28.290,0){\line(-1,0){44.710}}
\put(-48.790,1){\line(0,-1)1}
  \put(-54.790,2.5){\vector(4,-1)5}
  \put(-59.790,3){\makebox(0,0){card $n$}}
\put(-29.290,0){\line(0,-1)6}
\put(0,8){\line(0,-1){12}}
\put(-10,7){\line(0,-1){3}}
\put(-15,6){\line(0,-1){5}}
\put(-7,6){\vector(-1,0)3} \put(-3,6){\vector(+1,0)3}
\put(-5,6){\makebox(0,0){$\mathstrut d_2$}}
\put(-9.5,3){\vector(-1,0){5.5}} \put(-5.5,3){\vector(+1,0){5.5}}
\put(-7.5,3){\makebox(0,0){$\mathstrut d_3$}}
\put(-18.645,-2){\vector(-1,0){10.645}}
\put(-11.045,-2){\vector(+1,0){11.045}}
\put(-14.645,-2){\makebox(0,0){$\mathstrut d_{n+1}$}}
\put(-50,-4){\makebox(0,0){table}}
\endpicture
\enddisplay
To define the problem a bit more, we require the
edges of the cards to be parallel to the edge of the table;
otherwise we could increase the overhang by rotating the cards
so that their corners stick out a little farther. And to make the
answer simpler, we assume that each card is $2$~units long.

With one card, we get maximum overhang when its center of gravity
is just above the edge of the table. The center of gravity is in the
middle of the card, so we can create half a cardlength, or $1$~unit,
of overhang.

With two cards, it's not hard to convince ourselves that we get maximum
overhang when the center of gravity of the top card is just above the
edge of the second card, and the center of gravity of both cards
combined is just above the edge of the table. The joint center of
gravity of two cards will be in the middle of their common part,
so we are able to achieve an additional half unit of overhang.

This pattern suggests a general method, where we place cards so that
the center of gravity of the top~$k$ cards lies just above the edge of the
$k+1$st card (which supports those top~$k$). The
table plays the role of the $n+1$st~card. To
express this condition algebraically, we can let $d_k$ be the distance
from the extreme edge of the top card to the corresponding edge
of the $k$th card from the top. Then $d_1=0$, and we want to make
$d_{k+1}$ the center of gravity of the first $k$ cards:
\begindisplay
d_{k+1}={(d_1+1)+(d_2+1)+\cdots+(d_k+1)\over k}\,,\quad\hbox{for $1\le k\le n$}.
\eqno\eqref|card-stack-rec|
\enddisplay
(The center of gravity of $k$ objects, having respective weights $w_1$,
\dots,~$w_k$ and having respective centers of gravity at
positions $p_1$, \dots~$p_k$, is at position $(w_1p_1+\cdots+w_kp_k)/
(w_1+\cdots+w_k)$.) We can rewrite this recurrence in two equivalent forms
\begindisplay
kd_{k+1}&=k+d_1+\cdots+d_{k-1}+d_k\,,\qquad&\hbox{$k\ge0@$};\cr
(k-1)d_k&=k-1+d_1+\cdots+d_{k-1}\,,\qquad&\hbox{$k\ge1$}.\cr
\enddisplay
\setmathsize{(k-1)d_k=k+d_1+\cdots+d_{k-1}+d_k\,,\qquad}%
Subtracting these equations tells us that
\begindisplay
\mathsize{kd_{k+1}-(k-1)d_k=1+d_k\,,}\hbox{$k\ge1$};
\enddisplay
hence $d_{k+1}=d_k+1/k$. The second card will be offset half a unit
past the third, which is a third of a unit past the fourth, and so on.
The general formula
\begindisplay
d_{k+1}=H_k\eqno
\enddisplay
follows by induction, and if we set $k=n$ we get $d_{n+1}=H_n$ as the total
overhang when $n$ cards are stacked as described.

Could we achieve greater overhang by holding back, not pushing each card
to an extreme position but storing up ``potential gravitational
energy'' for a later advance?
No; any well-balanced card placement has
\begindisplay
d_{k+1}\le{(1+d_1)+(1+d_2)+\cdots+(1+d_k)\over k}\,,\qquad\hbox{$1\le k\le n$}.
\enddisplay
Furthermore $d_1=0$.
It follows by induction that $d_{k+1}\le H_k$.

Notice that it doesn't take too many cards for the top one to be completely
past the edge of the table. We need an overhang of more than one cardlength,
which is $2$~units. The first harmonic number to exceed~$2$ is $H_4={25\over12}$,
so we need only four cards.

And with 52 cards we have an $H_{52}$-unit overhang,
\g Anyone who actually tries to achieve this maximum overhang with 52 cards
is probably not dealing with a full deck\dash---%
or maybe he's~a real joker.\g
which turns out to be $H_{52}/2\approx2.27$ cardlengths. (We will soon
learn a formula that tells us how to compute an approximate value of $H_n$
for large~$n$ without adding up a whole bunch of fractions.)

\smallbreak
An amusing problem called the ``"worm" on the "rubber band"'' shows harmonic
numbers in another guise. A slow but persistent worm, $W$, starts at one
end of a meter-long rubber band and crawls one centimeter per minute toward
the other end. At the end of each minute, an equally persistent keeper
of the band, $K$, whose sole purpose in life is to frustrate~$W$,
stretches it one meter.
"!Kafkaesque scenario"
Thus after one minute of crawling, $W$~is $1$~centimeter from the start
and $99$~from the finish; then $K$ stretches it one meter. During the
stretching operation $W$ maintains his relative position, $1$\% from the
start and $99$\% from the finish; so $W$~is now $2\,$cm from the starting
point and $198\,$cm from the goal. After $W$ crawls for
another minute the score is
$3\,$cm traveled and $197$ to go; but $K$ stretches, and the distances
become $4.5$ and $295.5$. And so on. Does the worm ever reach the finish?
\g Metric units make this problem more scientific.\g
He keeps moving, but the goal seems to move away even faster. (We're
assuming an infinite longevity for $K$ and~$W$, an infinite elasticity of
the band, and an infinitely tiny worm.)

Let's write down some formulas. When $K$ stretches the rubber band, the fraction
of it that $W$ has crawled stays the same. Thus he crawls $1/100$th of it
the first minute, $1/200$th the second, $1/300$th the third, and so on.
After $n$~minutes the fraction of the band that he's crawled is
\begindisplay
{1\over100}\biggl({1\over1}+{1\over2}+{1\over3}+\cdots+{1\over n}\biggr)
={H_n\over100}\,.
\eqno\eqref|worm-ratio|
\enddisplay
So he reaches the finish if $H_n$ ever surpasses~$100$.

We'll see how to estimate $H_n$ for large~$n$ soon; for now, let's simply
check our analysis by considering how ``Superworm'' would perform in the
same situation. Superworm, unlike~$W$,
can crawl $50\,$cm per minute; so she will crawl
$H_n/2$ of the band length after $n$~minutes,
according to the argument we just
gave. If our reasoning is correct, Superworm should finish before $n$
reaches~$4$, since $H_4>2$. And yes, a simple calculation shows that
Superworm has only $33{1\over3}\,$cm left to travel after three minutes
have elapsed. She finishes in 3~minutes and 40~seconds flat.
\g A flatworm, eh?\g

Harmonic numbers appear also in Stirling's triangle. Let's try to
find a closed form for $n\brack2$, the number of permutations of
$n$~objects that have exactly two~cycles. Recurrence \eq(|stirl1-rec|)
tells us that
\begindisplay \openup3pt
{n+1\brack2}&=n{n\brack2}+{n\brack1}\cr
&=n{n\brack2}+(n-1)!\,,\qquad\hbox{if $n>0$};\cr
\enddisplay
and this recurrence is a natural candidate for the "summation factor"
technique of Chapter~2:
\begindisplay
{1\over n!}{n+1\brack 2}={1\over (n-1)!}{n\brack 2}+{1\over n}\,.
\enddisplay
Unfolding this recurrence tells us that ${1\over n!}{n+1\brack2}=H_n$; hence
\begindisplay
{n+1\brack2}=n!H_n\,.
\eqno\eqref|stirl-harmonic|
\enddisplay

We proved in Chapter 2 that the harmonic series $\sum_k1/k$ diverges,
which means that $H_n$ gets arbitrarily large as $n\to\infty$. But
our proof was indirect; we found that a certain infinite sum
\equ(2.|double-harmonic|) gave different answers when it was rearranged,
hence $\sum_k1/k$ could not be bounded. The fact that $H_n\to\infty$
seems counter-intuitive, because it implies among other things that
a large enough stack of cards will overhang a table by a mile or more,
and that the worm $W$ will eventually reach the end of his rope.
Let us therefore take a closer look at the size of~$H_n$ when $n$~is large.

The simplest way to see that $H_n\to\infty$ is probably to group its
terms according to powers of~$2$. We put one term into group~$1$,
two terms into group~$2$,
four into group~$3$,
eight into group~$4$, and so on:
\begindisplay \def\\#1{{\hbox to0pt{\hss$\scriptstyle{\rm group\ }#1$\hss}}}
&\underbrace{{1\over 1}}_\\1
	+\, \underbrace{{1\over 2} {+} {1\over 3}}_\\2
	\,+\, \underbrace{{1\over 4} {+} {1\over 5} {+} {1\over 6}
		{+} {1\over 7}}_\\3
\,+\, \underbrace{{1\over 8} {+} {1\over 9} {+} {1\over 10}
		{+} {1\over 11} {+} {1\over 12} {+} {1\over 13}
		{+} {1\over 14} {+} {1\over 15}}_\\4
	\,+\, \cdots \,.
\enddisplay
Both terms in group~$2$ are between $1\over4$ and $\half$, so the sum of that
group is between $2\cdt{1\over4}=\half$ and $2\cdt\half=1$. All four terms
in group~$3$ are between $1\over8$ and~$1\over4$, so their sum is also
between $\half$ and~$1$. In fact, each of the $2^{k-1}$ terms in group~$k$
is between $2^{-k}$ and $2^{1-k}$; hence the sum of each individual group is
between $\half$ and~$1$.

This grouping procedure tells us that if $n$ is in group~$k$, we must
have $H_n>k/2$ and $H_n\le k$ (by induction on~$k$). Thus $H_n\to\infty$,
and in fact
\begindisplay
{\lfloor \lg n\rfloor+1\over2}<H_n\le\lfloor\lg n\rfloor+1\,.
\eqno
\enddisplay
We now know $H_n$ within a factor of $2$. Although the harmonic numbers
approach infinity, they approach it only logarithmically\dash---that is,
\g We should call them the worm numbers, they're so~slow.\g
quite slowly.

Better bounds can be found with just a little more work and a dose of
calculus. We learned in Chapter~2 that $H_n$ is the discrete analog of
the continuous function $\ln n$. The natural "logarithm" is defined as the
area under a curve, so a geometric comparison is suggested:
\begindisplay
\tabref|nn:ln|
\unitlength=1pt
\beginpicture(210,75)(-20,-10)
\put(0,0){\vector(0,1){60}}
\put(0,0){\vector(1,0){210}}
\put(-10,60){\makebox(0,0){$f(x)$}}
\put(210,-15){\makebox(0,0){$x$}}
\put(30,0){\line(0,1){30}}
\put(60,0){\line(0,1){30}}
\put(90,0){\line(0,1){15}}
\put(120,0){\line(0,1){10}}
\put(150,0){\line(0,1){7.5}}
\put(180,0){\line(0,1){6}}
\put(30,30){\line(1,0){30}}
\put(60,15){\line(1,0){30}}
\put(90,10){\line(1,0){30}}
\put(120,7.5){\line(1,0){30}}
\put(150,6){\line(1,0){30}}
\put(0,-10){\makebox(0,0){$\mathstrut 0$}}
\put(30,-10){\makebox(0,0){$\mathstrut 1$}}
\put(60,-10){\makebox(0,0){$\mathstrut 2$}}
\put(90,-10){\makebox(0,0){$\mathstrut 3$}}
\put(120,-10){\makebox(0,0){$\mathstrut \ldots$}}
\put(150,-10){\makebox(0,0){$\mathstrut n$}}
\put(180,-10){\makebox(0,0){$\mathstrut n{+}1$}}
\put(35,50){\makebox(0,0){$f(x)=1/x$}}
\put(0,0){\squine(22.5,25.7142859,30.0,40.0,34.2857146,30.0)}
\put(0,0){\squine(30.0,33.3333335,37.5,30.0,26.6666665,24.0)}
\put(0,0){\squine(37.5,40.9090905,45.0,24.0,21.818183,20.0)}
\put(0,0){\squine(45.0,48.46154,52.5,20.0,18.4615374,17.142857)}
\put(0,0){\squine(52.5,55.9999976,60.0,17.142857,16.0,15.0)}
\put(0,0){\squine(60.0,63.5294123,67.5,15.0,14.1176476,13.3333334)}
\put(0,0){\squine(67.5,71.052631,75.0,13.3333334,12.6315774,12.0)}
\put(0,0){\squine(75.0,78.57143,82.5,12.0,11.42857,10.9090909)}
\put(0,0){\squine(82.5,86.086953,90.0,10.9090909,10.4347837,10.0)}
\put(0,0){\squine(90.0,93.599997,97.5,10.0,9.5999994,9.2307693)}
\put(0,0){\squine(97.5,101.111122,105.0,9.2307693,8.8888885,8.5714285)}
\put(0,0){\squine(105.0,108.62068,112.5,8.5714285,8.2758633,8.0)}
\put(0,0){\squine(112.5,116.129034,120.0,8.0,7.74193513,7.5)}
\put(0,0){\squine(120.0,123.636362,127.5,7.5,7.2727271,7.0588235)}
\put(0,0){\squine(127.5,131.14286,135.0,7.0588235,6.85714376,6.6666667)}
\put(0,0){\squine(135.0,138.648645,142.5,6.6666667,6.48648757,6.31578946)}
\put(0,0){\squine(142.5,146.15385,150.0,6.31578946,6.1538444,6.0)}
\put(0,0){\squine(150.0,153.658527,157.5,6.0,5.8536586,5.71428573)}
\endpicture
\enddisplay
The area under the curve between $1$ and $n$, which is $\int_1^n\,dx/x=\ln n$,
is less than the area of the $n$~rectangles, which is $\sum_{k=1}^n1/k=H_n$.
Thus $\ln n<H_n$; this is a sharper result than we had in \thiseq.
And by placing the rectangles a little differently, we get a similar upper bound:
\g\noindent\llap{``}I now see a way too how y$\rm^e\!$ aggregate of y$\rm^e\!$ termes of
Musicall progressions may bee found (much after y$\rm^e\!$ same manner)
by Logarithms, but y$\rm^e\!$ calculations for finding out those rules
would bee still more troublesom.''\par\hfill\dash---I. "Newton" [|newton-harm|]\g
\begindisplay
\unitlength=1pt
\beginpicture(210,75)(-20,-10)
\put(0,0){\vector(0,1){60}}
\put(0,0){\vector(1,0){210}}
\put(-10,60){\makebox(0,0){$f(x)$}}
\put(210,-10){\makebox(0,0){$x$}}
\put(30,0){\line(0,1){30}}
\put(60,0){\line(0,1){15}}
\put(90,0){\line(0,1){10}}
\put(120,0){\line(0,1){7.5}}
\put(150,0){\line(0,1){6}}
\put(30,30){\line(-1,0){30}}
\put(60,15){\line(-1,0){30}}
\put(90,10){\line(-1,0){30}}
\put(120,7.5){\line(-1,0){30}}
\put(150,6){\line(-1,0){30}}
\put(0,-10){\makebox(0,0){$\mathstrut 0$}}
\put(30,-10){\makebox(0,0){$\mathstrut 1$}}
\put(60,-10){\makebox(0,0){$\mathstrut 2$}}
\put(90,-10){\makebox(0,0){$\mathstrut 3$}}
\put(120,-10){\makebox(0,0){$\mathstrut \ldots$}}
\put(150,-10){\makebox(0,0){$\mathstrut n$}}
\put(35,50){\makebox(0,0){$f(x)=1/x$}}
\put(0,0){\squine(22.5,25.7142859,30.0,40.0,34.2857146,30.0)}
\put(0,0){\squine(30.0,33.3333335,37.5,30.0,26.6666665,24.0)}
\put(0,0){\squine(37.5,40.9090905,45.0,24.0,21.818183,20.0)}
\put(0,0){\squine(45.0,48.46154,52.5,20.0,18.4615374,17.142857)}
\put(0,0){\squine(52.5,55.9999976,60.0,17.142857,16.0,15.0)}
\put(0,0){\squine(60.0,63.5294123,67.5,15.0,14.1176476,13.3333334)}
\put(0,0){\squine(67.5,71.052631,75.0,13.3333334,12.6315774,12.0)}
\put(0,0){\squine(75.0,78.57143,82.5,12.0,11.42857,10.9090909)}
\put(0,0){\squine(82.5,86.086953,90.0,10.9090909,10.4347837,10.0)}
\put(0,0){\squine(90.0,93.599997,97.5,10.0,9.5999994,9.2307693)}
\put(0,0){\squine(97.5,101.111122,105.0,9.2307693,8.8888885,8.5714285)}
\put(0,0){\squine(105.0,108.62068,112.5,8.5714285,8.2758633,8.0)}
\put(0,0){\squine(112.5,116.129034,120.0,8.0,7.74193513,7.5)}
\put(0,0){\squine(120.0,123.636362,127.5,7.5,7.2727271,7.0588235)}
\put(0,0){\squine(127.5,131.14286,135.0,7.0588235,6.85714376,6.6666667)}
\put(0,0){\squine(135.0,138.648645,142.5,6.6666667,6.48648757,6.31578946)}
\put(0,0){\squine(142.5,146.15385,150.0,6.31578946,6.1538444,6.0)}
\put(0,0){\squine(150.0,153.658527,157.5,6.0,5.8536586,5.71428573)}
\endpicture
\enddisplay
This time the area of the $n$ rectangles, $H_n$, is less than the area of
the first rectangle plus the area under the curve. We have proved that
\begindisplay \postdisplaypenalty=10000
\ln n<H_n<\ln n+1\,,\qquad\hbox{for $n>1$}.
\eqno\eqref|harm-betw-logs|
\enddisplay
We now know the value of $H_n$ with an error of at most~$1$.

``Second order'' harmonic numbers $H_n^{(2)}$ arise when we sum the
"!harmonic numbers, second order"
squares of the reciprocals, instead of summing simply the reciprocals:
\begindisplay
H_n^{(2)}=1+{1\over4}+{1\over9}+\cdots+{1\over n^2}=\sum_{k=1}^n{1\over k^2}\,.
\enddisplay
Similarly, we define harmonic numbers of order $r$ by summing $(-r)$th powers:
\begindisplay
H_n^{(r)}=\sum_{k=1}^n{1\over k^r}\,.
\eqno
\enddisplay
\tabref|nn:hn-gen|%
If $r>1$, these numbers approach a limit as $n\to\infty$; we noted in
exercise 2.|zetaf|
that this limit is conventionally called "Riemann"'s "zeta function":
\begindisplay
\zeta(r)=H_\infty^{(r)}=\sum_{k\ge1}{1\over k^r}\,.
\eqno
\enddisplay

"Euler" [|euler-harmonics|]
discovered a neat way to use generalized harmonic numbers to approximate
"!harmonic numbers, approximate values"
the ordinary ones, $H_n^{(1)}$. Let's consider the infinite series
\begindisplay
\ln\biggl({k\over k-1}\biggr)={1\over k}+{1\over2k^2}+{1\over3k^3}
 +{1\over4k^4}+\cdots\,,
\eqno
\enddisplay
which converges when $k>1$. The left-hand side is
$\ln k-\ln(k-1)$; therefore if we sum both sides
for $2\le k\le n$ the left-hand sum telescopes and we get
\begindisplay  \openup3pt
\ln n-\ln1&=\sum_{k=2}^n\biggl({1\over k}+{1\over2k^2}+{1\over3k^3}
 +{1\over4k^4}+\cdots\,\biggr)\cr
&\tightplus=\textstyle\bigl(H_n{-}1\bigr)+\half\bigl(H_n^{(2)}{-}1\bigr)
 +{1\over3}\bigl(H_n^{(3)}{-}1\bigr)+{1\over4}\bigl(H_n^{(4)}{-}1\bigr)+\cdots\,.
\enddisplay
Rearranging, we have an expression for the difference between $H_n$ and
$\ln n$:
\begindisplay
\textstyle H_n-\ln n=1-\half\bigl(H_n^{(2)}{-}1\bigr)
 -{1\over3}\bigl(H_n^{(3)}{-}1\bigr)-{1\over4}\bigl(H_n^{(4)}{-}1\bigr)-\cdots\,.
\enddisplay
When $n\to\infty$, the right-hand side approaches the limiting value
\begindisplay
\textstyle1-\half\bigl(\zeta(2){-}1\bigr)
 -{1\over3}\bigl(\zeta(3){-}1\bigr)-{1\over4}\bigl(\zeta(4){-}1\bigr)-\cdots\,,
\enddisplay
which is now known as {\it "Euler's constant"\/} and conventionally denoted
by the Greek letter~"$\gamma$". In fact, $\zeta(r)-1$ is approximately
\g\noindent\llap{``}Huius igitur quantitatis constantis\/ $C$ valorem deteximus,
quippe est\/ ${C=0,577218}$.''\par\hfill\dash---L. "Euler" [|euler-harmonics|]\g
$1/2^r$, so this infinite series converges rather rapidly and we can
compute the decimal value
\begindisplay
\gamma=0.5772156649\ldots\,.
\eqno
\enddisplay
Euler's argument establishes the limiting relation
\begindisplay
\lim_{n\to\infty}(H_n-\ln n)=\gamma\,;
\eqno
\enddisplay
thus $H_n$ lies
about 58\% of the way between the two extremes in \eq(|harm-betw-logs|).
We are gradually homing in on its value.

Further refinements are possible, as we will see in Chapter~9. We will
prove, for example, that
\begindisplay
H_n=\ln n+\gamma+{1\over2n}-{1\over12n^2}+{\epsilon_n\over120n^4}\,,
\qquad\hbox{$0<\epsilon_n<1$}.
\eqno
\enddisplay
This formula allows us to conclude that the millionth harmonic number is
\begindisplay
H_{1000000}\approx14.3927267228657236313811275\,,
\enddisplay
without adding up a million fractions. Among other things, this implies
that a stack of a million cards can overhang the edge of a table by more than
seven cardlengths.

What does \thiseq\
tell us about the "worm" on the "rubber band"? Since $H_n$ is
unbounded, the worm will definitely reach the end, when $H_n$ first
exceeds~$100$. Our approximation to $H_n$ says that this will happen
when $n$ is approximately
\begindisplay
e^{100-\gamma}\approx e^{99.423}\,.
\enddisplay
In fact, exercise 9.|worm-finale| proves that the critical value of~$n$
\g\vskip-17pt
 Well, they can't really go at it this long; the world will have ended
much earlier, when the "Tower of Brahma" is fully transferred.\g
is either $\lfloor e^{100-\gamma}\rfloor$ or
$\lceil e^{100-\gamma}\rceil$. We can imagine $W$'s triumph when he
crosses the finish line at last, much to $K$'s chagrin, some
287 decillion centuries after his long crawl began. (The rubber band
will have stretched to more than $10^{27}$ light years long; its
molecules will be pretty far apart.)

\beginsection 6.4 Harmonic Summation

Now let's look at some sums involving harmonic numbers, starting with a review
of a few ideas we learned in Chapter~2. We proved in \equ(2.|harm-sum|)
and \equ(2.|harm-sum+|) that
\begindisplay \openup1pt
\sum_{0\le k<n}H_k&=nH_n-n\,;\eqno\eqref|+harm-sum|\cr
\sum_{0\le k<n}kH_k&={n(n-1)\over2}H_n\;-\;{n(n-1)\over4}\,.\eqno\eqref|+harm-sum+|\cr
\enddisplay
Let's be bold and take on a more general sum, which includes both of these
as special cases: What is the value of
\begindisplay \advance\abovedisplayskip-1.1pt\advance\belowdisplayskip-1.1pt
\sum_{0\le k<n}{k\choose m}H_k\,,
\enddisplay
when $m$ is a nonnegative integer?

The approach that worked best for \eq(|+harm-sum|) and \eq(|+harm-sum+|) in
Chapter~2 was called {\it "summation by parts"}. We wrote the summand in the
form $u(k)\Delta v(k)$, and we applied the general identity
\begindisplay
\sum\nolimits_a^b u(x)\Delta v(x)\,\delta x=
u(x)v(x)\big\vert_a^b\;-\;\sum\nolimits_a^b v(x+1)\Delta u(x)\,\delta x\,.
\eqno
\enddisplay
Remember? The sum that faces us now, $\sum_{0\le k<n}{k\choose m}H_k$,
is a natural for this method because we can let
\begindisplay \openup4pt
u(k)&=H_k\,,}\hfill{\qquad \qquad\Delta u(k)&=H_{k+1}-H_k={1\over k+1}\,;\cr
v(k)&={k\choose m{+}1}\,,}\hfill{\qquad\qquad \Delta v(k)&={k{+}1\choose m{+}1}-
 {k\choose m{+}1}={k\choose m}\,.\cr
\enddisplay
(In other words, harmonic numbers have a simple $\Delta$
and binomial coefficients have
a simple $\Delta^{-1}$, so we're in business.) Plugging into \thiseq\ yields
\begindisplay
\sum_{0\le k<n}\!{k\choose m}H_k=\!\sum\nolimits_0^n{x\choose m}H_x\,\delta x
&={x\choose m{+}1}H_x\bigg\vert_0^n-\sum\nolimits_0^n{x{+}1\choose m{+}1}
 {\delta x\over x{+}1}\cr
&={n\choose m{+}1}H_n\,-\!\sum_{0\le k<n}\!{k{+}1\choose m{+}1}{1\over k{+}1}\,.\cr
\enddisplay
The remaining sum is easy, since we can absorb the $(k+1)^{-1}$ using
our old standby, equation \equ(5.|bc-absorb|):
\begindisplay
\sum_{0\le k<n}{k+1\choose m+1}{1\over k+1}
=\sum_{0\le k<n}{k\choose m}{1\over m+1}
={n\choose m+1}{1\over m+1}\,.
\enddisplay
Thus we have the answer we seek:
\begindisplay
\sum_{0\le k<n}{k\choose m}H_k={n\choose m+1}\biggl(H_n-{1\over m+1}\biggr)\,.
\eqno\eqref|harm-sum++|
\enddisplay
(This checks nicely
with \eq(|+harm-sum|) and \eq(|+harm-sum+|) when $m=0$ and $m=1$.)

The next example sum uses division instead of multiplication: Let us
try to evaluate
\begindisplay
S_n=\sum_{k=1}^n{H_k\over k}\,.
\enddisplay
If we expand $H_k$ by its definition, we obtain a double sum,
\begindisplay
S_n=\sum_{1\le j\le k\le n}{1\over j\cdot k}\,.
\enddisplay
Now another method from Chapter 2 comes to our aid; equation
\equ(2.|upper-triangle|) tells us that
\begindisplay
S_n=\half \Biggl(\biggl(\sum_{k=1}^n {1\over k}\biggr)^2
 +\sum_{k=1}^n {1\over k^2}\Biggr)
=\half\bigl(H_n^2+H_n^{(2)}\bigr)\,.
\eqno\eqref|harm/k|
\enddisplay
It turns out that we could also
have obtained this answer in another way if we had tried
to sum by parts (see exercise |sum-triangle-by-parts|).

Now let's try our hands at a more difficult problem [|ungar|], which doesn't
submit to summation by parts:
\begindisplay
U_n=\sum_{k\ge1}{n\choose k}{(-1)^{k-1}\over k}(n-k)^n\,,
\qquad\hbox{integer $n\ge1$}.
\enddisplay
(This sum doesn't explicitly mention harmonic numbers either;
\g(Not to give the answer away or anything.)\g
but who knows when they might turn up?)

We will solve this problem in two ways, one by grinding out the answer and
the other by being clever and/or lucky. First, the grinder's approach.
We expand $(n-k)^n$ by the binomial theorem, so that the troublesome
$k$ in the denominator will combine with the numerator:
\begindisplay
U_n&=\sum_{k\ge1}{n\choose k}{(-1)^{k-1}\over k}\sum_j{n\choose j}(-k)^jn^{n-j}\cr
&=\sum_j{n\choose j}(-1)^{j-1}n^{n-j}\sum_{k\ge1}{n\choose k}(-1)^k k^{j-1}\,.\cr
\enddisplay
This isn't quite the mess it seems, because the $k^{j-1}$ in the inner sum
is a polynomial in~$k$, and identity \equ(5.|nth-diff|) tells us that we are
simply taking the "$n$th difference" of this polynomial. Almost; first we must
clean up a few things. For one, $k^{j-1}$ isn't a polynomial if $j=0$; so we
will need to split off that term and handle it separately. For another,
we're missing the term $k=0$ from the formula for $n$th difference; that
term is nonzero when $j=1$, so we had better restore it (and subtract it out again).
The result is
\begindisplay
U_n&=\sum_{j\ge1}{n\choose j}(-1)^{j-1}n^{n-j}\sum_{k\ge0}{n\choose k}(-1)^kk^{j-1}\cr
&\qquad -\sum_{j\ge1}{n\choose j}(-1)^{j-1}n^{n-j}{n\choose 0}0^{j-1}\cr
\noalign{\vskip1pt}
&\qquad -{n\choose 0}n^n\sum_{k\ge1}{n\choose k}(-1)^kk^{-1}\,.\cr
\enddisplay
OK, now the top line (the only remaining double sum) is zero: It's
the sum of multiples of $n$th differences of polynomials of degree less
than~$n$,
and such $n$th differences are zero. The second line is zero except when
$j=1$, when it equals $-n^n$. So the third line is the only residual
difficulty; we have reduced the original problem to a much simpler sum:
\begindisplay
U_n=n^n(T_n-1)\,,\qquad\hbox{where }
T_n=\sum_{k\ge1}{n\choose k}{(-1)^{k-1}\over k}\,.
\eqno\eqref|tn-def|
\enddisplay
For example, $U_3={3\choose1}{8\over1}-{3\choose2}{1\over2}
={45\over2}$; $T_3={3\choose1}{1\over1}-{3\choose2}{1\over2}+{3\choose3}{1\over3}
={11\over6}$; hence $U_3=27(T_3-1)$ as claimed.

How can we evaluate $T_n$? One way is to replace $n\choose k$ by
${n-1\choose k}+{n-1\choose k-1}$, obtaining a simple recurrence for $T_n$
in terms of $T_{n-1}$. But there's a more instructive way: We had a
similar formula in \equ(5.|recip-bc|), namely
\begindisplay
\sum_k{n\choose k}{(-1)^k\over x+k}= {n!\over x(x+1)\ldots(x+n)}\,.
\enddisplay
\looseness=-1
If we subtract out the term for $k=0$ and set $x=0$, we get $-T_n$.
So let's do~it:
\begindisplay \openup5pt
T_n&=\biggl({1\over x}-{n!\over x(x+1)\ldots(x+n)}\biggr)\,\bigg\vert_{x=0}\cr
&=\biggl({(x+1)\ldots(x+n)-n!\over x(x+1)\ldots(x+n)}\biggr)\,\bigg\vert_{x=0}\cr
&=\biggl({x^n{n+1\brack n+1}+\cdots+x{n+1\brack2}+{n+1\brack1}-n!
   \over x(x+1)\ldots(x+n)}\biggr)\,\bigg\vert_{x=0}
={1\over n!}{n+1\brack2}\,.
\enddisplay
(We have used the expansion \eq(|expand-rising-to-ord|)
of $(x+1)\ldots(x+n)=x\_^{n+1}\!/x$; we can divide $x$ out of the numerator
because ${n+1\brack1}=n!$.)
But we know from
\eq(|stirl-harmonic|) that ${n+1\brack2}=n!\,H_n$; hence $T_n=H_n$, and
we have the answer:
\begindisplay
U_n=n^n(H_n-1)\,.
\eqno
\enddisplay

That's one approach. The other approach will be to try to evaluate a much
more general sum,
\begindisplay
U_n(x,y)=\sum_{k\ge1}{n\choose k}{(-1)^{k-1}\over k}(x+ky)^n\,,
\qquad\hbox{integer $n\ge0$}@;
\eqno\eqref|unxy-def|
\enddisplay
the value of the original $U_n$ will drop out as the special case $U_n(n,-1)$.
(We are encouraged to try for more generality because the
previous derivation ``threw away'' most of the details of the given problem;
somehow those details
must be irrelevant, because the $n$th difference wiped them away.)

We could replay the previous derivation with small changes
and discover the value of $U_n(x,y)$. Or we could replace $(x+ky)^n$
by $(x+ky)^{n-1}(x+ky)$ and then replace $n\choose k$
by ${n-1\choose k}+{n-1\choose k-1}$, leading to the recurrence
\begindisplay
U_n(x,y)=xU_{n-1}(x,y)+x^n\!/n+yx^{n-1}\,;
\eqno\eqref|unxy-rec|
\enddisplay
this can readily be solved with a summation factor (exercise |alt-un|).

But it's easiest to use another trick that worked to our advantage
in Chapter~2: differentiation.
The derivative of
$U_n(x,y)$ with respect to~$y$ brings out a $k$ that cancels with the
$k$ in the denominator, and the resulting sum is trivial:
\begindisplay
{\partial\over\partial y}U_n(x,y)
&=\sum_{k\ge1}{n\choose k}(-1)^{k-1}n(x+ky)^{n-1}\cr
\noalign{\vskip2pt}
&={n\choose0}nx^{n-1}-\sum_{k\ge0}{n\choose k}(-1)^kn(x+ky)^{n-1}
 =nx^{n-1}\,.
\enddisplay
(Once again, the $n$th difference of a polynomial of degree $<n$ has
vanished.)

We've proved that the derivative of $U_n(x,y)$ with respect 
to~$y$ is $nx^{n-1}$, independent of~$y$. In general, if $f'(y)=c$ then
$f(y)=f(0)+cy$; therefore
 we must have $U_n(x,y)=U_n(x,0)+nx^{n-1}y$.

The remaining task is to determine $U_n(x,0)$.
But $U_n(x,0)$ is just $x^n$ times the sum
$T_n=H_n$ we've already considered in \eq(|tn-def|);
therefore the general sum in \eq(|unxy-def|) has the closed form
\begindisplay \postdisplaypenalty=10000
U_n(x,y)=x^nH_n+nx^{n-1}y\,.
\eqno
\enddisplay
In particular, the solution to the original problem is $U_n(n,-1)=n^n(H_n-1)$.

\beginsection 6.5 Bernoulli Numbers

The next important sequence of numbers on our agenda is named after
Jakob "Bernoulli" (1654--1705), who discovered curious relationships
while working out the formulas for sums of $m$th powers~[|bernoulli-ars|].
Let's write
\begindisplay
S_m(n)=0^m+1^m+\cdots+(n-1)^m=\sum_{k=0}^{n-1}\,k^m=\sum\nolimits_0^nx^m\,\delta x\,.
\eqno
\enddisplay
(Thus, when $m>0$ we have $S_m(n)=H_{n-1}^{(-m)}$ in the notation
of "generalized harmonic numbers".)
Bernoulli looked at the following sequence of formulas and
spotted a pattern:
\begindisplay \let\displaystyle=\textstyle \openup2pt%
 \def\preamble{\hfill$S_{##}(n)={}$&&\hfill$##$&$n^{##}$&${}##{}$}
0&&\cr
1&{1\over2}&2&-&\half&\cr
2&{1\over3}&3&-&\half&2&+&{1\over6}&\cr
3&{1\over4}&4&-&\half&3&+&{1\over4}&2\cr
4&{1\over5}&5&-&\half&4&+&{1\over3}&3&-&{1\over30}&\cr
5&{1\over6}&6&-&\half&5&+&{5\over12}&4&-&{1\over12}&2&\cr
6&{1\over7}&7&-&\half&6&+&{1\over2}&5&-&{1\over6}&3&+&{1\over42}&\cr
7&{1\over8}&8&-&\half&7&+&{7\over12}&6&-&{7\over24}&4&+&{1\over12}&2&\cr
8&{1\over9}&9&-&\half&8&+&{2\over3}&7&-&{7\over15}&5&+&{2\over9}&3&-&{1\over30}&\cr
9&{1\over10}&10&-&\half&9&+&{3\over4}&8&-&{7\over10}&6&+&{1\over2}&4&-&{3\over20}&2\cr
10&{1\over11}&11&-&\half&10&+&{5\over6}&9&-&&7&+&&5&-&{1\over2}&3&+&{5\over66}&\cr
\enddisplay
Can you see it too? The coefficient of $n^{m+1}$ in $S_m(n)$ is always $1/(m+1)$.
"!sum of consecutive cubes"
"!sum of consecutive $m$th powers"
The coefficient of $n^m$ is always $-1/2$. The coefficient of~$n^{m-1}$ is always
\dots\ let's see \dots~$m/12$. The coefficient of $n^{m-2}$ is always zero.
The coefficient of $n^{m-3}$ is always \dots\ let's see \dots\ hmmm \dots~yes,
it's $-m(m{-}1)(m{-}2)/720$.
The coefficient of $n^{m-4}$ is always zero. And it looks
as if the pattern will continue, with the coefficient of $n^{m-k}$ always
being some constant times $m\_k$.

That was Bernoulli's empirical discovery. (He did not give a proof.)
In modern notation we write the coefficients
in the form
\begindisplay \openup3pt
S_m(n)&={1\over m+1}\biggl(B_0\,n^{m+1}+{m+1\choose1}B_1\,n^m+\cdots+{m+1\choose m}B_m\,n\bigg)\cr
&={1\over m+1}\sum_{k=0}^m{m+1\choose k}B_k\,n^{m+1-k}\,.
\eqno\eqref|sm-bern|
\enddisplay

Bernoulli numbers are defined by an "implicit recurrence" relation,
\begindisplay
\sum_{j=0}^m{m+1\choose j}B_j=\[m=0]\,,\qquad\hbox{for all $m\ge0$}.
\eqno\eqref|bern-def|
\enddisplay
For example, ${\2\choose0}B_0+{\2\choose1}B_1=0$.
The first few values turn out to be
\begindisplay \let\preamble=\tablepreamble \let\strut=\bigstrut
n&&0&1&2&3&4&5&6&7&8&9&10&11&12\cr
\noalign{\hrule}
B_n&&1&-\half&{1\over6}&0&{-1\over30}&0&{1\over42}&0&{-1\over30}&0&{5\over66}
&0&{-691\over2730}\cr
\enddisplay
(All conjectures about a simple closed form for $B_n$ are wiped out by
the appearance of the strange fraction $-691/2730$.)

We can prove Bernoulli's formula \eq(|sm-bern|) by induction on~$m$,
using the "perturbation method"
"!sum of consecutive squares"
(one of the ways we found $S_2(n)=\Sq_n$ in Chapter~2):
\begindisplay
S_{m+1}(n)+n^{m+1}&=\sum_{k=0}^{n-1}\,(k+1)^{m+1}\cr
&=\sum_{k=0}^{n-1}\,\sum_{j=0}^{m+1}{m+1\choose j}k^j
 =\sum_{j=0}^{m+1}{m+1\choose j}S_j(n)\,.
\eqno
\enddisplay
Let $\widehat S_m(n)$ be the right-hand side of \eq(|sm-bern|); we wish
to show that $S_m(n)=\widehat S_m(n)$, assuming that $S_j(n)=\widehat S_j(n)$
for $0\le j<m$. We begin as we did for $m=2$ in Chapter~2, subtracting $S_{m+1}(n)$
from both sides of~\thiseq. Then we expand
each $S_j(n)$ using \eq(|sm-bern|),
and regroup so that the coefficients of powers of~$n$
on the right-hand side are brought together and simplified:
\begindisplay \openup3pt
n^{m+1}&=\sum_{j=0}^m{m+1\choose j}S_j(n)=\sum_{j=0}^m{m+1\choose j}\widehat S_j
 (n)+{m+1\choose m}\,\Delta\cr
&=\sum_{j=0}^m{m+1\choose j}{1\over j+1}\sum_{k=0}^j{j+1\choose k}B_kn^{j+1-k}+(m+1)\,\Delta\cr
&=\sum_{0\le k\le j\le m}{m+1\choose j}{j+1\choose k}{B_k\over j+1}n^{j+1-k}+(m+1)\,\Delta\cr
&=\sum_{0\le k\le j\le m}{m+1\choose j}{j+1\choose j-k}{B_{j-k}\over j+1}n^{k+1}+(m+1)\,\Delta\cr
&=\sum_{0\le k\le j\le m}{m+1\choose j}{j+1\choose k+1}{B_{j-k}\over j+1}n^{k+1}+(m+1)\,\Delta\cr
&=\sum_{0\le k\le m}{n^{k+1}\over k+1}\sum_{k\le j\le m}{m+1\choose j}{j\choose 
k}B_{j-k}+(m+1)\,\Delta\cr
&=\sum_{0\le k\le m}{n^{k+1}\over k+1}{m+1\choose k}\!\!
 \sum_{k\le j\le m}\!\!{m{+}1{-}k\choose j-k}B_{j-k}+(m+1)\,\Delta\cr
&=\sum_{0\le k\le m}{n^{k+1}\over k+1}{m+1\choose k}\!\!
 \sum_{0\le j\le m-k}\!\!{m{+}1{-}k\choose j}B_j+(m+1)\,\Delta\cr
&=\sum_{0\le k\le m}{n^{k+1}\over k+1}{m+1\choose k}\[m-k=0]+(m+1)\,\Delta\cr
&={n^{m+1}\over m+1}{m+1\choose m}+(m+1)\,\Delta\cr
&=n^{m+1}+(m+1)\,\Delta\,,\qquad\hbox{where $\Delta=S_m(n)-\widehat S_m(n)$.}\cr
\enddisplay
(This derivation is a good review of the standard manipulations
we learned in Chapter 5.) Thus $\Delta=0$ and $S_m(n)=\widehat S_m(n)$, QED.

In Chapter 7 we'll use generating functions
\g Here's some more neat stuff that you'll probably want to skim
through the first time.\par\hfill\dash---Friendly TA\par
\bigskip
\unitlength=1pc
\beginpicture(6,2)(0,0)
\put(.7,2){\line(1,0)5}\put(1,2){\vector(0,-1)2}\endpicture
\vskip-2pc \qquad Start\par\qquad Skimming\g
 to obtain a much simpler proof
of \eq(|sm-bern|). The key idea will be to show that the Bernoulli numbers
are the coefficients of the power series
\begindisplay
{z\over e^z-1}=\sum_{n\ge0}B_n{z^n\over n!}\,.
\eqno\eqref|bern-gf|
\enddisplay
Let's simply assume for now that equation \thiseq\ holds,
so that we can derive some of its amazing consequences.
If we add $\half z$ to both sides,
thereby cancelling the term $B_1z/1!=-\half z$
from the right, we get
\begindisplay
{z\over e^z-1}+{z\over2}&={z\over2}\,{e^z+1\over e^z-1}
 ={z\over2}\,{e^{z/2}+e^{-z/2}\over e^{z/2}-e^{-z/2}}
={z\over2}\coth{z\over2}\,.
\eqno\eqref|z2cothz2|
\enddisplay
Here coth is the ``hyperbolic cotangent'' function, otherwise known in
calculus books as $\cosh z/\!\sinh z$; we have
\begindisplay\advance\abovedisplayskip-.5pt\advance\belowdisplayskip-.5pt
\sinh z={e^z-e^{-z}\over2}\,;\qquad
\cosh z={e^z+e^{-z}\over2}\,.
\eqno
\enddisplay
Changing $z$ to $-z$ gives
$\bigl({-z\over2}\bigr)\coth\bigl({-z\over2}\bigr)=
{z\over2}\coth{z\over2}$; hence every odd-numbered coefficient of
${z\over2}\coth{z\over2}$ must be zero, and we have
\begindisplay
B_3=B_5=B_7=B_9=B_{11}=B_{13}=\cdots=0\,.
\eqno\eqref|odd-bern|
\enddisplay
Furthermore \eq(|z2cothz2|) leads to a closed form for the coefficients of coth:
\begindisplay
z\coth z={2z\over e^{2z}-1}+{2z\over2}=\sum_{n\ge0}B_{2n}{(2z)^{2n}\over(2n)!}
=\sum_{n\ge0}4^nB_{2n}{z^{2n}\over(2n)!}\,.
\eqno
\enddisplay
{\displaywidowpenalty=-200 % force a desirable page break (DEK May 88)
But there isn't much of a market for "hyperbolic functions"; people are
more interested in the ``real'' functions of trigonometry.
We can express ordinary "trigonometric functions" in terms of their
hyperbolic cousins by using the rules
\begindisplay
\sin z=-i\sinh iz\,,\qquad\cos z=\cosh iz\,;
\eqno
\enddisplay
the corresponding power series are}
\begindisplay \openup3pt
\sin z&={z^1\over1!}-{z^3\over3!}+{z^5\over5!}-\cdots\,,}\hfill{\qquad
\sinh z&={z^1\over1!}+{z^3\over3!}+{z^5\over5!}+\cdots\,;\cr
\cos z&={z^0\over0!}-{z^2\over2!}+{z^4\over4!}-\cdots\,,}\hfill{\qquad
\cosh z&={z^0\over0!}+{z^2\over2!}+{z^4\over4!}+\cdots\,.
\enddisplay
Hence $\cot z=\cos z/\!\sin z=i\cosh iz/\sinh iz=i\coth iz$,
\g I see, we get ``real'' functions by using imaginary numbers.\g
and we have
\begindisplay
z\cot z=\sum_{n\ge0}B_{2n}{(2iz)^{2n}\over(2n)!}=\sum_{n\ge0}(-4)^nB_{2n}{z^{2n}\over(2n)!}\,.
\eqno\eqref|cot-bern|
\enddisplay
Another remarkable
formula for $z\cot z$ was found by "Euler" (exercise |prove-cot-poles|):
\begindisplay
z\cot z=1-2\sum_{k\ge1}{z^2\over k^2\pi^2-z^2}\,.
\eqno\eqref|cot-poles|
\enddisplay
We can expand Euler's formula in powers of $z^2$, obtaining
\begindisplay \openup4pt
z\cot z&=1-2\sum_{k\ge1}\biggl({z^2\over k^2\pi^2}+{z^4\over k^4\pi^4}+
 {z^6\over k^6\pi^6}+\cdots\,\biggr)\cr
&=1-2\biggl({z^2\over\pi^2}H_\infty^{(2)}+{z^4\over\pi^4}H_\infty^{(4)}+
 {z^6\over\pi^6}H_\infty^{(6)}+\cdots\,\biggr)\,.\cr
\enddisplay
Equating coefficients of $z^{2n}$ with those in our other formula,
 \eq(|cot-bern|), gives
us an almost miraculous closed form for infinitely many infinite sums:
\begindisplay
\zeta(2n)=H_\infty^{(2n)}=(-1)^{n-1}{2^{2n-1}\pi^{2n}B_{2n}\over(2n)!}\,,
\qquad\hbox{integer $n>0$.}
\eqno\eqref|bern-zeta|
\enddisplay
For example,
\begindisplay \openup3pt
\zeta(2)&=H_\infty^{(2)}=\textstyle 1+{1\over4}+{1\over9}+\cdots=
\pi^2B_2=\pi^2\!/6\,;\eqno\cr
\zeta(4)&=H_\infty^{(4)}=\textstyle 1+{1\over16}+{1\over81}+\cdots=
-\pi^4B_4/3=\pi^4\!/90\,.\eqno\cr
\enddisplay
Formula \eq(|bern-zeta|) is not only a closed form for $H_\infty^{(2n)}$,
it also tells us the approximate size of $B_{2n}$, since $H_\infty^{(2n)}$
is very near~$1$ when $n$ is large. And it tells us that
$(-1)^{n-1}B_{2n}>0$ for all $n>0$; thus
 the nonzero Bernoulli numbers alternate in sign.

And that's not all. Bernoulli numbers also appear in the coefficients
\g\vskip.5in\unitlength=1pc
\beginpicture(6,2)(0,0)
\put(.7,2){\line(1,0)5}\put(1,2){\vector(0,-1)2}\endpicture
\vskip-2pc \qquad Start\par\qquad Skipping\g
of the "tangent" function,
\begindisplay
\tan z={\sin z\over\cos z}
=\sum_{n\ge0}(-1)^{n-1}4^n(4^n-1)B_{2n}{z^{2n-1}\over(2n)!}\,,
\eqno\eqref|tan-series|
\enddisplay
as well as other trigonometric functions (exercise |trig-bern|).
Formula \thiseq\ leads to another important fact about the Bernoulli numbers,
namely that 
\begindisplay
T_{2n-1}=(-1)^{n-1}{4^n(4^n-1)\over2n}\,B_{2n}\quad\hbox{is a positive integer}.
\eqno\eqref|bern-tan|
\enddisplay
We have, for example:
\begindisplay \let\preamble=\tablepreamble
n&&1&3&5&7&9&11&13\cr
\noalign{\hrule}
T_n&&1&2&16&272&7936&353792&22368256\cr
\enddisplay
(The $T$'s are called {\it "tangent numbers"}.)

One way to prove \thiseq, following an idea of B.\thinspace F. "Logan",
is to consider the power series
\begindisplay \openup3pt
{\sin z+x\cos z\over\cos z-x\sin z}
&=x+(1{+}x^2)z+(2x^3{+}2x){z^2\over2}+(6x^4{+}8x^2{+}2){z^3\over6}+\cdots\cr
&=\sum_{n\ge0}T_n(x){z^n\over n!}\,,
\eqno\eqref|tan-poly-gf|
\enddisplay
where $T_n(x)$ is a polynomial in $x$; setting $x=0$ gives
\g When $x=\tan w$, this is $\tan(z+w)$. Hence, by "Taylor"'s theorem,
the $n$th derivative of $\tan w$ is $T_n(\tan w)$.\g
$T_n(0)=T_n$, the~$n$th tangent number. If we differentiate \thiseq\
with respect to~$x$, we get
\begindisplay
{1\over(\cos z-x\sin z)^2}=\sum_{n\ge0}T'_n(x){z^n\over n!}\,;
\enddisplay
but if we differentiate with respect to $z$, we get
\begindisplay
{1+x^2\over(\cos z-x\sin z)^2}=\sum_{n\ge1}T_n(x){z^{n-1}\over(n-1)!}
=\sum_{n\ge0}T_{n+1}(x){z^n\over n!}\,.
\enddisplay
(Try it\dash---the cancellation is very pretty.) Therefore we have
\begindisplay
T_{n+1}(x)=(1+x^2)T'_n(x)\,,\qquad T_0(x)=x\,,
\eqno\eqref|tan-rec|
\enddisplay
a simple recurrence from which it follows that the coefficients of
$T_n(x)$ are nonnegative integers. Moreover, we can easily prove that
$T_n(x)$~has degree $n+1$, and that its coefficients are alternately
zero and positive. Therefore $T_{2n+1}(0)=T_{2n+1}$ is a positive
integer, as claimed in \eq(|bern-tan|).

Recurrence \eq(|tan-rec|) gives us a simple way to calculate Bernoulli
numbers, via tangent numbers, using only simple operations on integers;
"!Bernoulli numbers, calculation of"
%\g(This is discussed further in [|knuth-buckholtz|].)\g
by contrast, the defining recurrence
\eq(|bern-def|) involves difficult arithmetic with fractions.

\smallbreak
If we want to compute the sum of $n$th powers from $a$ to $b-1$ instead
of from $0$ to $n-1$, the theory of Chapter~2 tells us that
\begindisplay
\sum_{k=a}^{b-1}\,k^m=\sum\nolimits_a^b x^m\,\delta x=S_m(b)-S_m(a)\,.
\eqno
\enddisplay
This identity has interesting consequences when we consider negative values
of\/~$k$: We have
\begindisplay
\sum_{k=-n+1}^{-1}k^m=(-1)^m\sum_{k=0}^{n-1}\,k^m\,,\qquad\hbox{when $m>0$},
\enddisplay
hence
\begindisplay
S_m(0)-S_m(-n+1)=(-1)^m\bigl(S_m(n)-S_m(0)\bigr)\,.
\enddisplay
But $S_m(0)=0$, so we have the identity
\begindisplay
S_m(1-n)=(-1)^{m+1}S_m(n)\,,\qquad\hbox{$m>0$}.
\eqno
\enddisplay
\g \vskip-30pt
Johann Faulhaber implicitly used \thiseq\ in 1635~[|faulhaber|]
to find simple formulas for $S_m(n)$ as polynomials in
$n(n+1)/2$ when $m\le17$; see [|knuth-faul|].)\g
Therefore $S_m(1)=0$. If we write the polynomial $S_m(n)$ in factored form,
it will always have the factors $n$ and~$(n-1)$, because it has the roots
$0$ and~$1$. In general, $S_m(n)$ is a polynomial of degree $m+1$ with
leading term~${1\over m+1}n^{m+1}$.
 Moreover, we can set $n=\half$ in \thiseq\ to deduce that
$S_m(\half)=(-1)^{m+1}S_m(\half)$; if $m$ is even, this makes $S_m(\half)=0$,
so $(n-\half)$ will be an additional factor. These observations
 explain why we found the simple factorization
\begindisplay
S_2(n)=\textstyle{1\over3}n(n-\half)(n-1)
\enddisplay
in Chapter~2; we could have used such reasoning to deduce the value
of $S_2(n)$ without calculating it!
"!sum of consecutive squares"
Furthermore, \thiseq\ implies that the polynomial with the remaining factors,
$\hat S_m(n)=S_m(n)/(n-\half)$, always satisfies
\begindisplay
\hat S_m(1-n)=\hat S_m(n)\,,\qquad\hbox{$m$ even,\quad $m>0$.}
\enddisplay
It follows that $S_m(n)$ can always be written in the factored form
\begindisplay
S_m(n)=\cases{
\displaystyle{1\over m+1}\prod_{k=1}^{\lceil m/2\rceil}
 \textstyle\,(n-\half-\alpha_k)(n-\half+\alpha_k)\,,&$m$ odd;\cr
\noalign{\vskip6pt}
\displaystyle{(n-\half)\over m+1}\prod_{k=1}^{m/2}
 \textstyle\,(n-\half-\alpha_k)(n-\half+\alpha_k)\,,&$m$ even.\cr}
\eqno\eqref|bern-factors|
\enddisplay
Here $\alpha_1=\half$, and $\alpha_2$, \dots, $\alpha_{\lceil m/2\rceil}$
are appropriate complex numbers whose values depend on~$m$.
For example,
\begindisplay \let\displaystyle=\textstyle \openup3pt
S_3(n)&=n^2(n-1)^2\!/4\,;\cr
S_4(n)&=n(n{-}\half)(n{-}1)(n-\half+\sqrt{7/12}\,)(n-\half-\sqrt{7/12}\,)/5\,;\cr
S_5(n)&=n^2(n-1)^2(n-\half+\sqrt{3/4}\,)(n-\half-\sqrt{3/4}\,)/6\,;\cr
S_6(n)&=n(n{-}\half)(n{-}1)(n{-}\half+\alpha)(n{-}\half-\alpha)
 (n{-}\half+\overline\alpha)(n{-}\half-\overline\alpha)\,,\cr
&\qquad\hbox{where $\alpha=\displaystyle2^{-5/2\,}3^{-1/2\,}31^{1/4}\bigl(
 \sqrt{\sqrt{31}+\sqrt{27}}+i\,\sqrt{\sqrt{31}-\sqrt{27}}\,\bigr).$}\cr
\enddisplay
If $m$ is odd and greater than $1$, we have $B_m=0$; hence $S_m(n)$ is
divisible by~$n^2$ (and by $(n-1)^2$). Otherwise the roots of $S_m(n)$
don't seem to obey a simple law.
\g\unitlength=1pc
\beginpicture(6,2)(0,0)
\put(.7,0){\line(1,0)5}\put(1,2){\vector(0,-1)2}\endpicture
\vskip-2.1pc \qquad Stop\par\qquad Skipping\g

\smallbreak
Let's conclude our study of Bernoulli numbers by looking at how they
relate to "Stirling numbers". One way to compute $S_m(n)$ is to
change ordinary powers to falling powers, since the falling powers
have easy sums. After doing those easy sums we can convert back to
ordinary powers:
\begindisplay\openup4pt\def\\#1{^{\raise.5pt\hbox{$\scriptstyle\underline{#1}$}}}
S_m(n)=\sum_{k=0}^{n-1}\,k^m&=\sum_{k=0}^{n-1}\sum_{j\ge0}{m\brace j}k\\j
 =\sum_{j\ge0}{m\brace j}\sum_{k=0}^{n-1}\,k\\j\cr
&=\sum_{j\ge0}{m\brace j}{n\\{j+1}\over j+1}\cr
&=\sum_{j\ge0}{m\brace j}{1\over j+1}\sum_{k\ge0}(-1)^{j+1-k}{j+1\brack k}n^k\,.
\enddisplay
Therefore, equating coefficients with those in
\eq(|sm-bern|), we must have the identity
\begindisplay
\sum_{j\ge0}{m\brace j}{j+1\brack k}{(-1)^{j+1-k}\over j+1}
={1\over m+1}{m+1\choose k}B_{m+1-k}\,.
\eqno\eqref|stirl-stirl-bern|
\enddisplay
It would be nice to prove this relation directly, thereby discovering Bernoulli
numbers in a new way. But the identities in Tables |stirling-id1|
or~|stirling-id2| don't give us any obvious handle on a proof by induction that
the left-hand sum in \thiseq\ is a constant times $m\_{k-1}$. If $k=m+1$,
the left-hand sum is just ${m\brace m}{m+1\brack m+1}/(m{+}1)=1/(m{+}1)$, so
that case is easy. And if $k=m$, the left-hand side sums to ${m\brace m-1}
{m\brack m}m^{-1}-{m\brace m}{m+1\brack m}(m+1)^{-1}=\half(m-1)-\half m
=-\half$; so that case is pretty easy too. But if $k<m$, the left-hand
sum looks hairy. Bernoulli would probably not have discovered his numbers
if he had taken this route.

One thing we can do is replace $m\brace j$ by
${m+1\brace j+1}-(j+1){m\brace j+1}$.
The $(j+1)$ nicely cancels with the awkward denominator, and the left-hand
side becomes
\begindisplay
\sum_{j\ge0}{m+1\brace j+1}{j+1\brack k}{(-1)^{j+1-k}\over j+1}
\;-\;\sum_{j\ge0}{m\brace j+1}{j+1\brack k}(-1)^{j+1-k}\,.
\enddisplay
The second sum is zero, when $k<m$, by \eq(|stirl-inv|). That leaves us with
the first sum, which cries out for a change in notation; let's rename all
variables so that the index of summation is~$k$, and so that the other
parameters are $m$ and~$n$. Then identity \thiseq\ is equivalent to
\begindisplay
\sum_k{n\brace k}{k\brack m}{(-1)^{k-m}\over k}=
{1\over n}{n\choose m}B_{n-m}\,+\,\[m=n-1]\,.
\eqno\eqref|stirl-bern|
\enddisplay
Good, we have something that looks more pleasant\dash---although
Table~|stirling-id2| still doesn't suggest any obvious next step.

The convolution formulas in Table |stirling-convolutions| now come
to the rescue. We can use \eq(|s1-to-sigma|) and~\eq(|s2-to-sigma|) to
"!Stirling convolution"
rewrite the summand in terms of "Stirling polynomials":
\begindisplay \openup4pt \tightplus
{n\brace k}{k\brack m}
&=(-1)^{n-k+1}{n!\over(k-1)!}\sigma_{n-k}(-k)
 \cdot{k!\over(m-1)!}\sigma_{k-m}(k)\,;\cr
{n\brace k}{k\brack m}{(-1)^{k-m}\over k}
&=(-1)^{n+1-m}{n!\over(m-1)!}\,\sigma_{n-k}(-k)\,\sigma_{k-m}(k)\,.
\enddisplay
Things are looking up; the convolution in \eq(|st-conv-2a|), with $t=1$, yields
\begindisplay \openup3pt
\sum_{k=0}^n\sigma_{n-k}(-k)\,\sigma_{k-m}(k)
&=\sum_{k=0}^{n-m}\sigma_{n-m-k}\bigl(-n+(n{-}m{-}k)\bigr)\,\sigma_k(m+k)\cr
&={m-n\over(m)(-n)}\sigma_{n-m}\bigl(m-n+(n{-}m)\bigr)\,.
\enddisplay
Formula \eq(|stirl-bern|) is now verified, and we find that
 Bernoulli numbers are
 related to the constant terms in the Stirling polynomials:
\g\unitlength=1pc
\beginpicture(6,2)(0,0)
\put(.7,0){\line(1,0)5}\put(1,2){\vector(0,-1)2}\endpicture
\vskip-2.1pc \qquad Stop\par\qquad Skimming\g
\begindisplay
{B_m\over m!}=-m\sigma_m(0)\,.
\eqno\eqref|sigma-at-0|
\enddisplay

\beginsection 6.6 Fibonacci Numbers

Now we come to a special sequence of numbers that is perhaps the most
pleasant of all, the Fibonacci sequence $\<F_n\>$:
\begindisplay \let\preamble=\tablepreamble
n&&0&1&2&3&4&5&6&7&8&9&10&11&12&13&14\cr
\noalign{\hrule}
F_n&&0&1&1&2&3&5&8&13&21&34&55&89&144&233&377\cr
\enddisplay
Unlike the harmonic numbers and the Bernoulli numbers, the "Fibonacci
numbers" are nice simple integers. They are defined by the recurrence
\begindisplay \openup2pt
F_0&=0\,;\cr
F_1&=1\,;\cr
F_n&=F_{n-1}+F_{n-2}\,,\qquad\hbox{for $n>1$}.
\eqno\eqref|fib-rec|
\enddisplay
The simplicity of this rule\dash---the simplest possible recurrence in which
each number depends on the previous two\dash---accounts for the fact
that Fibonacci numbers occur in a wide variety of situations.

``"Bee trees"'' provide a good example of how Fibonacci numbers can
\g The back-to-nature nature of this example is shocking.
This book should be banned.\g
arise naturally. Let's consider the pedigree of a male bee. Each
male (also known as a drone) is produced asexually from a female
(also known as a queen); each female, however, has two parents,
a male and a female. Here are the first few levels of the tree:
\begindisplay
\unitlength=2pt
%\setbox0=\hbox{\thicklines\beginpicture(0,0)(0,0) % male bee
%	\put(0,0){\circle{6}}
%	\put(7.121,7.121){\line(-1,-1){5}} % too big for Lyn
%	\put(7.121,7.121){\line(-1,0){4}}
%	\put(7.121,7.121){\line(0,-1){4}}
%	\endpicture}
\setbox0=\hbox{\thicklines\beginpicture(0,0)(0,0) % male bee
	\put(0,0){\circle{6}}
	\put(2.051,2.191){\hbox{\smallln\char0}}
	\put(2.191,2.051){\hbox{\smallln\char0}}
	\put(4.621,4.621){\line(-1,0){2.5}}
	\put(4.621,4.621){\line(0,-1){2.5}}
	\endpicture}
\setbox2=\hbox{\thicklines\beginpicture(0,0)(0,0) % female bee
	\put(0,0){\circle{6}}
	\put(0,-7.5){\line(0,1){4.5}}
	\put(-1.5,-5.25){\line(1,0){3}}
	\endpicture}
\beginpicture(140,72)(-63,0)
\put(0,0){\copy0}
\put(0,16){\copy2}
\put(-45,30){\copy0}
\put(45,30){\copy2}
\put(-45,46){\copy2}
\put(20,46){\copy0}
\put(70,46){\copy2}
\put(-60,62){\copy0}
\put(-30,62){\copy2}
\put(20,62){\copy2}
\put(55,62){\copy0}
\put(85,62){\copy2}
\put(0,4){\line(0,1){3.5}}
\put(-45,34){\line(0,1){3.5}}
\put(20,50){\line(0,1){3.5}}
\put(0,20){\line(0,1){10}}
\put(45,34){\line(0,1){12}}
\put(-45,50){\line(0,1){12}}
\put(70,50){\line(0,1){12}}
\put(-41,30){\line(1,0){82}}
\put(24,46){\line(1,0){42}}
\put(-56,62){\line(1,0){22}}
\put(59,62){\line(1,0){22}}
\endpicture
\enddisplay
The "drone" has one grandfather and one grandmother; he has one
great-grandfather and two great-grandmothers; he has two
great-great-grandfathers and three great-great-grandmothers.
In general, it is easy to see by induction that he has exactly
$F_{n+1}$ great$^n$-grandpas and $F_{n+2}$ great$^n$-grandmas.

Fibonacci numbers are often found in nature, perhaps for reasons
similar to the bee-tree law. For example,
a typical sunflower has a large head that contains spirals of tightly packed
florets, usually with 34 winding in one direction and 55 in another.
\g "Phyllotaxis", n.\par The love of taxis.\g
Smaller heads will have 21 and~34, or 13 and 21; a gigantic "sunflower"
with 89 and 144 spirals was once exhibited in England.
Similar patterns are found in some species of pine cones.

And here's an example of a different nature [|moser-reflections|]:
Suppose we put two panes of glass back-to-back. How many ways
$a_n$ are there for light rays to pass through or be reflected
"!reflected light rays"
after changing direction $n$ times? The first few cases are:
\begindisplay
\unitlength=.8pt % Oren had .85, but it was a bit too wide
\beginpicture(384,64)(0,-24)
\thicklines
\put(0,0){\line(1,0){384}}
\put(0,16){\line(1,0){384}}
\put(0,32){\line(1,0){384}}
\put(11,40){\vector(1,-4){12}}
\put(44,40){\line(1,-4){6}}
\put(50,16){\vector(1,4){6}}
\put(62,40){\line(1,-4){10}}
\put(72,0){\vector(1,4){10}}
\put(109,40){\line(1,-4){6}}
\put(115,16){\line(1,4){4}}
\put(119,32){\vector(1,-4){10}}
\put(132,40){\line(1,-4){10}}
\put(142,0){\line(1,4){4}}
\put(146,16){\vector(1,-4){6}}
\put(155,40){\line(1,-4){10}}
\put(165,0){\line(1,4){8}}
\put(173,32){\vector(1,-4){10}}
\put(204,40){\line(1,-4){6}}
\put(210,16){\line(1,4){4}}
\put(214,32){\line(1,-4){4}}
\put(218,16){\vector(1,4){6}}
\put(230,40){\line(1,-4){6}}
\put(236,16){\line(1,4){4}}
\put(240,32){\line(1,-4){8}}
\put(248,0){\vector(1,4){10}}
\put(264,40){\line(1,-4){10}}
\put(274,0){\line(1,4){4}}
\put(278,16){\line(1,-4){4}}
\put(282,0){\vector(1,4){10}}
\put(298,40){\line(1,-4){10}}
\put(308,0){\line(1,4){8}}
\put(316,32){\line(1,-4){4}}
\put(320,16){\vector(1,4){6}}
\put(332,40){\line(1,-4){10}}
\put(342,0){\line(1,4){8}}
\put(350,32){\line(1,-4){8}}
\put(358,0){\vector(1,4){10}}
\thinlines
\put(20,-20){\makebox(0,0){$\thickmuskip=\thinmuskip a_0=1$}}
\put(76,-20){\makebox(0,0){$\thickmuskip=\thinmuskip a_1=2$}}
\put(152,-20){\makebox(0,0){$\thickmuskip=\thinmuskip a_2=3$}}
\put(278,-20){\makebox(0,0){$\thickmuskip=\thinmuskip a_3=5$}}
\put(0,28){\line(6,1){24}}
\put(0,24){\line(6,1){48}}
\put(0,20){\line(6,1){72}}
\multiput(0,16)(24,0){13}{\line(6,1){96}}
\put(312,16){\line(6,1){72}}
\put(336,16){\line(6,1){48}}
\put(360,16){\line(6,1){24}}
\put(0,4){\line(6,-1){24}}
\put(0,8){\line(6,-1){48}}
\put(0,12){\line(6,-1){72}}
\multiput(0,16)(24,0){13}{\line(6,-1){96}}
\put(312,16){\line(6,-1){72}}
\put(336,16){\line(6,-1){48}}
\put(360,16){\line(6,-1){24}}
\endpicture
\enddisplay
"!optical illusion"
When $n$ is even, we have an even number of bounces and the ray passes
through; when $n$ is odd, the ray is reflected and it re-emerges on the
same side it entered. The $a_n$'s seem to be Fibonacci numbers, and a little
staring at the figure tells us why: For $n\ge2$, the $n$-bounce rays
either take their first bounce off the opposite surface and continue in $a_{n-1}$
ways, or they begin by bouncing off the middle surface and then bouncing
 back again to finish
in $a_{n-2}$ ways. Thus we have the Fibonacci recurrence
$a_n=a_{n-1}+a_{n-2}$. The initial conditions are different, but not very
different, because
we have $a_0=1=F_2$ and $a_1=2=F_3$; therefore everything is simply shifted two
places, and $a_n=F_{n+2}$.

Leonardo "Fibonacci" introduced these numbers in 1202, and mathematicians
gradually began to discover more and more interesting things about them.
\'Edouard "Lucas", the perpetrator of the Tower of Hanoi puzzle discussed
\g\noindent\llap{``}La suite de Fibonacci poss\`ede des propri\'et\'es nombreuses fort
int\'eressantes.''\par\hfill\dash---E. Lucas [|lucas-theorie|]\g
in Chapter~1, worked with them extensively in the last half of the
nineteenth~century (in fact it was Lucas who popularized the name
``Fibonacci numbers''). One of his amazing results was to use properties
of Fibonacci numbers to prove that the 39-digit "Mersenne number"
$2^{127}-1$ is prime.

\def\\#1{^{\mskip2mu#1}}
One of the oldest theorems about Fibonacci numbers, due to the French
astronomer Jean-Dominique
"Cassini" in 1680 [|cassini|], is the identity
\begindisplay
F_{n+1\,}F_{n-1}-F_n\\2=(-1)^n\,,\qquad\hbox{for $n>0$}.
\eqno\eqref|cassini-id|
\enddisplay
When $n=6$, for example, "Cassini's identity" correctly claims that
$13\cdt5-8^2=1$.

A polynomial formula that involves Fibonacci numbers of the form $F_{n\pm k}$
for small values of\/~$k$ can be transformed into a formula that involves
only $F_n$ and $F_{n+1}$, because we can use the rule
\begindisplay
F_m=F_{m+2}-F_{m+1}
\eqno\eqref|fm-raise|
\enddisplay
to express $F_m$ in terms of higher Fibonacci numbers when $m<n$, and we can use
\begindisplay
F_m=F_{m-2}+F_{m-1}
\eqno\eqref|fm-lower|
\enddisplay
to replace $F_m$ by lower Fibonacci numbers when $m>n+1$. Thus, for example,
we can replace $F_{n-1}$ by $F_{n+1}-F_n$ in \eq(|cassini-id|) to get
"Cassini's identity" in the form
\begindisplay
F_{n+1}\\2-F_{n+1\,}F_n-F_n\\2=(-1)^n\,.
\eqno\eqref|cassini-n+1-n|
\enddisplay
Moreover, Cassini's identity reads
\begindisplay
F_{n+2\,}F_n-F_{n+1}\\2=(-1)^{n+1}
\enddisplay
when $n$ is replaced by~$n+1$; this is the same as
$(F_{n+1}+F_n)F_n-F_{n+1}\\2=(-1)^{n+1}$,
which is the same as \thiseq. Thus Cassini$(n)$ is true if and only if
Cassini$(n{+}1)$ is true; equation \eq(|cassini-id|) holds for all~$n$ by induction.

Cassini's identity is the basis of a geometrical paradox that was one
of Lewis "Carroll"'s favorite puzzles [|carroll-pictures|],\thinspace[|schlomilch|],%
\thinspace[|weaver|].
The idea is to take a chessboard and cut it into four pieces as shown here,
then to reassemble the pieces into a rectangle:
\begindisplay \advance\abovedisplayskip2pt
\unitlength=.0033333in
\beginpicture(320,320)(0,0)
\multiput(40,0)(40,0)7{\line(0,1){320}}
\multiput(0,40)(0,40)7{\line(1,0){320}}
\put(0,0){\squine(0,160,320,0,0,0)}
\put(0,0){\squine(0,0,0,0,160,320)}
\put(0,0){\squine(0,160,320,320,320,320)}
\put(0,0){\squine(320,320,320,0,160,320)}
\put(0,0){\squine(200,200,200,0,160,320)}
\put(0,0){\squine(0,100,200,200,160,120)}
\put(0,0){\squine(202,259,316,0,160,320)}
\endpicture
\hskip6em
\beginpicture(512,320)(0,-60)
\put(42,0){\line(0,1){199}}
\put(82,0){\line(0,1){199}}
\put(122,0){\line(0,1){199}}
\put(161,0){\line(0,1){199}}
\put(199,0){\line(0,1){199}}
\put(237,0){\line(0,1){199}}
\put(275,0){\line(0,1){199}}
\put(313,0){\line(0,1){199}}
\put(351,0){\line(0,1){199}}
\put(390,0){\line(0,1){199}}
\put(430,0){\line(0,1){199}}
\put(470,0){\line(0,1){199}}
\put(0,41){\line(1,0){512}}
\put(0,80){\line(1,0){512}}
\put(0,119){\line(1,0){512}}
\put(0,158){\line(1,0){512}}
\put(0,0){\squine(0,0,0,0,99.5,199)}
\put(0,0){\squine(512,512,512,0,99.5,199)}
\put(0,0){\squine(0,256,512,0,0,0)}
\put(0,0){\squine(0,256,512,199,199,199)}
\put(0,0){\squine(0,256,512,195,99.5,4)}
\put(0,0){\squine(199,199,199,0,60.5,121)}
\put(0,0){\squine(313,313,313,199,138.5,78)}
\endpicture
\enddisplay
"!optical illusion"
Presto: The original area of $8\times8=64$ squares has been rearranged
to yield $5\times13=65$ squares! A similar construction dissects any
\g The paradox is explained because \dots well, "magic" tricks aren't supposed
to be explained.\g
$F_n\times F_n$ square into four pieces, using $F_{n+1}$, $F_n$, $F_{n-1}$,
and~$F_{n-2}$ as dimensions wherever the illustration has
$13$, $8$, $5$, and $3$ respectively. The result is an $F_{n-1}\times
F_{n+1}$ rectangle; by \eq(|cassini-id|), one square has therefore been
gained or lost, depending on whether $n$ is even or odd.

Strictly speaking, we can't apply the reduction \eq(|fm-lower|) unless $m\ge2$,
because we haven't defined $F_n$ for negative~$n$. A lot of maneuvering
becomes easier if we eliminate this boundary condition and use \eq(|fm-raise|)
and~\eq(|fm-lower|) to define Fibonacci numbers with negative indices. For example,
$F_{-1}$ turns out to be $F_1-F_0=1$; then $F_{-2}$ is $F_0-F_{-1}=-1$. In
this way we deduce the values
\begindisplay \let\preamble=\tablepreamble
n&&0&-1&-2&-3&-4&-5&-6&-7&-8&-9&-10&-11\cr
\noalign{\hrule}
F_n&&0&1&-1&2&-3&5&-8&13&-21&34&-55&89\cr
\enddisplay
and it quickly becomes clear (by induction) that
\begindisplay \postdisplaypenalty=10000
F_{-n}=(-1)^{n-1}F_n\,,\qquad\hbox{integer $n$}.
\eqno\eqref|f-n|
\enddisplay
"Cassini's identity" \eq(|cassini-id|) is true for {\it all\/} integers~$n$,
not just for $n>0$, when we extend the Fibonacci sequence in this way.

The process of reducing $F_{n\pm k}$ to a combination of $F_n$ and~$F_{n+1}$
by using \eq(|fm-lower|) and \eq(|fm-raise|) leads to the sequence of formulas
\begindisplay \def\\{\hskip5em&}
&F_{n+2}=\phantom1F_{n+1}+\phantom1F_n\\F_{n-1}=\phantom1F_{n+1}&-\phantom1F_n\cr
&F_{n+3}=2F_{n+1}+\phantom1F_n\\F_{n-2}=-F_{n+1}&+2F_n\cr
&F_{n+4}=3F_{n+1}+2F_n\\F_{n-3}=\phantom02F_{n+1}&-3F_n\cr
&F_{n+5}=5F_{n+1}+3F_n\\F_{n-4}=-3F_{n+1}&+5F_n\cr
\enddisplay
in which another pattern becomes obvious:
\begindisplay
F_{n+k}=F_kF_{n+1}+F_{k-1}F_n\,.
\eqno\eqref|fn+k|
\enddisplay
This identity, easily proved by induction, holds for all integers
$k$ and~$n$ (positive, negative, or zero).

If we set $k=n$ in \thiseq, we find that
\begindisplay
F_{2n}=F_nF_{n+1}+F_{n-1}F_n\,;
\eqno\eqref|f2n|
\enddisplay
hence $F_{2n}$ is a multiple of $F_n$. Similarly,
\begindisplay
F_{3n}=F_{2n}F_{n+1}+F_{2n-1}F_n\,,
\enddisplay
and we may conclude that $F_{3n}$ is also a multiple of $F_n$. By induction,
\begindisplay
\hbox{$F_{kn}$ is a multiple of\/ $F_n$}\,,
\eqno\eqref|fmn|
\enddisplay
for all integers $k$ and $n$. This explains, for example, why $F_{15}$
(which equals $610$) is a multiple of both $F_3$ and~$F_5$ (which are equal to
$2$ and~$5$). Even more is true, in fact; exercise~|prove-fib-gcd| proves
that
\begindisplay
\gcd(F_m,F_n)=F_{\gcd(m,n)}\,.
\eqno\eqref|fib-gcd|
\enddisplay
For example, $\gcd(F_{12},F_{18})=\gcd(144,2584)=8=F_6$.

We can now prove a converse of \eq(|fmn|): If $n>2$ and if $F_m$ is a
multiple of $F_n$, then $m$ is a multiple of~$n$. For if $F_n\divides F_m$
then $F_n\divides\gcd(F_m,F_n)=F_{\gcd(m,n)}\le F_n$. This is possible
only if $F_{\gcd(m,n)}=F_n$; and our assumption that $n>2$ makes it
mandatory that $\gcd(m,n)=n$. Hence $n\divides m$.

An extension of these divisibility ideas was used by Yuri "Matijasevich"
in his famous proof [|mat-ich|] that there
is no algorithm to
decide if a given multivariate polynomial equation with integer coefficients
has a solution in integers.
Matijasevich's lemma states that, if $n>2$, the Fibonacci number~$F_m$ is
a multiple of $F_n\\2$ if and only if $m$ is a multiple of $nF_n$.

Let's prove this by looking at the sequence $\<F_{kn}\bmod F_n\\2\>$ for $k=1$,~%
$2$, $3$,~\dots, and seeing when $F_{kn}\bmod F_n\\2=0$. (We know that
$m$ must have the form $kn$ if $F_m\bmod F_n=0$.)
First we have $F_n\bmod F_n\\2=F_n$; that's not zero. Next we have
\begindisplay
F_{2n}=F_nF_{n+1}+F_{n-1}F_n\=2F_nF_{n+1}\pmod{F_n\\2}\,,
\enddisplay
by \eq(|fn+k|), since $F_{n+1}\=F_{n-1}$ \tmod{F_n}. Similarly
\begindisplay
F_{2n+1}=F_{n+1}\\2+F_n\\2\=F_{n+1}\\2\pmod{F_n\\2}\,.
\enddisplay
This congruence allows us to compute
\begindisplay
F_{3n}&=F_{2n+1}F_n+F_{2n}F_{n-1}\cr
&\=F_{n+1}\\2F_n+(2F_nF_{n+1})F_{n+1}=3F_{n+1}\\2F_n\;&\pmod{F_n\\2}\,;\cr
\noalign{\smallskip}
F_{3n+1}&=F_{2n+1}F_{n+1}+F_{2n}F_n\cr
&\=F_{n+1}\\3+(2F_nF_{n+1})F_n\=F_{n+1}\\3\;&\pmod{F_n\\2}\,.\cr
\enddisplay
In general, we find by induction on $k$ that
\begindisplay
F_{kn}\=kF_nF_{n+1}\\{k-1}\And F_{kn+1}\=F_{n+1}\\k
\pmod{F_n\\2}\,.
\enddisplay
Now $F_{n+1}$ is relatively prime to $F_n$, so
\begindisplay
F_{kn}\=0\pmod{F_n\\2}&\iff kF_n\=0\pmod{F_n\\2}\cr
 &\iff k\=0\pmod{F_n}\,.
\enddisplay
We have proved Matijasevich's lemma.

\smallbreak
One of the most important properties of the Fibonacci numbers is the
special way in which they can be used to represent integers. Let's write
\begindisplay
j\gg k\qquad\iff\qquad j\ge k+2\,.
\eqno
\enddisplay
Then {\sl every positive integer has a unique representation
of the form}
\begindisplay
n=F_{k_1}+F_{k_2}+\cdots+F_{k_r}\,,\qquad
\hbox{$k_1\gg k_2\gg \cdots\gg k_r\gg0$}.
\eqno\eqref|zeck|
\enddisplay
(This is ``"Zeckendorf"'s theorem'' [|zeck1|],\thinspace[|zeck2|].) For
example, the rep\-resenta\-tion of one million turns out to be
\begindisplay\def\preamble{\hfill\displaymath##{}$&&$\hfil##\hfil$&${}+{}##$}
1000000=&832040&&121393&&46368&&144&&55\cr
=&F_{30}&&F_{26}&&F_{24}&&F_{12}&&F_{10}\rlap{\thinspace.}\cr
\enddisplay
We can always find such a representation by using a ``"greedy"'' approach, 
choosing $F_{k_1}$ to be the largest Fibonacci number $\le n$, then
choosing $F_{k_2}$ to be the largest that is $\le n-F_{k_1}$, and so on.
(More precisely, suppose that $F_k\le n<F_{k+1}$; then we have
$0\le n-F_k<F_{k+1}-F_k=F_{k-1}$. If $n$ is a Fibonacci number, \thiseq\
holds with $r=1$ and $k_1=k$. Otherwise $n-F_k$ has a Fibonacci representation
$F_{k_2}+\cdots+F_{k_r}$, by induction on~$n$; and \thiseq\ holds if
we set $k_1=k$, because the inequalities $F_{k_2}\le n-F_k<F_{k-1}$
imply that $k\gg k_2$.)
Conversely, any representation of the form \thiseq\ implies that
\begindisplay
F_{k_1}\le n<F_{k_1+1}\,,
\enddisplay
because the largest possible value of $F_{k_2}+\cdots+F_{k_r}$ when
$k\gg k_2\gg\cdots\gg k_r\gg0$ is
\begindisplay
F_{k-2}+F_{k-4}+\cdots+F_{k\,{\rm mod}\,2+2}=F_{k-1}-1\,,
\qquad\hbox{if $k\ge2$}.
\eqno\eqref|10101-sum|
\enddisplay
(This formula is easy to prove by induction on~$k$; the left-hand side
is zero when $k$ is $2$ or~$3$.) Therefore $k_1$ is the greedily chosen value
described earlier, and the representation must be unique.

Any unique system of representation is a "number system"; therefore
Zeckendorf's theorem leads to the {\it "Fibonacci number system"}.
We can represent any nonnegative integer $n$ as a sequence of $0$'s and $1$'s, writing
\begindisplay
n=(b_mb_{m-1}\ldots b_2)_F\;\iff\;n=\sum_{k=2}^m b_kF_k\,.
\eqno
\enddisplay
This number system is something like binary (radix~$2$) notation, except
that there never are two adjacent~$1$'s. For example, here are the numbers
from $1$ to~$20$, expressed Fibonacci-wise:
\begindisplay \def\preamble{&\hfil$##={}$&$(##)_F$\qquad}
1&000001&6&001001&11&010100&16&100100\cr
2&000010&7&001010&12&010101&17&100101\cr
3&000100&8&010000&13&100000&18&101000\cr
4&000101&9&010001&14&100001&19&101001\cr
5&001000&10&010010&15&100010&20&101010\cr
\enddisplay
The Fibonacci representation of a million, shown a minute ago, can be
contrasted with its binary representation $2^{19}+2^{18}+2^{17}+2^{16}+2^{14}+2^9
+2^6$:
\begindisplay
(1000000)_{10}&=(10001010000000000010100000000)_F\cr
&=(11110100001001000000)_2\,.
\enddisplay
The Fibonacci representation needs a few more bits because adjacent $1$'s
are not permitted; but the two representations are analogous.

%To add $1$ in the Fibonacci number system, we can place `$.11$' at the right;
%this effectively adds $F_1+F_0=1$, because the digit immediately to the
%left of the radix point corresponds to $F_2$. Then we can ``carry'' as
%much as necessary by changing three adjacent digits `$011$' by `$100$',
%as many times as necessary until there are no two $1$'s in a row. Finally,
%the digits to the right of the radix point can be eliminated because
%they will be either $.00$ or $.01$, both of which represent $0$.
To add $1$ in the Fibonacci number system, there are two cases:
If the ``units digit'' is $0$, we change it to $1$; that adds $F_2=1$, since
the units digit refers to $F_2$. Otherwise the two least significant digits
will be $01$, and we change them to $10$ (thereby adding $F_3-F_2=1$).
Finally, we must ``carry'' as much as necessary by changing the digit
pattern `$011$' to `$100$' until there are no two $1$'s in a row.
(This carry rule is equivalent to replacing $F_{m+1}+F_m$ by $F_{m+2}$.)
For example, to go from $5=(1000)_F$ to $6=(1001)_F$ or from
$6=(1001)_F$ to $7=(1010)_F$ requires no carrying; but to go from
$7=(1010)_F$ to $8=(10000)_F$ we must carry twice.

\smallskip
So far we've been discussing lots of properties of the Fibonacci numbers,
but we haven't come up with a closed formula for them. We haven't found
closed forms for Stirling numbers, Eulerian numbers, or Bernoulli numbers
either; but we were able to discover the closed form $H_n={n+1\brack2}/n!$
for harmonic numbers. Is there a relation between $F_n$ and other
quantities we know? Can we ``solve'' the recurrence that
defines~$F_n$?

The answer is yes. In fact, there's a simple way to solve the recurrence
by using the idea of {\it"generating function"\/} that we looked at
\g\noindent\llap{``}Sit\/ $1+x+2xx+3x^3+5x^4+8x^5+13x^6+21x^7+34x^8$\&c Series
nata ex divisione Unitatis per Trinomium\par $1-x-xx$.''\par
\hfill\dash---A. de~Moivre~[|de-moivre|]\par\smallskip
\noindent\llap{``}The quantities\/ $r$, $s$,~$t$, which show the relation of the terms,
are the same as those in the denominator of the fraction. This property,
howsoever obvious it may be, M. "DeMoivre" was the first that applied it
to use, in the solution of problems about infinite series, which otherwise
would have been very intricate.''\par\hfill\dash---J. "Stirling"
[|stirling-method|]\g
briefly in Chapter~5. Let's consider the infinite series
\begindisplay
F(z)=F_0+F_1z+F_2z^2+\cdots=\sum_{n\ge0}F_nz^n\,.
\eqno
\enddisplay
If we can find a simple formula for $F(z)$, chances are reasonably good
that we can find a simple formula for its coefficients $F_n$.

In Chapter 7 we will focus on generating functions in detail, but it will
be helpful to have this example under our belts by the time we get there.
The power series $F(z)$ has a nice property if we look at what happens
when we multiply it by $z$ and by $z^2$:
\begindisplay
F(z)&=F_0\,+\,{}&F_1z\,+\,{}&F_2z^2\,+\,F_3z^3\,+\,F_4z^4\,+\,F_5z^5\,+\,\cdots\,,\cr
zF(z)&=&F_0z\,+\,{}&F_1z^2\,+\,F_2z^3\,+\,F_3z^4\,+\,F_4z^5\,+\,\cdots\,,\cr
z^2F(z)&=&&F_0z^2\,+\,F_1z^3\,+\,F_2z^4\,+\,F_3z^5\,+\,\cdots\,.\cr
\enddisplay
If we now subtract the last two equations from the first, the
terms that involve $z^2$, $z^3$, and higher powers of~$z$ will all
disappear, because of the Fibonacci recurrence. Furthermore the
constant term $F_0$ never actually appeared in the first place,
because $F_0=0$. Therefore all that's left after the subtraction is
$(F_1-F_0)z$, which is just~$z$. In other words,
\begindisplay
F(z)-zF(z)-z^2F(z)=z\,,
\enddisplay
and solving for $F(z)$ gives us the compact formula
\begindisplay
F(z)={z\over1-z-z^2}\,.
\eqno\eqref|fib-gf|
\enddisplay

We have now boiled down all the information in the Fibonacci sequence to
a simple (although unrecognizable) expression $z/(1-z-z^2)$. This, believe it
or not, is progress, because we can factor the denominator and then use
partial fractions to achieve a formula that we can easily expand in power
series. The coefficients in this power series will be a closed form
for the Fibonacci numbers.

The plan of attack just sketched can perhaps be understood better if
we approach it backwards. If we have a simpler generating function,
say $1/(1-\alpha z)$ where $\alpha$ is a constant, we know the
coefficients of all powers of~$z$, because
\begindisplay
{1\over1-\alpha z}=1+\alpha z+\alpha^2z^2+\alpha^3z^3+\cdots\,.
\enddisplay
Similarly, if we have a generating function of the form $A/(1-\alpha z)
+B/(1-\beta z)$, the coefficients are easily determined, because
\begindisplay \openup3pt
{A\over1-\alpha z}+{B\over1-\beta z}&=A\sum_{n\ge0}(\alpha z)^n
+B\sum_{n\ge0}(\beta z)^n \cr
&= \sum_{n\ge0}(A\alpha^n+B\beta^n)z^n\,.
\eqno\eqref|fib-step0|
\enddisplay
Therefore all we have to do is find constants $A$, $B$, $\alpha$, and~$\beta$
such that
\begindisplay
{A\over1-\alpha z}+{B\over1-\beta z}={z\over1-z-z^2}\,,
\enddisplay
and we will have found a closed form $A\alpha^n+B\beta^n$ for the coefficient
$F_n$ of $z^n$ in $F(z)$. The left-hand side can be rewritten
\begindisplay
{A\over1-\alpha z}+{B\over1-\beta z}={A-A\beta z+B-B\alpha z\over
(1-\alpha z)(1-\beta z)}\,,
\enddisplay
so the four constants we seek are the solutions to two polynomial equations:
\begindisplay \openup4pt
&(1-\alpha z)(1-\beta z)=1-z-z^2\,;\eqno\eqref|fib-step1|\cr
&(A+B)-(A\beta+B\alpha)z=z\,.\eqno\eqref|fib-step2|
\enddisplay
We want to factor the denominator of $F(z)$ into the form
$(1-\alpha z)(1-\beta z)$; then we will be able to express $F(z)$ as the sum of
two fractions in which the factors
$(1-\alpha z)$ and $(1-\beta z)$ are conveniently separated from each other.

Notice that the denominator factors in \eq(|fib-step1|) have been
written in the form $(1-\alpha z)
(1-\beta z)$, instead of the more usual form $c(z-\rho_1)(z-\rho_2)$ where
$\rho_1$ and~$\rho_2$ are the roots. The reason is that
$(1-\alpha z)(1-\beta z)$ leads to nicer expansions in power series.

We can find $\alpha$ and~$\beta$ in several ways, one of which uses a slick
trick:
\g As usual, the authors can't resist a~trick.\g
Let us introduce a new variable $w$ and try to find the
factorization
\begindisplay
w^2-wz-z^2=(w-\alpha z)(w-\beta z)\,.
\enddisplay
Then we can simply set $w=1$ and we'll have the factors of $1-z-z^2$.
The roots of $w^2-wz-z^2=0$ can be found by the quadratic formula; they are
\begindisplay
{z\pm\sqrt{z^{2^{\mathstrut}}+4z^2}\over2}={1\pm\sqrt5\over2}\,z\,.
\enddisplay
Therefore
\begindisplay \advance\abovedisplayskip-6pt \advance\belowdisplayskip-3pt
w^2-wz-z^2=\biggl(w-{1+\sqrt5\over2}\,z\biggr)
\biggl(w-{1-\sqrt5\over2}\,z\biggr)
\enddisplay
and we have the constants $\alpha$ and $\beta$ we were looking for.

The number $(1+\sqrt5\mskip2mu)/2\approx1.61803$ is
\g The ratio of one's height to the height of one's navel is
"!Office of navel research"
approximately\/ $1.618$, according to extensive empirical
observations by European scholars~[|gardner-phi|].\g
important in many parts of
mathematics as well as in the art world, where it has been considered since
ancient times to be the most pleasing ratio for many kinds of design. 
Therefore it has a special name, the {\it "golden ratio"}.
We denote it by the Greek letter~$\phi$, in honor of
"Phidias" who is said to have used it consciously in his sculpture.
The other root $(1-\sqrt5\mskip2mu)/2=-1/\phi\approx-.61803$ shares
many properties of~$\phi$, so it has the special name $\phihat$, ``phi hat.\qback''
"!phi" "!$\phi$"
These numbers are roots of the equation $w^2-w-1=0$, so we have
\begindisplay
\phi^2=\phi+1\,;\qquad \phihat^2=\phihat+1\,.
\eqno\eqref|phi-phihat-rec|
\enddisplay
(More about $\phi$ and $\phihat$ later.)

We have found the constants $\alpha=\phi$ and $\beta=\phihat$ needed in
\eq(|fib-step1|); now we merely need to find $A$ and~$B$ in \eq(|fib-step2|).
Setting $z=0$ in that equation tells us that $B=-A$, so \eq(|fib-step2|)
boils down to
\begindisplay
-\phihat A+\phi A=1\,.
\enddisplay
The solution is $A=1/(\phi-\phihat)=1/\mskip-1mu\sqrt5$; the
partial fraction expansion of \eq(|fib-gf|) is therefore
\begindisplay
F(z)={1\over\sqrt5}\biggl({1\over1-\phi z}-{1\over1-\phihat z}\biggr)\,.
\eqno
\enddisplay
Good, we've got $F(z)$ right where we want it. Expanding the fractions
into power series as in \eq(|fib-step0|) gives a closed form for the
coefficient of $z^n$:
\begindisplay \postdisplaypenalty=100000
F_n={1\over\sqrt5}\bigl(\phi^n-\phihat^n\bigr)\,.
\eqno\eqref|fib-sol|
\enddisplay
{\widowpenalty=10000
(This formula was first published by Leonhard "Euler" [|euler-middle|] in 1765,
but people forgot about it until it was rediscovered by
 Jacques "Binet" [|binet|] in 1843.)\par}

Before we stop to marvel at our derivation, we should check its
accuracy. For $n=0$ the formula correctly gives $F_0=0@$; for
$n=1$, it gives $F_1=(\phi-\phihat)/\mskip-1mu\sqrt5$, which is indeed~$1$.
For higher powers, equations \eq(|phi-phihat-rec|) show that the
numbers defined by \eq(|fib-sol|) satisfy the Fibonacci recurrence,
so they must be the Fibonacci numbers by induction.
(We could also expand $\phi^n$ and~$\phihat^n$ by the
binomial theorem and chase down the various powers of $\sqrt5$;
but that gets pretty messy.
The point of a closed form is not necessarily to provide us with a
fast method of calculation, but rather to tell us how $F_n$ relates
to other quantities in mathematics.)

With a little clairvoyance we could simply have guessed formula \eq(|fib-sol|)
and proved it by induction. But the method of generating functions is
a powerful way to discover it; in Chapter~7 we'll see that the same method
leads us to the solution of recurrences that are considerably
more difficult. Incidentally, we never worried about whether
the infinite sums in our derivation of \eq(|fib-sol|)
were convergent; it turns out that
most operations on the coefficients of power series can be justified rigorously
whether or not the sums actually converge [|henrici|].
 Still, skeptical readers who
suspect fallacious reasoning with infinite sums can take
comfort in the fact that equation \eq(|fib-sol|), once found by using
infinite series, can be verified by a solid induction proof.

One of the interesting consequences of \eq(|fib-sol|) is that the integer
$F_n$ is extremely close to the irrational number
$\phi^n\!/\mskip-1mu\sqrt5$ when $n$ is large. (Since $\phihat$
is less than~$1$ in absolute value, $\phihat^n$ becomes exponentially
small and its effect is almost negligible.) For example,
$F_{10}=55$ and $F_{11}=89$ are very near
\begindisplay \openup4pt
{\phi^{10}\over\sqrt5}\approx55.00364\And
{\phi^{11}\over\sqrt5}\approx88.99775\,.\cr
\enddisplay
We can use this observation to derive another closed form,
\begindisplay
F_n=\biggl\lfloor{\phi^n\over\sqrt5}+\half\biggr\rfloor=
 {\phi^n\over\sqrt5}\quad \hbox{rounded to the nearest integer},
\eqno\eqref|fib-sol2|
\enddisplay
because $\bigl\vert\phihat^n\!/\mskip-1mu\sqrt5\,\bigr\vert<\half$ for all $n\ge0$.
When $n$ is even, $F_n$ is a little bit less than
$\phi^n\!/\mskip-1mu\sqrt5\,$; otherwise it is a little greater.

"Cassini's identity" \eq(|cassini-id|) can be rewritten
\begindisplay
{F_{n+1}\over F_n}\,-\,{F_n\over F_{n-1}}={(-1)^n\over F_{n-1\,}F_n}\,.
\enddisplay
When $n$ is large, $1/@F_{n-1}@F_n$ is very small, so
$F_{n+1}/F_n$ must be very nearly the same as $F_n/F_{n-1}$; and \eq(|fib-sol2|)
tells us that this ratio approaches $\phi$.
In fact, we have
\begindisplay
F_{n+1}=\phi F_n+\phihat^n\,.
\eqno\eqref|f+-vs-phif|
\enddisplay
(This identity is true by inspection when $n=0$ or $n=1$, and by induction when
$n>1$; we can also prove it directly by plugging in \eq(|fib-sol|).) The ratio
$F_{n+1}/F_n$ is very close to $\phi$, which it alternately overshoots
and undershoots.

By coincidence, $\phi$ is also very nearly the number of "kilometer"s in
a "mile". (The exact number is $1.609344$, since $1$~inch is exactly
$2.54$ centimeters.) This gives us a handy way to convert mentally between
kilometers and miles, because a distance of $F_{n+1}$ kilometers is
\g If the USA ever goes metric, our speed limit signs will go from\/
55 mi/hr to 89 km/hr. Or maybe the highway people will be generous and
let us go~90.\g
(very nearly) a distance of $F_n$ miles.

Suppose we want to convert a non-Fibonacci number from kilometers to miles;
what is $30$ km, American style? Easy: We just use the Fibonacci number system
and mentally convert $30$ to its Fibonacci representation $21+8+1$ by
the greedy approach explained earlier. Now we can shift each number down one
notch, getting $13+5+1$. (The former `$1$' was $F_2$, since $k_r\gg0$
in \eq(|zeck|); the new `$1$' is $F_1$.) Shifting down divides by~$\phi$,
more or less. Hence $19$~miles is our estimate. (That's pretty close;
the correct answer
is about $18.64$ miles.) Similarly, to go from miles to kilometers we
can shift up a notch; $30$~miles is approximately $34+13+2=49$ kilometers.
(That's not quite as close; the correct number is about $48.28$.)

It turns out that this shift-down
rule gives the correctly rounded number of miles
per $n$ kilometers for all $n\le100$, except in the cases $n=4$, $12$, $62$,
$75$, $91$, and $96$, when it is off by less than $2/3$~mile.
And the shift-up rule
gives either the correctly rounded number of kilometers
\g The ``shift down'' rule changes $n$ to $f(n/\phi)$ and the
``shift up'' rule changes $n$ to $f(n\phi)$, where $f(x)=
 \lfloor x+\phi^{-1}\rfloor$.\g
% proof: Consider $F_n\phi=F_{n+1}-\phihat^n$; $F_n/\phi=F_{n-1}-\phihat^n$.
for $n$ miles, or $1$~km too many, for all $n\le126$. 
(The only really embarrassing case is $n=4$, where the individual rounding
errors
for $n=3+1$ both go the same direction instead of cancelling each other out.)

\beginsection 6.7 Continuants

Fibonacci numbers have important connections to the Stern--Brocot tree
that we studied in Chapter~4, and they have important generalizations
to a sequence of polynomials that "Euler" studied extensively. These
polynomials are called {\it "continuants"}, because they are the key
to the study of "continued fractions" like
\begindisplay
a_0+{1^{\mathstrut}\over
\displaystyle a_1+{1^{\mathstrut}\over
\displaystyle a_2+{1^{\mathstrut}\over
\displaystyle a_3+{1^{\mathstrut}\over
\displaystyle a_4+{1^{\mathstrut}\over
\displaystyle a_5+{1^{\mathstrut}\over
\displaystyle a_6+{1^{\mathstrut}\over
a_7}}}}}}}\,.
\eqno
\enddisplay % I made this fraction 3 lines longer to get a good page break...

The continuant polynomial $K_n(x_1,x_2,\ldots,x_n)$ has $n$ parameters,
and it is defined by the following recurrence:
\tabref|nn:continuant|%
\begindisplay
K_0()&=1\,;\cr
K_1(x_1)&=x_1\,;\cr
K_n(x_1,\ldots,x_n)&=K_{n-1}(x_1,\ldots,x_{n-1})x_n+K_{n-2}(x_1,\ldots,x_{n-2})\,.
\eqno\eqref|cont-rec|
\enddisplay
For example, the next three cases after $K_1(x_1)$ are
\begindisplay
K_2(x_1,x_2)&=x_1x_2+1\,;\cr
K_3(x_1,x_2,x_3)&=x_1x_2x_3+x_1+x_3\,;\cr
K_4(x_1,x_2,x_3,x_4)&=x_1x_2x_3x_4+x_1x_2+x_1x_4+x_3x_4+1\,.\cr
\enddisplay
It's easy to see, inductively, that the number of terms is a "Fibonacci number":
\begindisplay
K_n(1,1,\ldots,1)=F_{n+1}\,.
\eqno\eqref|cont1|
\enddisplay

When the number of parameters is implied by the context, we can write
simply `$K$' instead of `$K_n$', just as we can omit the number of
parameters when we use the hypergeometric functions~$F$ of Chapter~5.
For example, $K(x_1,x_2)=K_2(x_1,x_2)=x_1x_2+1$.
The subscript $n$ is of course necessary in formulas like \thiseq.

\unitlength=3pt
\def\mdot{\beginpicture(1.5,2)(-.75,-1)\put(0,0){\disk{.7}}\endpicture}%
\def\mdash{\beginpicture(2.5,2)(-.5,-1)
 \put(0,0){\thicklines\line(1,0){1.5}}\endpicture}%
Euler observed that $K(x_1,x_2,\ldots,x_n)$ can be obtained by starting
with the product $x_1x_2\ldots x_n$ and then striking out adjacent pairs
$x_kx_{k+1}$ in all possible ways. We can represent Euler's rule graphically
by constructing all ``Morse code'' sequences of dots and dashes having
length~$n$, where each dot contributes $1$ to the length and each dash
contributes~$2$; here are the Morse code sequences of length~$4$:
\begindisplay
\mdot\mdot\mdot\mdot\qquad\mdot\mdot\mdash\qquad\mdot\mdash\mdot
\qquad\mdash\mdot\mdot\qquad\mdash\mdash
\enddisplay
These dot-dash patterns correspond
 to the terms of $K(x_1,x_2,x_3,x_4)$; a dot signifies a
variable that's included and a dash signifies a pair of variables that's
excluded. For example, $\mdot\mdash\mdot$ corresponds to $x_1x_4$.

A "Morse code" sequence of length $n$ that has $k$ dashes has $n-2k$~dots
and $n-k$ symbols altogether. These dots and dashes can be arranged
in $n-k\choose k$ ways;
therefore if we replace each dot by~$z$ and each dash by~$1$ we get
\begindisplay
K_n(z,z,\ldots,z)=\sum_{k=0}^n{n-k\choose k}z^{n-2k}\,.
\eqno\eqref|kzzz|
\enddisplay
We also know that the total number
of terms in a continuant is a Fibonacci number; hence we have the identity
\begindisplay
F_{n+1}=\sum_{k=0}^n{n-k\choose k}\,.
\eqno
\enddisplay
(A closed form for \eq(|kzzz|), generalizing the "Euler"--"Binet" formula
\eq(|fib-sol|) for Fibonacci numbers, appears in \equ(5.|bc-gen-fib|).)

The relation between continuant polynomials and Morse code sequences shows
that continuants have a mirror symmetry:
\begindisplay
K(x_n,\ldots,x_2,x_1)=K(x_1,x_2,\ldots,x_n)\,.
\eqno
\enddisplay
Therefore they obey a recurrence that adjusts parameters at the left,
in addition to the right-adjusting
recurrence in definition~\eq(|cont-rec|):
\begindisplay
K_n(x_1,\ldots,x_n)&=x_1K_{n-1}(x_2,\ldots,x_n)+K_{n-2}(x_3,\ldots,x_n)\,.
\eqno\eqref|cont-rec'|
\enddisplay
Both of these recurrences are special cases of a more general law:
\begindisplay
&K_{m+n}(x_1,\ldots,x_m,x_{m+1},\ldots,x_{m+n})\cr
&\qquad= K_m(x_1,\ldots,x_m)\,K_n(x_{m+1},\ldots,x_{m+n})\cr
&\qquad\quad+K_{m-1}(x_1,\ldots,x_{m-1})\,K_{n-1}(x_{m+2},\ldots,x_{m+n})\,.
%\qquad\hbox{if $m,n>0$}.
\eqno\eqref|cm+n|
\enddisplay
This law is easily understood from the Morse code analogy: The first product
$K_mK_n$ yields the terms of $K_{m+n}$
in which there is no dash in the $[m,m+1]$ position, while
the second product yields the terms in which there is a dash there.
If we set all the $x$'s equal to~$1$, this identity tells us that
$F_{m+n+1}=F_{m+1}F_{n+1}+F_mF_n$; thus, \eq(|fn+k|) is a special case of~\thiseq.

Euler [|euler-continuants|]
discovered that continuants obey an even more remarkable law, which
generalizes "Cassini's identity":
\begindisplay
&K_{m+n}(x_1,\ldots,x_{m+n})\,K_k(x_{m+1},\ldots,x_{m+k})\cr
&\ =K_{m+k}(x_1,\ldots,x_{m+k})\,K_n(x_{m+1},\ldots,x_{m+n})\cr
&\ \qquad+(-1)^kK_{m-1}(x_1,\ldots,x_{m-1})
 \,K_{n-k-1}(x_{m+k+2},\ldots,x_{m+n})\,.
\eqno\eqref|euler-cont|
\enddisplay
This law (proved in exercise~|prove-euler-cont|)
 holds whenever the subscripts on the $K$'s are all nonnegative.
For example, when $k=2$, $m=1$, and $n=3$, we have
\begindisplay
K(x_1,x_2,x_3,x_4)\,K(x_2,x_3)=K(x_1,x_2,x_3)\,K(x_2,x_3,x_4)+1\,.
\enddisplay

Continuant polynomials are intimately connected with "Euclid's algorithm".
Suppose, for example, that the computation of $\gcd(m,n)$ finishes in
four steps: % ALSO OK WOULD BE three steps:
\begindisplay
\gcd(m,n)&=\gcd(n_0,n_1)\qquad}\hfill{n_0&=m\,,\qquad n_1=n\,;\cr
&=\gcd(n_1,n_2)\qquad}\hfill{n_2&=n_0\bmod n_1=n_0-q_1n_1\,;\cr
&=\gcd(n_2,n_3)\qquad}\hfill{n_3&=n_1\bmod n_2=n_1-q_2n_2\,;\cr
&=\gcd(n_3,n_4)\qquad}\hfill{n_4&=n_2\bmod n_3=n_2-q_3n_3\,;\cr
&=\gcd(n_4,0)=n_4\,\rlap.\qquad\qquad}\hfill{0&=n_3\bmod n_4=n_3-q_4n_4\,.\cr
\enddisplay
Then we have
\begindisplay
n_4&=n_4&=K()n_4\,;\cr
n_3&=q_4n_4&=K(q_4)n_4\,;\cr
n_2&=q_3n_3+n_4&=K(q_3,q_4)n_4\,;\cr
n_1&=q_2n_2+n_3&=K(q_2,q_3,q_4)n_4\,;\cr
n_0&=q_1n_1+n_2&=K(q_1,q_2,q_3,q_4)n_4\,.\cr
\enddisplay
In general, if Euclid's algorithm finds the greatest common divisor~$d$
in $k$~steps, after computing the sequence of quotients $q_1$, \dots,~$q_k$,
then the starting numbers were $K(q_1,q_2,\ldots,q_k)d$ and $K(q_2,\ldots,q_k)d$.
(This fact was noticed early in the eighteenth century by
 Thomas Fantet de "Lagny"~[|lagny|],
who seems to have been the first person to consider continuants explicitly.
Lagny pointed out that consecutive Fibonacci numbers,
which occur as continuants when the $q$'s take their minimum values,
are therefore the smallest inputs that cause Euclid's algorithm to
take a given number of steps.)

Continuants are also intimately connected with "continued fractions",
from which they get their name. We have, for example,
\begindisplay
a_0+{1^{\mathstrut}\over
\displaystyle a_1+{1^{\mathstrut}\over
\displaystyle a_2+{1^{\mathstrut}\over
a_3}}}=
{K(a_0,a_1,a_2,a_3)\over K(a_1,a_2,a_3)}\,.
\eqno\eqref|cf-example|
\enddisplay
The same pattern holds for continued fractions of any depth. It is easily
proved by induction; we have, for example,
\begindisplay
{K(a_0,a_1,a_2,a_3+1/a_4)\over K(a_1,a_2,a_3+1/a_4)}=
{K(a_0,a_1,a_2,a_3,a_4)\over K(a_1,a_2,a_3,a_4)}\,,
\enddisplay
because of the identity
\begindisplay
&K_n(x_1,\ldots,x_{n-1},x_n+y)\cr
&\qquad=K_n(x_1,\ldots,x_{n-1},x_n)+
 K_{n-1}(x_1,\ldots,x_{n-1})\mskip2mu y\,.
\eqno\eqref|cont-deriv|
\enddisplay
(This identity is proved and generalized in exercise |gen-cont-deriv|.)

Moreover, continuants are closely connected with the "Stern--Brocot
tree" discussed in Chapter~4. Each node in that tree can be represented
as a sequence of $L$'s and $R$'s, say
\begindisplay
R^{a_0}L^{a_1}R^{a_2}L^{a_3}\ldots R^{a_{n-2}}L^{a_{n-1}}\,,\eqno\eqref|LR-gen|
\enddisplay
where $a_0\ge0$, $a_1\ge1$, $a_2\ge1$, $a_3\ge1$, \dots,
$a_{n-2}\ge1$, $a_{n-1}\ge0$, and $n$ is even.
Using the $2\times2$ matrices $L$ and $R$ of \equ(4.|LR|),
it is not hard to prove by induction that the matrix equivalent of
\thiseq\ is
\begindisplay
\pmatrix{K_{n-2}(a_1,\ldots,a_{n-2})&
 K_{n-1}(a_1,\ldots,a_{n-2},a_{n-1})\cr
\noalign{\smallskip}
K_{n-1}(a_0,a_1,\ldots,a_{n-2})&
 K_n(a_0,a_1,\ldots,a_{n-2},a_{n-1})\cr}\,.
\eqno\eqref|LR-closed|
\enddisplay
(The proof is part of exercise |matrices-and-continuants|.) For example,
\begindisplay
R^aL^{\!b}R^cL^{\!d}=
\pmatrix{bc+1&&bcd+b+d\cr abc+a+c&& abcd+ab+ad+cd+1\cr}\,.
\enddisplay
Finally, therefore, we can use \equ(4.|f-of-S|) to
write a closed form for the fraction in the
Stern--Brocot tree whose $L$-and-$R$ representation is \eq(|LR-gen|):
\begindisplay
f(R^{a_0}\ldots L^{a_{n-1}})={K_{n+1}(a_0,a_1,\ldots,a_{n-1},1)\over
 K_n(a_1,\ldots,a_{n-1},1)}\,.
\eqno\eqref|f-closed|
\enddisplay
(This is ``"Halphen"'s theorem'' [|halphen|].)
For example, to find the fraction for $LRRL$ we have $a_0=0$, $a_1=1$,
$a_2=2$, $a_3=1$, and $n=4$; equation \thiseq\ gives
\begindisplay
{K(0,1,2,1,1)\over K(1,2,1,1)}=
{K(2,1,1)\over K(1,2,1,1)}=
{K(2,2)\over K(3,2)}={5\over7}\,.
\enddisplay
(We have used the rule $K_n(x_1,\ldots,x_{n-1},x_n+1)=K_{n+1}(x_1,\ldots,
x_{n-1},x_n,1)$ to absorb leading
and trailing $1$'s in the parameter lists; this rule is obtained
by setting $y=1$ in \eq(|cont-deriv|).)

A comparison of \eq(|cf-example|) and \thiseq\ shows that the
fraction corresponding to a general node \eq(|LR-gen|) in the Stern--Brocot
tree has the continued fraction representation
\begindisplay
f(R^{a_0}\ldots L^{a_{n-1}})=
a_0+{1\over
\displaystyle a_1+{1^{\mathstrut}\over
\displaystyle a_2+{1^{\mathstrut}\over
\displaystyle \ldots\,+{1^{\mathstrut}\over
\displaystyle a_{n-1}+{1^{\mathstrut}\over1^{\mathstrut}}}}}}\,.
\eqno
\enddisplay
Thus we can convert at sight between continued fractions and the corresponding
nodes in the Stern--Brocot tree. For example,
\begindisplay
f(LRRL)=
0+{1\over
\displaystyle 1+{1^{\mathstrut}\over
\displaystyle 2+{1^{\mathstrut}\over
\displaystyle 1+{1^{\mathstrut}\over1^{\mathstrut}}}}}\,.
\enddisplay

We observed in Chapter 4 that
irrational numbers define infinite paths in the Stern--Brocot tree, and
"!Stern--Brocot representation"
that they can be represented as an infinite string of $L$'s and $R$'s.
If the infinite string for $\alpha$ is $R^{a_0}L^{a_1}R^{a_2}L^{a_3}\ldots\,$,
there is a corresponding infinite continued fraction
\begindisplay
\alpha=
a_0+{1^{\mathstrut}\over
\displaystyle a_1+{1^{\mathstrut}\over
\displaystyle a_2+{1^{\mathstrut}\over
\displaystyle a_3+{1^{\mathstrut}\over
\displaystyle a_4+{1^{\mathstrut}\over
\displaystyle a_5+{1^{\mathstrut}\over\ddots}}}}}}\,.
\eqno\eqref|inf-cf|
\enddisplay % again I extended it...
This infinite continued fraction can also be obtained directly: Let
$\alpha_0=\alpha$ and for $k\ge0$ let
\begindisplay
a_k=\lfloor\alpha_k\rfloor\,;\qquad \alpha_k=a_k+{1\over\alpha_{k+1}}\,.
\eqno
\enddisplay
The $a$'s are called the ``partial quotients'' of $\alpha$.
If $\alpha$ is rational, say $m/n$, this process runs through the
quotients found by Euclid's algorithm and then stops (with $\alpha_{k+1}=\infty$).

Is "Euler's constant" $\gamma$ rational or irrational? Nobody knows.
\g Or if they do, they're not talking.\g
We can get partial information about this
famous unsolved problem by looking for $\gamma$
in the Stern--Brocot tree; if it's rational we will find it, and
if it's irrational we will find all the closest rational approximations
to it.
The continued fraction for $\gamma$ begins with the following
partial quotients:
\begindisplay \let\preamble=\tablepreamble
k&&0&1&2&3&4&5&6&7&8\cr
\noalign{\hrule}
a_k&&0&1&1&2&1&2&1&4&3\cr
\enddisplay
Therefore its Stern--Brocot representation begins 
$LRLLRLLRLLLLRRRL\ldots\,$; no pattern is evident. Calculations by
Richard "Brent" [|brent-gamma|]
have shown that, if $\gamma$ is rational, its denominator must
be more than 10,000 decimal digits long. Therefore nobody believes
\g Well, $\gamma$ must be irrational, because of
a little-known "Einstein"ian assertion:
``"God" does not throw huge denominators at the universe.''\g
that $\gamma$ is rational; but nobody so far has been able to prove
that it isn't.

\smallbreak
Let's conclude this chapter by proving a remarkable identity that
ties a lot of these ideas together. We introduced the notion of
"spectrum" in Chapter~3; the spectrum of~$\alpha$
 is the multiset of numbers
$\lfloor n\alpha\rfloor$, where $\alpha$ is a given constant.
The infinite series
\begindisplay
\sum_{n\ge1}z^{\lfloor n\phi\rfloor}=z+z^3+z^4+z^6+z^8+z^9+\cdots
\enddisplay
can therefore be said to be the generating function for the spectrum of
$\phi$, where $\phi=(1+\sqrt5\mskip2mu)/2$ is the golden ratio.
The identity we will prove, discovered in 1976 by J.\thinspace L.
 "Davison"~[|davison|], is an infinite
continued fraction that relates this generating function to the Fibonacci
sequence:
\begindisplay
{z^{F_1}\over1+\displaystyle{z^{F^{\mathstrut}_2}\over
1+\displaystyle{z^{F^{\mathstrut}_3}\over
1+\displaystyle{z^{F^{\mathstrut}_4}\over\ddots}}}}=
(1-z)\sum_{n\ge1}z^{\lfloor n\phi\rfloor}\,.
\eqno\eqref|davison-id|
\enddisplay

Both sides of \thiseq\ are interesting;
let's look first at the numbers $\lfloor n\phi\rfloor$. If the Fibonacci
representation \eq(|zeck|)
 of~$n$ is $F_{k_1}+\cdots+F_{k_r}$, we expect $n\phi$ to
be approximately $F_{k_1+1}+\cdots+F_{k_r+1}$, the number we get from
shifting the Fibonacci representation left (as when converting from
miles to kilometers). In fact, we know from \eq(|f+-vs-phif|) that
\begindisplay
n\phi=F_{k_1+1}+\cdots+F_{k_r+1}-
\bigl(\phihat^{k_1}+\cdots+\phihat^{k_r}\bigr)\,.
\enddisplay
Now $\phihat=-1/\phi$ and $k_1\gg\cdots\gg k_r\gg0$, so we have
\begindisplay
\bigl\vert\phihat^{k_1}+\cdots+\phihat^{k_r}\bigr\vert
&<\phi^{-k_r}+\phi^{-k_r-2}+\phi^{-k_r-4}+\cdots\cr
&={\phi^{-k_r}\over
 1-\phi^{-2}}=\phi^{1-k_r}\le\phi^{-1}<1\,;
\enddisplay
and $\phihat^{k_1}+\cdots+\phihat^{k_r}$ has the same sign as
$(-1)^{k_r}$, by a similar argument. Hence
\begindisplay
\lfloor n\phi\rfloor=F_{k_1+1}+\cdots+F_{k_r+1}\;-\;\bigi[
\hbox{$k_r(n)$ is even}\bigr]\,.
\eqno\eqref|floor-nphi|
\enddisplay
Let us say that a number $n$ is {\it Fibonacci odd\/} (or $F@$-odd for short)
"!Fibonacci odd and even"
if its least significant Fibonacci bit is~$1$; this is the same as saying
that $k_r(n)=2$. Otherwise $n$ is {\it Fibonacci even\/} ($F@$-even).
For example, the smallest $F@$-odd numbers are $1$,~$4$, $6$, $9$,
$12$, $14$, $17$, and~$19$.
If $k_r(n)$ is even, then $n-1$ is $F@$-even, by \eq(|10101-sum|);
similarly, if $k_r(n)$ is odd, then $n-1$ is $F@$-odd.
Therefore
\begindisplay
\hbox{$k_r(n)\,$ is even}\;\iff\;\hbox{$n-1\,$ is $F@$-even}.
\enddisplay
Furthermore, if $k_r(n)$ is even, \eq(|floor-nphi|) implies that
$k_r\bigl(\lfloor n\phi\rfloor\bigr)=2$; if $k_r(n)$ is odd,
\eq(|floor-nphi|) says that $k_r\bigl(\lfloor n\phi\rfloor\bigr)=
k_r(n)+1$. Therefore $k_r\bigl(\lfloor n\phi\rfloor\bigr)$ is always even,
and we have proved that
\begindisplay
\hbox{$\lfloor n\phi\rfloor-1\,$ is always $F@$-even.}
\enddisplay
Conversely, if $m$ is any $F@$-even number, we can reverse this computation
and find an~$n$ such that $m+1=\lfloor n\phi\rfloor$. (First add~$1$ in
$F@$-notation as explained earlier. If no carries occur, $n$ is $(m+2)$
shifted right; otherwise $n$ is $(m+1)$ shifted right.) The right-hand sum of
\eq(|davison-id|) can therefore be written
\begindisplay
\sum_{n\ge1}z^{\lfloor n\phi\rfloor}=z\sum_{m\ge0}z^m@\[\hbox{$m$ is $F@$-even}]\,.
\eqno\eqref|even-spec|
\enddisplay

How about the fraction on the left? Let's rewrite
\eq(|davison-id|)
so that the continued fraction looks like \eq(|inf-cf|),
with all numerators~$1$:
\begindisplay
{1\over
\displaystyle z^{-F_0}+{1^{\mathstrut}\over
\displaystyle z^{-F_1}+{1^{\mathstrut}\over
\displaystyle z^{-F_2}+{1^{\mathstrut}\over\ddots}}}}
={1-z\over z}\,\sum_{n\ge1}z^{\lfloor n\phi\rfloor}\,.
\eqno\eqref|davison-id'|
\enddisplay
(This transformation is a bit tricky! The numerator and denominator of the original
 fraction having $z^{F_n}$ as numerator should be divided by $z^{F_{n-1}}$.)
If we stop this new continued fraction at $1/z^{-F_n}$, its value
will be a ratio of continuants,
\begindisplay
{K_{n+2}(0,z^{-F_0},z^{-F_1},\ldots,z^{-F_n})\over
K_{n+1}(z^{-F_0},z^{-F_1},\ldots,z^{-F_n})}
={K_{n}(z^{-F_1},\ldots,z^{-F_n})\over
K_{n+1}(z^{-F_0},z^{-F_1},\ldots,z^{-F_n})}\,,
\enddisplay
as in \eq(|cf-example|).
Let's look at the denominator first, in hopes that it will be tractable.
Setting $Q_n=
K_{n+1}(z^{-F_0},\ldots,z^{-F_n})$, we find $Q_0=1$, $Q_1=1+z^{-1}$,
$Q_2=1+z^{-1}+z^{-2}$, $Q_3=1+z^{-1}+z^{-2}+z^{-3}+z^{-4}$, and in general
everything fits beautifully and gives a geometric series
\begindisplay
Q_n=1+z^{-1}+z^{-2}+\cdots+z^{-(F_{n+2}-1)}\,.
\enddisplay
The corresponding numerator is $P_n=K_n(z^{-F_1},\ldots,z^{-F_n})$;
this turns out to be like $Q_n$ but with fewer terms. For example, we have
\begindisplay
P_5=z^{-1}+z^{-2}+z^{-4}+z^{-5}+z^{-7}+z^{-9}+z^{-10}+z^{-12}\,,
\enddisplay
compared with $Q_5=1+z^{-1}+\cdots+z^{-12}$. A closer look reveals the pattern
governing which terms are present: We have
\begindisplay
P_5={1{+}z^2{+}z^3{+}z^5{+}z^7{+}z^8{+}z^{10}{+}z^{11}\over z^{12^{\mathstrut}}}
=z^{-12}\sum_{m=0}^{12}z^m\,\[\hbox{$m$ is $F@$-even}]\,;
\enddisplay
and in general we can prove by induction that
\begindisplay
P_n=z^{1-F_{n+2}}\sum_{m=0}^{F_{n+2}-1}z^m\,\[\hbox{$m$ is $F@$-even}]\,.
\enddisplay
Therefore
\begindisplay
{P_n\over Q_n}=
{\sum_{m=0_{\mathstrut}}^{F_{n+2}-1}z^m\,\[\hbox{$m$ is $F@$-even}]\over
 \sum_{m=0}^{F_{n+2}^{\mathstrut}-1}z^m}\,.
\enddisplay
Taking the limit as $n\to\infty$ now gives \eq(|davison-id'|),
because of \eq(|even-spec|).

\beginexercises

\subhead \kern-.05em Warmups

\ex:
What are the ${4\brack\mskip2mu 2\mskip2mu}=11$
 permutations of $\{1,2,3,4\}$ that have
exactly two cycles? (The cyclic forms appear in \eq(|stirl1-42|);
non-cyclic forms like $2314$ are desired instead.)
\answer $2314$, $2431$, $3241$, $1342$, $3124$, $4132$, $4213$, $1423$,
$2143$, $3412$, $4321$.

\ex:
There are $m^n$ functions from a set of $n$ elements into a set of $m$~elements.
How many of them range over exactly $k$ different function values?
\answer ${n\brace k}m\_k$, because every such function partitions its
domain into $k$ non\-empty subsets, and there are $m\_k$ ways to assign
function values for each partition. (Summing over~$k$ gives a combinatorial
proof of \equ(6.|expand-ord-to-falling|).)

\ex:
"Card stack"ers in the real world know that it's wise to allow a bit of
slack so that the cards will not topple over when a breath of wind comes
along.  Suppose the center of gravity of the top~$k$ cards is required to
be at least $\epsilon$~units from the edge of the $k+1$st card. (Thus, for
example, the first card can overhang the second by at most $1-\epsilon$
units.) Can we still achieve arbitrarily large overhang, if we have enough
cards?
\answer Now $d_{k+1}\le($center of gravity$)-\epsilon=
1-\epsilon+(d_1+\cdots+d_k)/k$. This recurrence is like \equ(6.|card-stack-rec|)
but with $1-\epsilon$ in place of~$1$; hence the optimum solution is
$d_{k+1}=(1-\epsilon)H_k$. This is unbounded as long as $\epsilon<1$.

\ex:\exref|subtract-out-even|%
Express $1/1+1/3+\cdots+1/(2n{+}1)$ in terms of harmonic numbers.
\answer $H_{2n+1}-\half H_n$. (Similarly $\sum_{k=1}^{2n}(-1)^{k-1}\!/k
=H_{2n}-H_n$.)

\ex:\exref|alt-un|%
Explain how to get
the recurrence \eq(|unxy-rec|) from the definition of\/ $U_n(x,y)$ in
\eq(|unxy-def|), and solve the recurrence.
\answer $U_n(x,y)$ is equal to
\begindisplay\tightplus\let\displaystyle=\textstyle
x\sum_{k\ge1}{n\choose k}(-1)^{k-1}k^{-1}(x+ky)^{n-1}
+y\sum_{k\ge1}{n\choose k}(-1)^{k-1}(x+ky)^{n-1}\,,
\enddisplay
and the first sum is
\begindisplay
U_{n-1}(x,y)+\sum_{k\ge1}{n-1\choose k-1}(-1)^{k-1}k^{-1}(x+ky)^{n-1}\,.
\enddisplay
The remaining $k^{-1}$ can be absorbed, and we have
\begindisplay\tightplus
\sum_{k\ge1}{n\choose k}(-1)^{k-1}(x+ky)^{n-1}
&=x^{n-1}+\sum_{k\ge0}{n\choose k}(-1)^{k-1}(x+ky)^{n-1}\cr
&=x^{n-1}\,.\cr
\enddisplay
This proves \equ(6.|unxy-rec|).
 Let $R_n(x,y)=x^{-n}U_n(x,y)$; then $R_0(x,y)=0$ and $R_n(x,y)=
R_{n-1}(x,y)+1/n+y/x$, hence $R_n(x,y)=H_n+ny/x$. (Incidentally, the
original sum $U_n=U_n(n,-1)$ doesn't lead to a recurrence such as
this; therefore the more general sum, which detaches $x$ from its dependence
on~$n$, is easier to solve inductively than its special case. This is
another instructive example where a strong "induction" hypothesis makes the
difference between success and failure.)

\ex: An explorer has left a pair of baby "rabbits" on an island. If baby rabbits
\g If the harmonic numbers are worm numbers, the Fibonacci numbers
are~rabbit~numbers.\g
become adults after one month, and if each pair of adult rabbits produces
one pair of baby rabbits every month, how many pairs of rabbits are
present after $n$ months? (After two months there are two pairs, one
of which is newborn.) Find a connection between this problem
and the ``bee tree'' in the text.
\answer Each pair of babies {\sevenbf bb} present at the end of a month
\g The Fibonacci recurrence is additive, but the
rabbits are multiplying.\g
becomes a pair of adults {\subtitle aa} at the end of the next month;
and each pair {\subtitle aa} becomes an {\subtitle aa} and a {\sevenbf bb}.
Thus each {\sevenbf bb} behaves like a drone in the bee tree and each
{\subtitle aa} behaves like a queen, except that the bee tree goes backward
in time while the rabbits are going forward. There are $F_{n+1}$ pairs
of rabbits after $n$ months; $F_n$ of them are adults and $F_{n-1}$ are babies.
(This is the context in which "Fibonacci" originally introduced
his numbers.)
\source{"Fibonacci" [|fibonacci|, p.~283].}

\ex:
Show that "Cassini's identity" \eq(|cassini-id|) is a special case of
\eq(|fn+k|), and a special case of \eq(|euler-cont|).
\answer (a) Set $k=1-n$ and apply \equ(6.|f-n|). (b)~Set $m=1$ and $k=n-1$
and apply \equ(6.|cont1|).

\ex:
Use the "Fibonacci number system" to convert 65 mi/hr into an approximate number
"!kilometers" "!miles"
of km/hr.
\answer $55+8+2$ becomes $89+13+3=105$; the true value is $104.607361$.
\g That ``true value'' is the length of 65 international miles,
but the international mile is actually only .999998 as big as
a U.\thinspace S. statute mile.\par
There are exactly 6336 kilometers in 3937 U.\thinspace S. statute
"miles"; the Fibonacci method converts 3937 to 6370.\g

\ex: About how many square kilometers are in $8$ square miles?
\answer $21$. (We go from $F_n$ to $F_{n+2}$ when the units are squared. The
true answer is about $20.72$.)

\ex:
What is the continued fraction representation of $\phi$?
\answer The partial quotients $a_0$, $a_1$, $a_2$, \dots\ are all equal
to~$1$, because $\phi=1+1/\phi$. (The "Stern--Brocot representation"
is therefore $RLRLRLRLRL\ldots\,\,$.)

\subhead Basics

\ex:
What is $\sum_k(-1)^k{n\brack k}$, the row sum of Stirling's cycle-number triangle
with alternating signs, when $n$ is a nonnegative integer?
\answer $(-1)\_^n=\[n=0]-\[n=1]$; see \equ(6.|expand-rising-to-ord|).

\ex:\exref|stirling-inversion|%
Prove that Stirling numbers have an inversion law analogous to
\equ(5.|binomial-inversion|):
\begindisplay
g(n)=\sum_k{n\brace k}(-1)^k f(k)\!\iff\!
f(n)=\sum_k{n\brack k}(-1)^k g(k)\,.
\enddisplay
\answer This is a consequence of \equ(6.|stirl-inv|) and its
dual in Table |stirling-id1|.

\ex:\exref|small-vartheta|%
The "differential operators" $D={d\over dz}$ and $\vartheta=zD$ are
mentioned in Chapters 2 and~5. We have
\begindisplay
\vartheta^2=z^2D^2+zD\,,
\enddisplay
because $\vartheta^2f(z)=\vartheta zf'(z)=z{d\over dz}zf'(z)=
z^2f''(z)+zf'(z)$, which is\break
$(z^2D^2+zD)f(z)$. Similarly it can be shown that
$\vartheta^3=z^3D^3+3z^2D^2+zD$. Prove the general formulas
\begindisplay \openup4pt
\vartheta^n&=\sum_k{n\brace k}z^kD^k\,,\cr
z^nD^n&=\sum_k{n\brack k}(-1)^{n-k}\vartheta^k\,,\cr
\enddisplay
for all $n\ge0$. (These can be used to convert between differential expressions
of the forms $\sum_k\alpha_kz^kf^{(k)}(z)$ and $\sum_k\beta_k\vartheta^kf(z)$,
as in \equ(5.|hyp-alt-diff-eq|).)
\answer The two formulas are equivalent, by exercise |stirling-inversion|.
We can use induction. Or we can observe that $z^nD^n$ applied to $f(z)=z^x$
gives $x\_n@z^x$ while $\vartheta^n$ applied to the same function gives
$x^nz^x$; therefore the sequence $\<\vartheta^0,\vartheta^1,\vartheta^2,
\ldots\,\>$ must relate to $\<z^0D^0,z^1D^1,z^2D^2,\ldots\,\>$ as
$\<x^0,x^1,x^2,\ldots\,\>$ relates to $\<x\_0,x\_1,x\_2,\ldots\,\>$.

\ex:\exref|prove-eulerian-expansion|%
Prove the power identity \eq(|expand-ord-to-consec|)
for Eulerian numbers.
\answer We have
\begindisplay
x{x+k\choose n}=(k+1){x+k\choose n+1}+(n-k){x+k+1\choose n+1}\,,
\enddisplay
because $(n+1)x=(k+1)(x+k-n)+(n-k)(x+k+1)$. (It suffices to verify
the latter identity when $k=0$, $k=-1$, and~$k=n$.)

\ex:
Prove the Eulerian identity \eq(|expand-stirling-to-eulerian|) by taking
the $m$th difference of \eq(|expand-ord-to-consec|).
\answer Since $\Delta\bigl({x+k\choose n}\bigr)={x+k\choose n-1}$, we have
the general formula
\begindisplay
\sum_k{n\euler k}{x+k\choose n-m}=\Delta^m(x^n)=\sum_j{m\choose j}(-1)^{m-j}
(x+j)^n\,.
\enddisplay
Set $x=0$ and appeal to \equ(6.|sid-5|).
\source{[|knuth3|, exercise 5.1.3--2].}

\ex:
What is the general solution of the double recurrence
\begindisplay
A_{n,0}&=a_n\,\[n\ge0]\,;\qquad A_{0,k}=0\,,\qquad\hbox{if $k>0@$};\cr
A_{n,k}&=kA_{n-1,k}+A_{n-1,k-1}\,,\qquad\hbox{integers $k,n$},\cr
\enddisplay
when $k$ and $n$ range over the set of {\it all\/} integers?
\answer $A_{n,k}=\sum_{j\ge0}a_j{n-j\brace k}$; this sum is always finite.

\ex:\exref|nk-examples|%
Solve the following recurrences, assuming that
\def\\{\atopwithdelims\vert\vert}
$n\\k$ is zero when $n<0$ or $k<0$:
\par\nobreak\medskip
\setmathsize{{n\\k}=(n-k){n-1\\k}+{n-1\\k-1}+\[n=k=0]\,,}
\itemitem{a}\mathsize{{n\\k}={n-1\\k}+n\,{n-1\\k-1}+\[n=k=0]\,,}
\qquad\hbox{for $n,k\ge0$}.
\par\nobreak\medskip
\itemitem{b}\mathsize{{n\\k}=(n-k){n-1\\k}+{n-1\\k-1}+\[n=k=0]\,,}
\qquad\hbox{for $n,k\ge0$}.
\par\nobreak\medskip
\itemitem{c}\mathsize{{n\\k}=k\,{n-1\\k}+k\,{n-1\\k-1}+\[n=k=0]\,,}
\qquad\hbox{for $n,k\ge0$}.
\answer\def\\{\atopwithdelims\vert\vert}%
(a) ${n\\k}={n+1\brack n+1-k}$. (b) ${n\\k}=n\_{n-k}=n!\,\[n\ge k]/k!$.
(c)~${n\\k}=k!{n\brace k}$.

\ex:\exref|sigma-rec|%
Prove that the "Stirling polynomials" satisfy
\begindisplay
(x+1)\,\sigma_n(x+1)=(x-n)\,\sigma_n(x)+x@\sigma_{n-1}(x)\,.
\enddisplay
\answer This is equivalent to \equ(6.|stirl2-rec|) or \equ(6.|stirl1-rec|).

\ex:
Prove that the generalized Stirling numbers satisfy
\begindisplay\openup4pt
&\sum_{k=0}^n{x+k\brace x}{x\brack x-n+k}(-1)^k\bigg/{x+k\choose n+1}
=0\,,\quad\hbox{integer $n>0$}.\cr
&\sum_{k=0}^n{x+k\brack x}{x\brace x-n+k}(-1)^k\bigg/{x+k\choose n+1}
=0\,,\quad\hbox{integer $n>0$}.\cr
\enddisplay
\answer Use Table |stirling-convolutions|.

\ex:
Find a closed form for $\sum_{k=1}^nH_k^{(2)}$.
\answer $\sum_{1\le j\le k\le n}1/j^2=\sum_{1\le j\le n}(n+1-j)/j^2=
(n+1)H_n^{(2)}-H_n$.

\ex:\exref|prove-h-not-int|%
Show that if $H_n=a_n/b_n$, where $a_n$ and $b_n$ are integers, the
denominator $b_n$ is a multiple of $2^{\lfloor\lg n\rfloor}$.
\Hint: Consider the number $2^{\lfloor\lg n\rfloor-1}H_n-\half$.
\answer The hinted number is a sum of fractions with odd denominators,
so it has the form $a/b$ with $a$ and~$b$ odd.
(Incidentally, "Bertrand's postulate" implies that $b_n$ is also divisible
by at least one odd prime, whenever $n>2$.)
\source{"Theisinger" [|theisinger|].}

\ex:\exref|z-harm-def|%
Prove that the infinite sum
\begindisplay
\sum_{k\ge1}\biggl({1\over k}-{1\over k+z}\biggr)
\enddisplay
converges for all complex numbers $z$, except when $z$ is a negative integer;
"!harmonic numbers, complex"
and show that it equals $H_z$ when $z$ is a nonnegative integer. (Therefore
we can use this formula to define harmonic numbers~$H_z$ when $z$ is complex.)
\answer $\bigl\vert z/k(k+z)\bigr\vert\le2\vert z\vert/k^2$ when
$k>2\vert z\vert$, so the sum is well defined when the denominators are
not zero. If $z=n$ we have $\sum_{k=1}^m\bigl(1/k-1/(k+n)\bigr)=
H_m-H_{m+n}+H_n$, which approaches $H_n$ as $m\to\infty$.
(The quantity $H_{z-1}-\gamma$ is often called the psi function $\psi(z)$.)

\ex:\exref|z/ez+1|%
Equation \eq(|bern-gf|) gives the coefficients of $z/(e^z-1)$, when
expanded in powers of~$z$. What are the coefficients of $z/(e^z+1)$?
\Hint: Consider the identity $(e^z+1)(e^z-1)=e^{2z}-1$.
\answer $z/(e^z+1)=z/(e^z-1)-2z/(e^{2z}-1)=\sum_{n\ge0}(1-2^n)B_nz^n\!/n!$.

\ex:\exref|genocchi-answer|%
Prove that the tangent number $T_{2n+1}$ is a multiple of $2^n$. \Hint:
Prove that all coefficients of\/ $T_{2n}(x)$ and $T_{2n+1}(x)$ are
multiples of $2^n$.
\answer When $n$ is odd, $T_n(x)$ is a polynomial in $x^2$, hence its
coefficients are multiplied by even numbers when we form the derivative
and compute $T_{n+1}(x)$ by \equ(6.|tan-rec|). (In fact we can prove more:
The Bernoulli number $B_{2n}$ always has $2$ to the first power in its
denominator, by exercise~|prove-staudt-clausen|;
hence $2^{2n-k\,}\edivides T_{2n+1} \iff 2^k\edivides(n+1)$.
The odd positive integers $(n+1)T_{2n+1}/2^{2n}$ are called
"Genocchi numbers"~$\<1,1,3,17,155,2073,\ldots\,\>$,
after Genocchi~[|genocchi|].)

\ex:
Equation \eq(|worm-ratio|) proves that the worm will
eventually reach the end of the rubber band at some time~$N$.
Therefore there must come a first time~$n$ when he's
closer to the end after $n$~minutes than he was after $n-1$ minutes.
Show that $n<\half N$.
\answer $100n-nH_n<100(n-1)-(n-1)H_{n-1}\iff H_{n-1}>99$.
(The least such $n$ is approximately $e^{99-\gamma}$, while he finishes at
$N\approx e^{100-\gamma}$, about $e$~times as long. So he is getting
closer during the final 63\% of his journey.)
\source{"Gardner" [|gardner-rope|] credits Denys "Wilquin".}

\ex:\exref|sum-triangle-by-parts|%
Use summation by parts to evaluate $S_n=\sum_{k=1}^n H_k/k$. \Hint: Consider
also the related sum $\sum_{k=1}^n H_{k-1}/k$.
\answer Let $u(k)=H_{k-1}$ and $\Delta v(k)=1/k$, so that $u(k)=v(k)$.
Then we have $S_n-H_n^{(2)}=\sum_{k=1}^nH_{k-1}/k=H_{k-1}^2\between_1^{n+1}
-S_n=H_n^2-S_n$.

\ex:\exref|prove-fib-gcd|%
Prove the gcd law \eq(|fib-gcd|) for Fibonacci numbers.
\answer Observe that when $m>n$ we have $\gcd(F_m,F_n)=\gcd(F_{m-n},F_n)$
by \equ(6.|fn+k|). This yields a proof by induction.
\source{"Lucas" [|lucas-gcd|].}

\ex:\exref|lucas-numbers|%
The {\it"Lucas number"\/} $L_n$ is defined to be $F_{n+1}+F_{n-1}$. Thus,
according to \eq(|f2n|), we have $F_{2n}=F_nL_n$. Here is a table of the
first few values:
\begindisplay \let\preamble=\tablepreamble
n&&0&1&2&3&4&5&6&7&8&9&10&11&12&13\cr
\noalign{\hrule}
L_n&&2&1&3&4&7&11&18&29&47&76&123&199&322&521\cr
\enddisplay
\itemitem{a}Use the "repertoire method" to show that the solution~$Q_n$
to the general recurrence
\begindisplay
Q_0&=\alpha\,;\qquad Q_1&=\beta\,;\qquad Q_n=Q_{n-1}+Q_{n-2}\,,\quad\hbox{$n>1$}
\enddisplay
can be expressed in terms of $F_n$ and $L_n$.
\itemitem{b}Find a closed form for $L_n$ in terms of $\phi$ and $\phihat$.
\answer (a) $Q_n=\alpha(L_n-F_n)/2+\beta F_n$. (The solution can also be
written $Q_n=\alpha F_{n-1}+\beta F_n$.) (b)~$L_n=\phi^n+\phihat^n$.
\source{"Lucas" [|lucas-theorie|, chapter~18].}

\ex:\exref|prove-euler-cont|%
Prove Euler's identity for continuants, equation \eq(|euler-cont|).
\answer When $k=0$ the identity is \equ(6.|cm+n|). When $k=1$ it is, essentially,
\begindisplay
K(x_1,\ldots,x_n)x_m&=K(x_1,\ldots,x_m)\,K(x_m,\ldots,x_n)\cr
&\qquad -K(x_1,\ldots,x_{m-2})\,K(x_{m+2},\ldots,x_n)\,;
\enddisplay
in Morse code terms, the second product on the right subtracts out the
cases where the first product has intersecting dashes. When $k>1$, an
induction on~$k$ suffices, using both
\equ(6.|cont-rec|) and \equ(6.|cont-rec'|).
(The identity is also true when one or more of the subscripts
on~$K$ become~$-1$, if we adopt the convention that
 $K_{-1}=0$. When multiplication is not
commutative, Euler's identity remains valid if we write it in the form
\begindisplay
&K_{m+n}(x_1,\ldots,x_{m+n})\,K_k(x_{m+k},\ldots,x_{m+1})\cr
&\qquad=K_{m+k}(x_1,\ldots,x_{m+k})\,K_n(x_{m+n},\ldots,x_{m+1})\cr
&\qquad\qquad+(-1)^kK_{m-1}(x_1,\ldots,x_{m-1})\,K_{n-k-1}(x_{m+n},\ldots,x_{m+k+2})\,.
\enddisplay
For example, we obtain the somewhat surprising noncommutative factorizations
\begindisplay \advance\abovedisplayskip-3pt \advance\belowdisplayskip-3pt
(abc+a+c)(1+ba)=(ab+1)(cba+a+c)
\enddisplay
from the case $k=2$, $m=0$, $n=3$.)

\ex:\exref|gen-cont-deriv|%
Generalize \eq(|cont-deriv|) to find an expression for the incremented
continuant
$K(x_1,\ldots,x_{m-1},x_m+y,x_{m+1},\ldots,x_n)$, when $1\le m\le n$.
\answer The derivative of $K(x_1,\ldots,x_n)$ with respect to $x_m$ is
\begindisplay
K(x_1,\ldots,x_{m-1})\,K(x_{m+1},\ldots,x_n)\,,
\enddisplay
and the second derivative is zero;
hence the answer is
\begindisplay
K(x_1,\ldots,x_n)+K(x_1,\ldots,x_{m-1})\,K(x_{m+1},\ldots,x_n)\mskip2muy\,.
\enddisplay

\subhead Homework exercises

\ex:\exref|nk-rise-and-fall|%
Find a closed form for the
\def\\{\atopwithdelims\vert\vert}%
coefficients $n\\k$ in the representation of rising powers
by falling powers:
\begindisplay
x\_^n=\sum_k{n\\k}x\_k\,,\qquad\hbox{integer $n\ge0$}.
\enddisplay
$\bigl($For example, $x\_^4=x\_4+12x\_3+36x\_2+24x\_1$, hence ${4\\2}=36$.$\bigr)$.
\answer Since $x\_^n=(x+n-1)\_n=\sum_k{n\choose k}x\_k(n-1)\_{n-k}$, we have
\def\\{\atopwithdelims\vert\vert}%
${n\\k}={n\choose k}{(n-1)}\_{n-k}$. These coefficients, incidentally, satisfy
the recurrence
\g${n\\k}={-k\\-n}$.\g
\begindisplay
{n\\k}=(n-1+k){n-1\\k}+{n-1\\k-1}\,,\qquad\hbox{integers $n,k>0$}.
\enddisplay
\source{"Lah" [|lah|]; R.\thinspace W. "Floyd".*}

\ex:
In Chapter 5 we obtained the formulas
\begindisplay
\sum_{k\le m}{n+k\choose k}={n+m+1\choose m}\And
 \sum_{0\le k\le m}{k\choose n}={m+1\choose n+1}
\enddisplay
by unfolding the recurrence
${n\choose k}={n-1\choose k}+{n-1\choose k-1}$
in two ways. What identities appear when the analogous recurrence
${n\brace k}=k{n-1\brace k}+{n-1\brace k-1}$
is unwound?
\answer $\sum_{k\le m}k{n+k\brace k}={m+n+1\brace m}$ and
$\sum_{0\le k\le n}{k\brace m}(m+1)^{n-k}={n+1\brace m+1}$,
both of which appear in Table~|stirling-id2|.

\ex:
Table |stirling-id1| gives the values of $n\brack2$ and $n\brace2$.
What are closed forms (not involving Stirling numbers) for the next
cases, $n\brack3$ and $n\brace3$?
\answer If $n>0$, we have ${n\brack3}=\half(n-1)!\,(H_{n-1}^2-H_{n-1}^{(2)})$,
by \equ(6.|harm/k|);
${n\brace3}={1\over6}(3^n-3\cdt 2^n+3)$, by \equ(6.|sid-5|).

\ex:
What are $-1\euler k$ and $-2@\euler k$, if the basic recursion relation
\eq(|eulerian-rec|) is assumed to hold for all integers $k$ and~$n$,
"!Eulerian numbers, generalized"
and if ${n\euler k}=0$ for all $k<0$?
\answer We have ${-1\euler k}=1/(k+1)$, \ ${-2\euler k}=H_{k+1}^{(2)}$,
and in general $n\euler k$ is given by \equ(6.|eulerian-expansion|)
for all integers~$n$.

\ex:
Prove that, for every $\epsilon>0$, there exists an integer~$n>1$ (depending
on~$\epsilon$) such that $H_n\bmod1<\epsilon$.
\answer Let $n$ be the least integer $>1/\epsilon$ such that
$\lfloor H_n\rfloor>\lfloor H_{n-1}\rfloor$.
\source{1977 midterm.}

\ex:
Is it possible to stack $n$ "bricks" in such a way that the topmost brick
is not above any point of the bottommost brick, yet a person who weighs
the same as $100$ bricks can balance on the middle of the top brick
without toppling the pile?
\answer Now $d_{k+1}=\bigl(100+(1+d_1)+\cdots+(1+d_k)\bigr)/(100+k)$,
and the solution is $d_{k+1}=H_{k+100}-H_{101}+1$ for $k\ge1$. This exceeds~$2$
when $k\ge176$.

\ex:
Express $\sum_{k=1}^{mn}(k\bmod m)/k(k+1)$ in terms
of harmonic numbers, assuming that $m$ and~$n$ are positive integers.
What is the limiting value as $n\to\infty$?
\answer The sum (by parts) is $H_{mn}-\bigl({m\over m}+{m\over2m}+\cdots
+{m\over mn}\bigr)=H_{mn}-H_n$. The infinite sum is therefore $\ln m$.
(It follows that
\begindisplay \advance\abovedisplayskip-3pt \advance\belowdisplayskip-3pt
\sum_{k\ge1}{\nu_m(k)\over k(k+1)}={m\over m-1}\ln m\,,
\enddisplay
"!sideways addition" "!nu function" "!$\nu$"
 because $\nu_m(k)=(m-1)\sum_{j\ge1}(k\bmod m^j)/m^j$.)
\source{"Shallit" [|shallit-log|].}

\ex:
Find the indefinite sum $\sum{r\choose k}(-1)^kH_k\,\delta k$.
\answer $(-1)^k\bigl({r-1\choose k}r^{-1}-{r-1\choose k-1}H_k\bigr)+C$.
 (By parts, using \equ(5.|bc-alt-sum|).)

\ex:
Express $\sum_{k=1}^n H_k^2$ in terms of $n$ and $H_n$.
\answer Write it as $\sum_{1\le j\le n}j^{-1}\sum_{j\le k\le n}H_k$
and sum first on $k$ via \equ(6.|+harm-sum|), to get
\begindisplay
(n+1)H_n^2-(2n+1)H_n+2n\,.
\enddisplay
\source{[|knuth1|, exercise 1.2.7--15].}

\ex:
Prove that $1979$ divides the numerator of $\sum_{k=1}^{1319}(-1)^{k-1}\!/k$,
and give a similar result for $1987$.
\g Ah! Those were prime years.\g
\Hint: Use "Gauss's trick" to obtain a sum of fractions whose numerators
are $1979$. See also exercise~|subtract-out-even|.
\answer If $6n-1$ is prime, the numerator of
\begindisplay
\sum_{k=1}^{4n-1}{(-1)^{k-1}\over k}=H_{4n-1}-H_{2n-1}
\enddisplay
is divisible by $6n-1$, because the sum is
\begindisplay
\sum_{k=2n}^{4n-1}{1\over k}
=\sum_{k=2n}^{3n-1}\biggl({1\over k}+{1\over 6n-1-k}\biggr)
=\sum_{k=2n}^{3n-1}{6n-1\over k(6n-1-k)}\,.
\enddisplay
Similarly if $6n+1$ is prime, the numerator of $\sum_{k=1}^{4n}(-1)^{k-1}\!/k
=H_{4n}-H_{2n}$ is a multiple of~$6n+1$. For $1987$ we sum up to $k=1324$.
\source{"Klamkin" [|olympiads2|, problem 1979/1].}

\ex:
Evaluate the sum
\begindisplay
\sum_k{\bigl\lfloor(n+k)/2\bigr\rfloor\choose k}
\enddisplay
in closed form, when $n$ is an integer (possibly negative).
\answer $S_{n+1}=\sum_k{\lfloor(n+1+k)/2\rfloor\choose k}
=\sum_k{\lfloor(n+k)/2\rfloor\choose k-1}$, hence we have $S_{n+1}+S_n
=\sum_k{\lfloor(n+k)/2+1\rfloor\choose k}=S_{n+2}$. The answer is $F_{n+2}$.
\source{1973 midterm.}

\ex:
If $S$ is a set of integers, let $S+1$ be the ``shifted''
set $\{x+1\mid x\in S\}$.
How many subsets of $\{1,2,\ldots,n\}$ have the property that
$S\cup(S+1)=\{1,2,\ldots,n+1\}$?
\answer $F_n$.

\ex:
Prove that the infinite sum
\begindisplay \openup-2pt \advance\belowdisplayskip-3pt
&.1\cr
+\,&.01\cr
+\,&.002\cr
+\,&.0003\cr
+\,&.00005\cr
+\,&.000008\cr
+\,&.0000013\cr
&\quad\vdots
\enddisplay
converges to a rational number.
\answer Set $z={1\over10}$ in $\sum_{n\ge0}F_nz^n=z/(1-z-z^2)$ to get $10\over89$.
The sum is a repeating decimal with period length 44:
\begindisplay
0.11235\,95505\,61797\,75280\,89887\,64044\,94382\,02247\,19101\,12359\,55{+}\,.
\enddisplay
\source{"Brooke" and "Wall" [|brooke|].}

\ex:
Prove the converse of "Cassini's identity" \eq(|cassini-n+1-n|):
If $k$ and $m$ are integers
such that $\vert m^2-km-k^2\vert=1$, then there is an integer~$n$ such that
$k=\pm F_n$ and $m=\pm F_{n+1}$.
\answer Replace $(m,k)$ by $(-m,-k)$ or $(k,-m)$ or $(-k,m)$, if necessary,
so that $m\ge k\ge0$. The result is clear if $m=k$.
If $m>k$, we can replace $(m,k)$ by $(m-k,m)$ and use induction.
\source{Mati{\t\i}asevich [|mat-ich|]."!Matijasevich"}

\ex:
Use the "repertoire method" to solve the general recurrence
\begindisplay
X_0=\alpha\,;\qquad X_1=\beta\,;\qquad
X_n=X_{n-1}+X_{n-2}+\gamma n+\delta\,.
\enddisplay
\answer $X_n=A(n)\alpha+B(n)\beta+C(n)\gamma+D(n)\delta$, where
$B(n)=F_n$, $A(n)=F_{n-1}$, $A(n)+B(n)-D(n)=1$, and
$B(n)-C(n)+3D(n)=n$.

\ex:\exref|cos36|%
What are $\cos36^\circ$ and $\cos72^\circ$?
\answer $\phi/2$ and $\phi^{-1}\!/2$. Let $u=\cos72^\circ$ and $v=\cos36^\circ$;
then $u=2v^2-1$ and $v=1-2\sin^2 18^\circ=1-2u^2$. Hence $u+v=2(u+v)(v-u)$,
and $4v^2-2v-1=0$. We can pursue this investigation to find the five
complex fifth roots of unity:
\begindisplay
1\,,\quad{\phi^{-1}\pm i\sqrt{2^{\mathstrut}+\phi}\over2}\,,\quad
{-\phi\pm i\sqrt{3^{\mathstrut}-\phi}\over2}\,.
\enddisplay
\source{"Francesca" [|pacioli|]; "Wallis" [|wallis-phi|, chapter~4].}

\ex:
Show that
\begindisplay
2^{n-1}F_n=\sum_k{n\choose 2k{+}1}\,5^k\,,
\enddisplay
and use this identity to deduce the values of $F_p\bmod p$ and $F_{p+1}\bmod p$
when $p$ is prime.
\answer $2^n\sqrt5\,F_n=(1+\sqrt5\mskip2mu)^n-(1-\sqrt5\mskip2mu)^n$,
\g\noindent\llap{``}Let $p$ be any old prime.''\par(See [|hall-groups|], p.~419.)\g
and the even powers of $\sqrt5$ cancel out. Now let $p$ be an odd prime.
Then ${p\choose2k+1}\=0$ except when $k=(p-1)/2$, and
${p+1\choose2k+1}\=0$ except when $k=0$ or $k=(p-1)/2$;
hence $F_p\=5^{(p-1)/2}$ and $2F_{p+1}\=1+5^{(p-1)/2}$ \tmod p.
It can be shown that $5^{(p-1)/2}\=1$ when $p$ has the form $10k\pm1$,
and $5^{(p-1)/2}\=-1$ when $p$ has the form $10k\pm3$.
\source{"Lucas" [|lucas-gcd|].}

\ex:
Prove that zero-valued parameters can be removed from continuant polynomials
by collapsing their neighbors together:
\begindisplay
&K_n(x_1,\ldots,x_{m-1},0,x_{m+1},\ldots,x_n)\cr
&\ =K_{n-2}(x_1,\ldots,x_{m-2},x_{m-1}{+}x_{m+1},x_{m+2},\ldots,x_n)\,,\;\;
\hbox{$1<m<n$}.
\enddisplay
\answer Let $K_{i,j}=K_{j-i+1}(x_i,\ldots,x_j)$. Using \equ(6.|cm+n|)
repeatedly, both sides expand to $(K_{1,m-2}(x_{m-1}+x_{m+1})+K_{1,m-3})
K_{m+2,n}+K_{1,m-2}K_{m+3,n}$.
\source{[|knuth2|, exercise 4.5.3--9(c)].}

\ex:\exref|davison-cf|%
Find the continued fraction representation of the number
$\sum_{n\ge1}2^{-\lfloor n\phi\rfloor}$.
\answer Set $z=\half$ in \equ(6.|davison-id'|);
the partial quotients are $0$, $2^{F_0}$, $2^{F_1}$, $2^{F_2}$, \dots~.
("Knuth"~[|knuth-trans|] noted that this number is transcendental.)
\source{"Davison" [|davison|].}

\ex:
Define $f(n)$ for all positive integers $n$ by the recurrence
\begindisplay
f(1)&=1\,;\cr
f(2n)&=f(n)\,;\cr
f(2n+1)&=f(n)+f(n+1)\,.
\enddisplay
\itemitem{a}For which $n$ is $f(n)$ even?
\par\nobreak\smallskip
\itemitem{b}Show that $f(n)$ can be expressed in terms of continuants.
\answer (a) $f(n)$ is even $\iff$ $3\divides n$. (b) If the binary representation
of~$n$ is $(1^{a_1}0^{a_2}\!\ldots 1^{a_{m-1}}0^{a_m})_2$, where $m$~is even,
we have $f(n)\!=\!K(a_1,a_2,\ldots,a_{m-1})$.
\source{1985 midterm; "Rham" [|rham|]; "Dijkstra"~[|dijkstra|, pp.~230--232].}

\subhead Exam problems

\ex:\exref|prove-wolstenholme|%
Let $p$ be a prime number.
\smallskip
\itemitem{a}Prove that ${p\brace k}\={@p@\brack k}\=0$ \tmod p, for $1<k<p$.
\smallskip
\itemitem{b}Prove that ${p-1\brack k}\=1$ \tmod p, for $1\le k<p$.
\smallskip
\itemitem{c}Prove that ${2p-2@\brace p}\={@2p-2@\brack p}\=0$ \tmod p, if
$p>2$.
\smallskip
\itemitem{d}Prove that if $p>3$ we have ${@p@\brack2}\=0$ \tmod{p^2}. \Hint:
Consider $p\_p$.
\answer (a) Combinatorial proof: The arrangements of $\{1,2,\ldots,p\}$ into
$k$ subsets or cycles are divided into ``orbits'' of $1$ or~$p$ arrangements
each, if we add~$1$ to each element modulo~$p$. For example,
\begindisplay
&\{1,2,4\}\cup\{3,5\}\to
\{2,3,5\}\cup\{4,1\}\to
\{3,4,1\}\cup\{5,2\}\cr
&\qquad\to\{4,5,2\}\cup\{1,3\}\to
\{5,1,3\}\cup\{2,4\}\to
\{1,2,4\}\cup\{3,5\}\,.
\enddisplay
We get an orbit of size $1$ only when this transformation takes an arrangement
into itself; but then $k=1$ or $k=p$. Alternatively, there's an algebraic
proof: We have $x^p\=x\_p+x\_1$ and $x\_p\=x^p-x$ \tmod p, since Fermat's
theorem tells us that $x^p-x$ is divisible by $(x-0)(x-1)\ldots\bigl(x-(p{-}1)\bigr)$.
\par(b) This result follows from (a) and Wilson's theorem; or we can use
$x\_^{p-1}\=x\_^p\!/(x-1)\=(x^p-x)/(x-1)=x^{p-1}+x^{p-2}+\cdots+x$.
\par(c) We have ${p+1\brace k}\={p+1\brack k}\=0$ for $3\le k\le p$, then
${p+2\brace k}\={p+2@\brack k}\=0$ for $4\le k\le p$, etc.
(Similarly, we have ${2p-1\brack p}\=-{2p-1\brace p}\=1$.)
\par(d) $p!=p\_p=\sum_k(-1)^{p-k\,}p^k{p\brack k}=p^p{p\brack p}-p^{p-1}{p\brack p-1}
+\cdots+p^3{p\brack3}-p^2{p\brack2}+p{p\brack1}$. But $p{p\brack1}=p!$, so
\begindisplay
{p\brack2}=p{p\brack3}-p^2{p\brack4}+\cdots+p^{p-2}{p\brack p}
\enddisplay
is a multiple of~$p^2$. (This is called "Wolstenholme's theorem".)
\source{"Waring" [|waring|]; "Lagrange" [|lagrange-wilson|];
"Wolstenholme" [|wolstenholme|].}

\ex:\exref|basic-esw-levine|%
Let $H_n$ be written in lowest terms as $a_n/b_n$.
"!harmonic number divisibility"
\itemitem{a}Prove that $p\divides b_n\iff p\ndivides a_{\lfloor n/p\rfloor}$,
if $p$ is prime.
\itemitem{b}Find all $n>0$ such that $a_n$ is divisible by~$5$.
\answer (a) Observe that $H_n=H_n^\ast+H_{\lfloor n/p\rfloor}/p$, where
$H_n^\ast=\sum_{k=1}^n\[k\rp p]/k$.
(b)~Working mod~$5$ we have $H_r=\<0,1,4,1,0\>$
for $0\le r\le 4$. Thus the first solution is $n=4$. By part~(a) we know that
$5\divides a_n\implies 5\divides a_{\lfloor n/5\rfloor}$; so the next
possible range is $n=20+r$, $0\le r\le4$, when we have $H_n=H_n^\ast+{1\over5}H_4
=H_{20}^\ast+{1\over5}H_4+H_r+\sum_{k=1}^r20/k(20+k)$. The
numerator of $H_{20}^\ast$, like the numerator of $H_4$, is divisible by~$25$.
Hence the only solutions in this range are $n=20$ and~$n=24$. The next
possible range is $n=100+r$; now $H_n=H_n^\ast+{1\over5}H_{20}$, which is
${1\over5}H_{20}+H_r$ plus a fraction whose numerator is a multiple of~$5$.
If ${1\over5}H_{20}\=m$ \tmod5, where $m$ is an integer, the harmonic number
$H_{100+r}$ will
have a numerator divisible by~$5$ if and only if $m+H_r\=0$ \tmod5; hence
$m$ must be $\=0$, $1$, or~$4$. Working modulo~$5$ we find ${1\over5}H_{20}
={1\over5}H_{20}^\ast+{1\over25}H_4\={1\over25}H_4={1\over12}\=3$; hence
there are no solutions for $100\le n\le104$. Similarly there are none for
$120\le n\le124$; we have found all three solutions.\par
(By exercise 6.|prove-wolstenholme|(d), we always have $p^2\divides a_{p-1}$,
$p\divides a_{p^2-p}$, and $p\divides a_{p^2-1}$, if $p$ is any prime~$\ge5$.
\g(Attention, computer programmers:
 Here's an interesting condition to test, for
as many primes as you can.)\g
The argument just given shows that these are the only solutions to
$p\divides a_n$ if and only if there are no solutions to
$p^{-2}H_{p-1}+H_r\=0$ \tmod p for $0\le r<p$. The latter condition holds
not only for $p=5$ but also for $p=13$, $17$, $23$, $41$, and
$67$\dash---perhaps for infinitely many primes. The numerator of $H_n$
is divisible by~$3$ only when $n=2$, $7$, and~$22$; it is
divisible by~$7$ only when $n=6$, $42$, $48$, $295$, $299$,
$337$, $341$, $2096$, $2390$, $14675$, $16731$, $16735$, and~$102728$.)
\source{"Eswarathasan" and "Levine" [|esw-levine|].}

\ex:
Find a closed form for $\sum_{k=0}^m{n\choose k}^{-1}(-1)^kH_k$,
when $0\le m\le n$.
\Hint: Exercise 5.|bc-alt-recip| has the sum without the $H_k$ factor.
\answer Summation by parts yields
\begindisplay
{n+1\over(n+2)^2}\biggl({(-1)^m\over{n+1\choose m+1}
}\bigl((n+2)H_{m+1}-1\bigr)-1\biggr)\,.
\enddisplay
\source{"Kauck\'y" [|kaucky|] treats a special case.}

\ex:\exref|prove-staudt-clausen|%
Let $n>0$. The purpose of this exercise is to show that the denominator
"!Bernoulli numbers"
of $B_{2n}$ is the product of all primes~$p$ such that $(p{-}1)\divides(2n)$.
\itemitem{a} Show that $S_m(p)+\bigi[(p{-}1)\divides m\bigr]$ is a
multiple of $p$, when $p$~is prime and $m>0$.
\itemitem{b} Use the result of part (a) to show that
\begindisplay
B_{2n}+\sum_{p\,\,\rm prime}{\bigi[(p{-}1)\divides(2n)\bigr]\over p}=I_{2n}
\quad\hbox{is an integer.}
\enddisplay
\Hint: It suffices to prove that, if $p$ is any prime, the denominator
of the fraction $B_{2n}+\bigi[(p{-}1)\divides(2n)\bigr]/p$ is not divisible by~$p$.
\itemitem{c}Prove that the denominator of $B_{2n}$ is always an odd multiple
of~$6$, and it is equal to~$6$ for infinitely many~$n$.
\answer(a) If $m\ge p$ we have $S_m(p)\=S_{m-(p{-}1)}(p)$ \tmod p, since
$k^{p-1}\=1$ when $1\le k<p$. Also $S_{p-1}(p)\=p-1\=-1$. If $0<m<p-1$,
we can write
\begindisplay
S_m(p)=\sum_{j=0}^m{m\brack j}(-1)^{m-j}\sum_{k=0}^{p-1}k\_j
=\sum_{j=0}^m{m\brack j}(-1)^{m-j}{p\_{j+1}\over j+1}\=0\,.
\enddisplay
(b) The condition in the hint implies that the denominator of $I_{2n}$ is
not divisible by any prime~$p$; hence $I_{2n}$ must be an integer.
\g(The \undertext{numerators} of Bernoulli numbers played an important
role in early studies of "Fermat's Last Theorem";
see "Ribenboim"~[|ribenboim|].)\g
To prove the hint, we may assume that $n\!>\!1$. Then
\begindisplay
B_{2n}+{\bigi[(p{-}1)\divides
(2n)\bigr]\over p}+\sum_{k=0}^{2n-2}{2n+1\choose k}B_k@{p^{2n-k}\over2n{+}1}
\enddisplay
is an integer, by \equ(6.|sm-bern|), \equ(6.|odd-bern|), and part~(a).
 So we want
to verify that none of the fractions ${2n+1\choose k}B_kp^{2n-k}\!/(2n+1)
={2n\choose k}B_kp^{2n-k}\!/(2n-k+1)$ has a denominator divisible by~$p$.
The denominator of ${2n\choose k}B_kp$ isn't divisible by~$p$, since
$B_k$ has no $p^2$ in its denominator (by induction); and the denominator
of $p^{2n-k-1}\!/(2n-k+1)$ isn't divisible by~$p$, since $2n-k+1<p^{2n-k}$
when $k\le2n-2$;
QED. (The numbers $I_{2n}$ are tabulated in [|knuth-buckholtz|].
"Hermite" calculated them through $I_{18}$ in 1875~[|hermite-staudt|].
It turns out that $I_2=I_4=I_6=I_8=I_{10}=I_{12}=1$; hence
there {\it is\/} actually a ``simple'' pattern to
the Bernoulli numbers displayed in the text, including $-691\over2730$(!).
But the numbers $I_{2n}$ don't seem to have any memorable features when~$2n>12$.
For example, $B_{24}=-86579-\half-{1\over3}-{1\over5}-{1\over7}-{1\over13}$,
and $86579$ is prime.)\par
(c) The numbers $2-1$ and $3-1$ always divide $2n$. If $n$ is prime, the
only divisors of $2n$ are $1$, $2$, $n$, and~$2n$, so the denominator of
$B_{2n}$ for prime $n>2$ will be~$6$ unless $2n+1$ is also prime. In the
latter case we can try $4n+3$, $8n+7$, \dots, until we eventually hit
a nonprime (since $n$ divides $2^{n-1}n+2^{n-1}-1$). (This proof does not
need the more difficult, but true, theorem that there are infinitely many
primes of the form $6k+1$.) The denominator of $B_{2n}$ can be~$6$
also when $n$ has nonprime values, such as~$49$.
\source{"Staudt" [|staudt|]; "Clausen" [|clausen3|]; "Rado" [|rado|].}

\ex:
Prove \eq(|harm-sum++|) as a corollary of a more general identity,
by summing
\begindisplay
\sum_{0\le k<n}{k\choose m}{x+k\choose k}
\enddisplay
and differentiating with respect to $x$.
\answer The stated sum is ${m+1\over x+m+1}{x+n\choose n}{n\choose m+1}$,
by Vandermonde's convolution. To get \equ(6.|harm-sum++|), differentiate
and set $x=0$.
\source{"Andrews" and "Uchimura" [|andrews-uchimura|].}

\ex:
Evaluate $\sum_{k\ne m}{n\choose k}(-1)^k@ k^{n+1}\!/(k-m)$
in closed form as a function of the integers $m$ and~$n$. (The sum is
over all integers~$k$ except for the value $k=m$.)
\answer First replace $k^{n+1}$ by $\bigl((k-m)+m\bigr){}^{n+1}$ and
expand in powers of $k-m@$; simplifications occur as in the
derivation of \equ(6.|tn-def|). If $m>n$ or $m<0$, the answer is
$(-1)^n n!-m^n\!/{n-m\choose n}$. Otherwise we need to take the limit of
\equ(5.|recip-bc|) minus the term for $k=m$, as $x\to-m$; the
answer comes to $(-1)^nn!+(-1)^{m+1}{n\choose m}m^n(n+1+mH_{n-m}-mH_m)$.
\source{1986 midterm.}

\ex:
The ``"wraparound binomial coefficients" of order~$5$'' are defined by
\begindisplay
\double(n\choose k)=\double(n-1\choose k)\;+\;
 \double(n-1\choose{(k-1)}\bmod5)\,,\qquad\hbox{$n>0$},
\enddisplay
and $\double(0\choose k)=\[k=0]$. Let $Q_n$ be the difference between the
largest and smallest of these numbers in row~$n$:
\begindisplay
Q_n=\max_{0\le k<5}\double(n\choose k)\;-\;\min_{0\le k<5}\double(n\choose k)\,.
\enddisplay
Find and prove a relation between $Q_n$ and the Fibonacci numbers.
\answer First prove by induction that the $n$th row contains at most three
distinct values $A_n\ge B_n\ge C_n$; if $n$ is even they occur in
the cyclic order $[C_n,B_n,A_n,B_n,C_n]$, while if $n$ is odd they occur in
the cyclic order $[C_n,B_n,A_n,A_n,B_n]$. Also
\begindisplay
A_{2n+1}&=A_{2n}+B_{2n}\,;}\qquad\qquad\hfill{A_{2n}&=2A_{2n-1}\,;\cr
B_{2n+1}&=B_{2n}+C_{2n}\,;}\qquad\qquad\hfill{B_{2n}&=A_{2n-1}+B_{2n-1}\,;\cr
C_{2n+1}&=2C_{2n}\,;}\qquad\hfill{C_{2n}&=B_{2n-1}+C_{2n-1}\,.\cr
\enddisplay
It follows that $Q_n=A_n-C_n=F_{n+1}$. (See exercise 5.|wraparound3| for
wraparound binomial coefficients of order~$3$.)
\source{1984 midterm, suggested by R.\thinspace W. "Floyd".*}

\ex:
Find closed forms for $\sum_{n\ge0}F_n^{@2}z^n$ and
$\sum_{n\ge0}F_n^{@3}z^n$.
What do you deduce about the quantity
 $F_{n+1}^{@3}-4F_n^{@3}-F_{n-1}^{@3}$?
\answer {\binoppenalty=10000
 (a)~$\sum_{n\ge0}F_n^{@2}z^n=z(1-z)/(1+z)(1-3z+z^{@2})
={1\over5}\bigl((2-3z)/(1-3z+z^{@2})-\allowbreak 2/(1+z)\bigr)$.}
(Square Binet's formula \equ(6.|fib-sol|) and sum on~$n$, then
combine terms so that $\phi$ and $\phihat$ disappear.) (b)~Similarly,
\begindisplay\tightplus
\sum_{n\ge0}F_n^{@3}z^n={z@(1-2z-z^{@2})\over(1-4z-z^2)(1+z-z^2)}
={1\over5}\left({2z\over 1-4z-z^2}+{3z\over1+z-z^2}\right)\,.
\enddisplay
It follows that
$F_{n+1}^{@3}-4F_n^{@3}-F_{n-1}^{@3}=3(-1)^nF_n$.
(The corresponding recurrence for $m$th powers involves the
"Fibonomial coefficients" of exercise~|fibonomial|; it was discovered by
"Jarden" and "Motzkin" [|jarden|].)
\source{[|knuth1|, exercise 1.2.8--30]; 1982 midterm.\kern-3.5pt}

\ex:\exref|fib-complete-3n|%
Prove that if $m$ and $n$ are positive integers, there exists an integer~$x$
such that $F_x\=m$~\tmod{3^n}.
\answer Let $m$ be fixed. We can prove by induction on $n$ that it is, in fact,
possible to find such an~$x$ with the additional condition $x\not\=2$
\tmod4. If $x$ is such a solution, we can move up to a solution
modulo $3^{n+1}$ because
\begindisplay
F_{8\cdot3^{n-1}}\=3^n\,,\qquad
F_{8\cdot3^{n-1}-1}\=3^n+1\pmod{3^{n+1}}\,;
\enddisplay
either $x$ or $x+8\cdt 3^{n-1}$ or $x+16\cdt 3^{n-1}$ will do the job.
\source{"Burr"~[|burr-fib|].}

\ex:
Find all positive integers $n$ such that either $F_n+1$ or $F_n-1$ is a prime
number.
\answer $F_1+1$, $F_2+1$, $F_3+1$, $F_4-1$, and $F_6-1$ are the only cases.
Otherwise the Lucas numbers of exercise~|lucas-numbers| arise in the
factorizations
\begindisplay
F_{2m}+(-1)^m&=L_{m+1\,}F_{m-1}\,;}\qquad\hfill{
F_{2m+1}+(-1)^m&=L_{m\,}F_{m+1}\,;\cr
F_{2m}-(-1)^m&=L_{m-1\,}F_{m+1}\,;}\qquad\hfill{
F_{2m+1}-(-1)^m&=L_{m+1\,}F_m\,.\cr
\enddisplay
(We have $F_{m+n}-(-1)^nF_{m-n}=L_mF_n$ in general.)

\ex:
Prove the identity
\begindisplay
\sum_{k=0}^n {1\over F_{2^k}}=3-{F_{2^n-1}\over F_{2^n}}\,,
\qquad\hbox{integer $n\ge1$}.
\enddisplay
What is $\sum_{k=0}^n 1/F_{3\cdot2^k}$?
\answer $1/F_{2m}=F_{m-1}/F_m-F_{2m-1}/F_{2m}$ when $m$ is even and positive.
The second sum is $5/4-F_{3\cdot2^n-1}/F_{3\cdot2^n}$, for $n\ge1$.
\source{1976 final exam.}

\ex:\exref|recip-fib+|%
Let $A_n=\phi^n+\phi^{-n}$ and $B_n=\phi^n-\phi^{-n}$.
\itemitem{a}Find constants $\alpha$ and $\beta$ such that
$A_n=\alpha A_{n-1}+\beta A_{n-2}$ and
$B_n=\alpha B_{n-1}+\beta B_{n-2}$ for all $n\ge0$.
\itemitem{b}Express $A_n$ and $B_n$ in terms of $F_n$ and $L_n$
(see exercise~|lucas-numbers|).
\itemitem{c}Prove that $\sum_{k=1}^n1/(F_{2k+1}+1)=B_n/A_{n+1}$.
\itemitem{d}Find a closed form for $\sum_{k=1}^n1/(F_{2k+1}-1)$.
\answer (a) $A_n=\sqrt5\,A_{n-1}-A_{n-2}$ and
$B_n=\sqrt5\,B_{n-1}-B_{n-2}$. Incidentally, we also have $\sqrt5\,A_n+B_n
=2A_{n+1}$ and $\sqrt5\,B_n-A_n=2B_{n-1}$. (b)~A table of small values reveals
that
\begindisplay \def\\{\noalign{\smallskip}}
A_n=\cases{L_n,&$n$ even;\cr\\ \sqrt5\,F_n,&$n$ odd;}\qquad\qquad
B_n=\cases{\sqrt5\,F_n,&$n$ even;\cr\\ L_n,&$n$ odd.}
\enddisplay
(c) $B_n/A_{n+1}-B_{n-1}/A_n=1/(F_{2n+1}+1)$ because
$B_nA_n-B_{n-1}A_{n+1}=\sqrt5$ and $A_nA_{n+1}=\sqrt5\,(F_{2n+1}+1)$.
Notice that $B_n/A_{n+1}=(F_n/F_{n+1})\[\hbox{$n$ even}]+(L_n/L_{n+1})
\[\hbox{$n$~odd}]$.
(d)~Similarly, $\sum_{k=1}^n1/(F_{2k+1}-1)=(A_0/B_1-A_1/B_2)+\break\cdots+
(A_{n-1}/B_n-A_n/B_{n+1})=2-A_n/B_{n+1}$. This quantity can also be
expressed as $(5F_n/L_{n+1})\[\hbox{$n$ even}]+
(L_n/F_{n+1})\[\hbox{$n$ odd}]$.
\source{"Borwein" and "Borwein" [|borweins|, \S3.7].}

\subhead Bonus problems\g Bogus problems\g

\ex:
How many permutations $\pi_1\pi_2\ldots\pi_n$ of $\{1,2,\ldots,n\}$ have
exactly $k$ indices~$j$ such that
\itemitem{a}$\pi_i<\pi_j$ for all $i<j$? (Such $j$ are called
``"left-to-right maxima".\qback'')
\itemitem{b}$\pi_j>j$? (Such $j$ are called ``"excedances".\qback'')
\answer (a) $n\brack k$. There are $n-1\brack k-1$ with $\pi_n=n$
and $(n-1){n-1\brack k}$ with $\pi_n<n$. (b)~$n\euler k$. Each permutation
$\rho_1\ldots\rho_{n-1}$ of $\{1,\ldots,n-1\}$ leads to $n$ permutations
$\pi_1\pi_2\ldots\pi_n=\rho_1\ldots\rho_{j-1}\,n\,\rho_{j+1}\ldots\rho_{n-1}
\rho_j$. If $\rho_1\ldots\rho_{n-1}$ has $k$ excedances, there are
$k+1$ values of~$j$ that yield $k$ excedances in $\pi_1\pi_2\ldots\pi_n$;
the remaining $n-1-k$ values yield $k+1$. Hence the total number of ways
to get $k$ excedances in $\pi_1\pi_2\ldots\pi_n$ is $(k+1){n-1\euler k}
+\bigl((n-1)-(k-1)\bigr){n-1\euler k-1}={n\euler k}$.
\source{[|knuth1|, section 1.2.10]; "Stanley" [|stanley|, proposition 1.3.12].}

\ex:\exref|half-stirling-upper|%
What is the denominator of ${1/2\brack1/2-n\,}$, when this fraction is
reduced to lowest terms?
\answer The denominator of ${1/2@\choose2n}$ is $2^{4n-\nu_2(n)}$,
by the proof in exercise 5.|frac-bc|. The denominator of $1/2\brack1/2-n\,$
is the same, by \equ(6.|gen-st1|), because $\Euler n0=1$ and
$\Euler nk$ is even for $k>0$.

\ex:
Prove the identity
\begindisplay
\int_0^1\ldots\int_0^1\,f\bigl(\lfloor x_1+\cdots+x_n\rfloor\bigr)\,
 dx_1\ldots dx_n=\sum_k{n\euler k}{f(k)\over n!}\,.
\enddisplay
\answer This is equivalent to saying that ${n\euler k}/n!$ is the probability
that we have $\lfloor x_1+\cdots+x_n\rfloor=k$, when $x_1$, \dots,~$x_n$ are
independent random numbers uniformly distributed between $0$ and~$1$.
Let $y_j=(x_1+\cdots+x_j)\bmod1$. Then $y_1$, \dots,~$y_n$ are independently
and uniformly distributed, and $\lfloor x_1+\cdots+x_n\rfloor$ is the
number of descents in the $y$'s. The permutation of the $y$'s is random,
and the probability of $k$~descents is
the same as the probability of $k$ ascents.
\source{"Tanny" [|tanny|].}

\ex:
What is $\sum_k(-1)^k{n\euler k}$, the $n$th alternating row sum of
"Euler's triangle"?
\answer $2^{n+1}(2^{n+1}-1)B_{n+1}/(n+1)$, if $n>0$.
(See \equ(7.|euler-double-gf|) and \equ(6.|tan-series|); the desired numbers are
essentially the coefficients of $1-\tanh z$.)
\source{[|knuth3|, exercise 5.1.3--3].}

\ex:
Prove that
\begindisplay
\sum_k{n+1\brace k+1}{n-k\choose m-k}(-1)^{m-k}k!={n\euler m}\,.
\enddisplay
\answer It is $\sum_k\bigl({n\brace k+1}(k+1)!+{n\brace k}k!\bigr){n-k
\choose n-m}(-1)^{m-k}=\sum_k{n\brace k}k!(-1)^{m-k}\kern-3pt\*
\bigl({n-k\choose n-m}
-{n+1-k\choose n-m}\bigr)=\sum_k{n\brace k}k!(-1)^{m+1-k}{n-k\choose n-m-1}
={n\euler n-m-1}$ by \equ(6.|stirl2-rec|) and
\equ(6.|expand-eulerian-to-stirling|). Now use \equ(6.|eulerian-sym|).
(This identity has a combinatorial interpretation~[|chung-graham|].)
\source{"Chung" and "Graham" [|chung-graham|].}

\ex:
Show that $\Euler n1=2{n\euler1}$, and find a closed form for $\Euler n2$.
\answer We have the general formula
\begindisplay
\Euler nm=\sum_{k=0}^m{2n+1\choose k}{n+m+1-k\brace m+1-k}(-1)^k\,,
\qquad\hbox{for $n>m\ge0$},
\enddisplay
analogous to \equ(6.|eulerian-expansion|). When $m=2$ this equals
\begindisplay \openup5pt
\Euler n2&={n+3\brace3}-(2n+1){n+2\brace2}+{2n+1\choose2}{n+1\brace1}\cr
&=\textstyle\half3^{n+2}-(2n+3)2^{n+1}+\half(4n^2+6n+3)\,.
\enddisplay
\source{"Logan" [|logan-stirl|].}

\ex:
Find a closed form for $\sum_{k=1}^n k^2H_{n+k}$.
\answer ${1\over3}n(n+\half)(n+1)(2H_{2n}-H_n)-{1\over36}n(10n^2+9n-1)$.
(It would be nice to automate the derivation of formulas such as this.)
\source{[|knuth3|, exercise 6.1--13].}

\ex:
Show that the "complex harmonic numbers" of exercise~|z-harm-def|
have the power series expansion
%\begindisplay
%H_z=\sum_{n\ge2}(-1)^nH_\infty^{(n)}z^{n-1}\,.
%\enddisplay
$H_z=\sum_{n\ge2}(-1)^nH_\infty^{(n)}z^{n-1}$.
\answer $1/k-1/(k+z)=z/k^2-z^2\!/k^3+\cdots\,$, which converges
when $\vert z\vert<1$.

\ex:
Prove that the "generalized factorial" of equation \equ(5.|f-def-lim|) can
be written
\begindisplay
\prod_{k\ge1}\Big(1+{z\over k}\Bigr)e^{-z/k}
={e^{\gamma z}\over z!}\,,
\enddisplay
by considering the limit as $n\to\infty$ of the first $n$ factors of
this infinite product. Show that ${d\over dz}(z!)$ is related to the
general harmonic numbers of exercise~|z-harm-def|.
\answer Note that $\prod_{k=1}^n(1+z/k)e^{-z/k}={n+z\choose n}n^{-z}
e^{(\ln n-H_n)z}$. If $f(z)={d\over dz}(z!)$ we find $f(z)/z!+\gamma=H_z$.

\ex:\exref|trig-bern|%
Prove that the "tangent" function has the power series \eq(|tan-series|),
"!trigonometric functions"
and find the corresponding series for $z/\!\sin z$ and $\ln\bigl((\tan z)/z\bigr)$.
\answer For $\tan z$, we can use $\tan z=\cot z-2\cot 2z$ (which is equivalent
to the identity of exercise~|z/ez+1|). Also $z/\!\sin z=z\cot z+z\tan\half z$
has the power series $\sum_{n\ge0}(-1)^{n-1}(4^n-2)B_{2n}z^{2n}\!/(2n)!$; and
\begindisplay
\ln{\tan z\over z}&=\ln{\sin z\over z}-\ln\cos z\cr
&=\sum_{n\ge1}(-1)^n{4^nB_{2n}z^{2n}\over(2n)(2n)!}
\;-\;\sum_{n\ge1}(-1)^n{4^n(4^n{-}1)B_{2n}z^{2n}\over(2n)(2n)!}\cr
&=\sum_{n\ge1}(-1)^{n-1}\,{4^n(4^n-2)B_{2n}z^{2n}\over(2n)(2n)!}\,,
\enddisplay
because ${d\over dz}\ln\sin z=\cot z$ and ${d\over dz}\ln\cos z=-\tan z$.
\source{"Euler" [|euler-diff-calc|, part 2, chapter 8].}

\ex:\exref|prove-cot-poles|%
Prove that $z\cot z$ is equal to
\begindisplay
{z\over2^n}\cot{z\over2^n}-{z\over2^n}\tan{z\over2^n}
+\sum_{k=1}^{2^n-1}{z\over2^n}\biggl(\cot{z+k\pi\over2^n}
 +\cot{z-k\pi\over2^n}\biggr)\,,
\enddisplay
for all integers $n\ge1$,
and show that the limit of the $k$th summand is $2z^2\!/(z^2-k^2\pi^2)$ for
fixed~$k$ as $n\to\infty$.
\answer $\cot(z+\pi)=\cot z$ and $\cot(z+\half\pi)=-\tan z$; hence the
identity is equivalent to
\begindisplay
\cot z={1\over2^n}\sum_{k=0}^{2^n-1}\cot{z+k\pi\over2^n}\,,
\enddisplay
which follows by induction from the case $n=1$. The stated limit follows since
$z\cot z\to1$ as $z\to0$. It can be shown that term-by-term passage to
the limit is justified, hence \equ(6.|cot-poles|) is valid.
(Incidentally, the general formula
\begindisplay
\cot z={1\over n}\sum_{k=0}^{n-1}\cot{z+k\pi\over n}
\enddisplay
is also true. It can be proved from \equ(6.|cot-poles|), or from
\begindisplay
{1\over e^{nz}-1}={1\over n}\sum_{k=0}^{n-1}{1\over e^{z+2k\pi i/n}-1},
\enddisplay
which is equivalent to the "partial fraction expansion" of $1/(z^n-1)$.)
\source{"Euler" [|euler-intro-anal|, chapters 9 and 10];
"Schr\"oter" [|schroeter|].}

\ex:
Find a relation between the numbers $T_n(1)$ and the coefficients of
$1/\!\cos z$.
\answer Since $\tan2z+\sec2z=(\sin z+\cos z)/(\cos z-\sin z)$,
setting $x=1$ in \equ(6.|tan-poly-gf|) gives
$T_n(1)=2^nT_n$ when $n$~is odd, $T_n(1)=2^n\vert E_n\vert$ when $n$~is even,
where $1/\!\cos z=\sum_{n\ge0}\vert E_{2n}\vert z^{2n}\!/(2n)!$. (The
coefficients $\vert E_n\vert$ are
called {\it "secant numbers"}; with alternating signs they are called
{\it "Euler numbers"}, not to be confused with the Eulerian
numbers $n\euler k$. We have $\<E_0,E_2,E_4,\ldots\,\>=
\<1,-1,5,-61,1385,-50521,2702765,\ldots\,\>$.)


\ex:
Prove that the "tangent numbers" and the coefficients of $1/\!\cos z$
appear at the edges of the infinite triangle that begins as follows:
\begindisplay\openup-3pt\advance\abovedisplayskip3pt%
\advance\belowdisplayskip-3pt
\vbox{\halign{&\hbox to1.3em{$\hfil#\hfil$}\cr
&&&&&&1\cr
&&&&&0&&1\cr
&&&&1&&1&&0\cr
&&&0&&1&&2&&2\cr
&&5&&5&&4&&2&&0\cr
&0&&5&&10&&14&&16&&16\cr
61&&61&&56&&46&&32&&16&&0\cr}}
\enddisplay
Each row contains partial sums of the previous row, going alternately
left-to-right and right-to-left. {\it Hint:\/} Consider the coefficients
of the power series $(\sin z+\cos z)/\cos(w+z)$.
\answer Let $G(w,z)=\sin z/\cos(w+z)$ and $H(w,z)=\cos z/\cos(w+z)$, and
let $G(w,z)+H(w,z)=\sum_{m,n}A_{m,n}w^mz^n\!/m!\,n!$. Then the equations
$G(w,0)=0$ and $\bigl({\partial\over\partial z}-{\partial\over\partial w}
\bigr)G(w,z)=H(w,z)$ imply that $A_{m,0}=0$ when $m$ is odd, $A_{m,n+1}=
A_{m+1,n}+A_{m,n}$ when $m+n$ is even; the equations $H(0,z)=1$ and
$\bigl({\partial\over\partial w}-{\partial\over\partial z}\bigr)H(w,z)=
G(w,z)$ imply that $A_{0,n}=\[n=0]$ when $n$ is even, $A_{m+1,n}=
A_{m,n+1}+A_{m,n}$ when $m+n$ is odd. Consequently the $n$th row below
the apex of the triangle contains the numbers $A_{n,0}$, $A_{n-1,1}$,
\dots,~$A_{0,n}$. At the left, $A_{n,0}$ is the "secant number"
$\vert E_n\vert$; at the right, $A_{0,n}=T_n+\[n=0]$."!Euler number"
\source{"Arnold" [|arnold|].}

\ex:
Find a closed form for the sum
\begindisplay
\sum_k(-1)^k{n\brace k}2^{n-k}k!\,,
\enddisplay
and show that it is zero when $n$ is even.
\answer Let $A_n$ denote the sum. Looking ahead to equation
\equ(7.|exp-power-gf|), we see that
$\sum_n A_nz^n\!/n!=
\sum_{n,k}(-1)^k{n\brace k}2^{n-k}k!\,z^n\!/n!= % optional
\sum_k(-1)^k2^{-k}(e^{2z}-1)^k=2/(e^{2z}+1)=1-\tanh z$.
Therefore, by exercise~|z/ez+1| or |trig-bern|,
\begindisplay
A_n=(2^{n+1}-4^{n+1})B_{n+1}/(n+1)=(-1)^{(n+1)/2}T_n+\[n=0]\,.
\enddisplay
\source{"Lengyel" [|lengyel|].}

\ex:
When $m$ and $n$ are integers, $n\ge0$, the value of $\sigma_n(m)$ is
given by \eq(|s2-to-sigma|) if $m<0$, by \eq(|s1-to-sigma|) if $m\ge n$,
and by \eq(|sigma-at-0|) if $m=0$. Show that in the remaining cases we have
\begindisplay
\sigma_n(m)={(-1)^{m+n-1}\over m!\,(n-m)!}\sum_{k=0}^{m-1}{m\brack m-k}
{B_{n-k}\over n-k}\,,\quad\hbox{integer $n>m>0$.}
\enddisplay
\answer This follows by induction on $m$, using the recurrence in
exercise~|sigma-rec|. It can also be proved from \equ(6.|stirl-poly-gf|),
using the fact that
\begindisplay
{(-1)^{m-1}(m-1)!\over(e^z-1)^m}
&=(D+1)\_^{m-1}{1\over e^z-1}\cr
&=\sum_{k=0}^{m-1}{m\brack m-k}{d^{m-k-1}\over dz^{m-k-1}}{1\over e^z-1}\,,
\quad\hbox{integer $m>0$.}
\enddisplay
The latter equation, incidentally, is equivalent to
\begindisplay
{d^m\over dz^m}{1\over e^z-1}=
(-1)^m\sum_k{m+1\brace k}{(k-1)!\over(e^z-1)^k}\,,
\quad\hbox{integer $m\ge0$.}
\enddisplay

\ex:
Prove the following relation that connects "Stirling numbers", "Bernoulli numbers",
and "Catalan numbers":
\begindisplay
\sum_{k=0}^n{n+k\brace k}{2n\choose n+k}{(-1)^k\over k+1}=B_n{2n\choose n}{1\over n+1}\,.
\enddisplay
\answer If $p(x)$ is any polynomial of degree $\le n$, we have
\begindisplay
p(x)=\sum_k p(-k){-x\choose k}{x+n\choose n-k}\,,
\enddisplay
because this equation holds for $x=0$, $-1$, \dots, $-n$. The stated identity
is the special case where $p(x)=x\sigma_n(x)$ and $x=1$.
Incidentally, we obtain a simpler expression for Bernoulli numbers in terms
of Stirling numbers by setting $k=1$ in \equ(6.|stirl-stirl-bern|):
\begindisplay
\sum_{k\ge0}{m\brace k}(-1)^k{k!\over k+1}=B_m\,.
\enddisplay
\source{"Logan" [|logan-stirl|].}

\ex:
Show that the four chessboard pieces of the $64=65$ paradox can also be
reassembled to prove that $64=63$.
\answer Sam "Loyd" [|loyd-cyclopedia|, pages 288 and 378]
gave the construction "!optical illusion"
\begindisplay
\unitlength=.0033333in
\beginpicture(520,442)(0,0)
\put(40,200){\line(0,1){242}}
\put(80,200){\line(0,1){242}}
\put(120,200){\line(0,1){242}}
\put(160,200){\line(0,1){242}}
\put(200,200){\line(0,1){40}}
\put(240,200){\line(0,1){40}}
\put(280,200){\line(0,1){40}}
\put(320,200){\line(0,1){40}}
\put(360,0){\line(0,1){242}}
\put(400,0){\line(0,1){242}}
\put(440,0){\line(0,1){242}}
\put(480,0){\line(0,1){242}}
\put(0,402){\line(1,0){200}}
\put(0,362){\line(1,0){200}}
\put(0,322){\line(1,0){200}}
\put(0,282){\line(1,0){200}}
\put(0,242){\line(1,0){200}}
\put(320,200){\line(1,0){200}}
\put(320,160){\line(1,0){200}}
\put(320,120){\line(1,0){200}}
\put(320,80){\line(1,0){200}}
\put(320,40){\line(1,0){200}}
\put(0,0){\squine(0,0,0,200,321,442)}
\put(0,0){\squine(0,100,200,442,442,442)}
\put(0,0){\squine(200,200,200,442,342,242)}
\put(0,0){\squine(200,360,520,242,242,242)}
\put(0,0){\squine(520,520,520,242,121,0)}
\put(0,0){\squine(520,420,320,0,0,0)}
\put(0,0){\squine(320,320,320,0,100,200)}
\put(0,0){\squine(320,160,0,200,200,200)}
\put(0,0){\squine(0,260,520,319,221,123)}
\endpicture
\enddisplay
and claimed to have invented (but not published) the $64=65$ arrangement in 1858.
(Similar paradoxes go back at least to the eighteenth century, but Loyd found
better ways to present them.)
\source{Comic section, {\sl Boston Herald}, August~21, 1904.}

\ex:
A sequence defined by the recurrence
%\begindisplay
%A_1=x\,,\qquad A_2=y\,,\qquad A_n=A_{n-1}+A_{n-2}
%\enddisplay
$A_1=x$, $A_2=y$, and $A_n=A_{n-1}+A_{n-2}$
has $A_m=1000000$ for some $m$. What positive integers $x$ and~$y$
make~$m$ as large as possible?
\answer We expect $A_m/A_{m-1}\approx\phi$, so we try $A_{m-1}=618034+r$
and $A_{m-2}=381966-r$. Then $A_{m-3}=236068+2r$, etc., and we find
$A_{m-18}=144-2584r$, $A_{m-19}=154+4181r$. Hence $r=0$, $x=154$, $y=144$,
$m=20$.
\source{"Silverman" and "Dunn" [|silverman-links|].}

\ex:
The text describes a way to change a formula involving $F_{n\pm k}$
to a formula that involves $F_n$ and~$F_{n+1}$ only. Therefore it's natural
to wonder if two such ``reduced'' formulas can be equal when they aren't
identical in form.
Let $P(x,y)$ be a polynomial in $x$ and~$y$ with integer coefficients.
Find a necessary and sufficient condition that $P(F_{n+1},F_n)=0$
for all $n\ge0$.
\answer If $P(F_{n+1},F_n)=0$ for infinitely many {\it even\/} values of~$n$,
then $P(x,y)$ is divisible by $U(x,y)-1$, where
$U(x,y)=x^2-xy-y^2$. For if $t$ is the total degree of~$P$, we can write
\begindisplay
P(x,y)=\sum_{k=0}^tq_kx^ky^{t-k}+\sum_{j+k<t}r_{j,k}x^jy^k=Q(x,y)+R(x,y)\,.
\enddisplay
Then
\begindisplay
{P(F_{n+1},F_n)\over F_n^t}=\sum_{k=0}^tq_k\left(F_{n+1}\over F_n\right)
^{\!k}+O(1/F_n)
\enddisplay
and we have $\sum_{k=0}^tq_k\phi^k=0$ by taking the limit as $n\to\infty$.
 Hence $Q(x,y)$ is a multiple
of $U(x,y)$, say $A(x,y)@U(x,y)$. But $U(F_{n+1},F_n)=(-1)^n$ and $n$ is even, so
$P_0(x,y)=P(x,y)-\bigl(U(x,y)-1\bigr)A(x,y)$ is another polynomial
such that $P_0(F_{n+1},F_n)=0$. The total degree of $P_0$ is less than~$t$,
so $P_0$ is a multiple of $U-1$ by induction on~$t$.
\par Similarly, $P(x,y)$ is divisible by $U(x,y)+1$ if
$P(F_{n+1},F_n)=0$ for infinitely many {\it odd\/} values of~$n$.
A combination of these two facts gives the desired necessary and sufficient
condition: $P(x,y)$ is divisible by $U(x,y)^2-1$.

\ex:
Explain how to add positive integers, working entirely in the "Fibonacci
number system".
\answer First add the digits without carrying, getting digits
$0$, $1$, and $2$. Then use the two carry rules
\begindisplay
0\,(d{+}1)\,(e{+}1)&\to 1\,d\,e\,,\cr
0\,(d{+}2)\,0\,e&\to 1\,d\,0\,(e+1)\,,\cr
\enddisplay
\looseness=-1
always applying the leftmost applicable carry. This process terminates because
the binary value obtained by reading $(b_m\ldots b_2)_F$ as
$(b_m\ldots b_2)_2$ increases whenever a carry is performed. But a carry
might propagate to the right of the ``Fibonacci point''; for example,
$(1)_F+(1)_F$ becomes $(10.01)_F$. Such rightward propagation extends at
most two positions; and those two digit positions can be zeroed again
by using the text's ``add~$1$'' algorithm if~necessary.\par
Incidentally, there's a corresponding ``multiplication'' operation on\break
nonnegative integers: If $m=F_{j_1}+\cdots+F_{j_q}$ and
$n=F_{k_1}+\cdots+F_{k_r}$ in the Fibo\-nacci number system, let
$m\circ n=\sum_{b=1}^q\sum_{c=1}^r F_{j_b+k_c}$, by analogy with
multiplication of binary numbers. (This definition implies that $m\circ n\approx
\g \advance\baselineskip2pt
 Exercise: $m\circ n=$\par
 \quad $mn+{}$\par
 \quad$\bigl\lfloor(m{+}1)/\phi\bigr\rfloor@n+{}$\par
 \quad$m@\bigl\lfloor(n{+}1)/\phi\bigr\rfloor$.\g
\sqrt5\,mn$ when $m$ and $n$ are large, although $1\circ n\approx\phi^2 n$.)
Fibonacci addition leads to a proof of the associative law
$l\circ(m\circ n)=(l\circ m)\circ n$.
\source{[|knuth-fib-mult|].}

\ex:
Is it possible that
a sequence $\<A_n\>$ satisfying the Fibonacci recurrence $A_n=A_{n-1}+A_{n-2}$
can contain no prime numbers, if $A_0$ and~$A_1$ are relatively prime?
\answer Yes; for example, we can take
\begindisplay
A_0&=\phantom0         331635635998274737472200656430763\,;\cr
A_1&=                 1510028911088401971189590305498785\,.
\enddisplay
The resulting sequence
has the property that $A_n$ is divisible by (but unequal to) $p_k$
when $n\bmod m_k=r_k$, where the numbers $(p_k,m_k,r_k)$ have the
following 18 respective values:
\begindisplay \def\preamble{&$##$\hfill\qquad}
(3,4,1)&
(2,3,2)&
(5,5,1)\cr
(7,8,3)&
(17,9,4)&
(11,10,2)\cr
(47,16,7)&
(19,18,10)&
(61,15,3)\cr
(2207,32,15)&
(53,27,16)&
(31,30,24)\cr
(1087,64,31)&
(109,27,7)&
(41,20,10)\cr
(4481,64,63)&
(5779,54,52)&
(2521,60,60)\cr
\enddisplay
One of these triples applies to every integer~$n$; for example, the six triples
in the first column cover every odd value of~$n$, and the middle column
covers all even~$n$ that are not divisible by~$6$. The remainder of the proof
is based on the
fact that $A_{m+n}=A_mF_{n-1}+A_{m+1}F_n$, together with the congruences
\begindisplay
A_0&\=F_{m_k-r_k}\bmod{p_k}\,,\cr
A_1&\=F_{m_k-r_k+1}\bmod{p_k}\,,\cr
\enddisplay
for each of the triples $(p_k,m_k,r_k)$. (An improved solution, in which
$A_0$ and~$A_1$ are numbers of ``only'' 17 digits each, is also
possible~[|knuth-fib-composites|].)
\source{[|graham-fib-composites|], modulo a numerical error.}

\ex:
Let $m$ and $n$ be odd, positive integers. Find closed forms for
\begindisplay
S_{m,n}^+=\sum_{k\ge0}\,{1\over F_{2mk+n}+F_m}\,;\qquad
S_{m,n}^-=\sum_{k\ge0}\,{1\over F_{2mk+n}-F_m}\,.
\enddisplay
\Hint: The sums in exercise~|recip-fib+| are
$S_{1,3}^+-S_{1,2n+3}^+$ and
$S_{1,3}^--S_{1,2n+3}^-$.
\answer The sequences of exercise |recip-fib+| satisfy $A_{-m}=A_m$,
$B_{-m}=-B_m$, and
\begindisplay
A_mA_n&=A_{m+n}+A_{m-n}\,;\cr
A_mB_n&=B_{m+n}-B_{m-n}\,;\cr
B_mB_n&=A_{m+n}-A_{m-n}\,.
\enddisplay
Let $f_k=B_{mk}/A_{mk+l}$ and
$g_k=A_{mk}/B_{mk+l}$, where $l=\half(n-m)$. Then $f_{k+1}-f_k=
A_lB_m/(A_{2mk+n}+A_m)$ and
$g_k-g_{k+1}=A_lB_m/(A_{2mk+n}-A_m)$; hence we have
\begindisplay\openup5pt
S_{m,n}^+&={\sqrt5\over A_lB_m}\lim_{k\to\infty}(f_k-f_0)&={\sqrt5\over\phi^lA_lL_m}\,;\cr
S_{m,n}^-&={\sqrt5\over A_lB_m}\lim_{k\to\infty}(g_0-g_k)&=
 {\sqrt5\over A_lL_m}\biggl({2\over B_l}-{1\over\phi^l}\biggr)\cr
&&={2\over F_lL_lL_m}-S_{m,n}^+\,.
\enddisplay

\ex:
Characterize all $N$ such that the Fibonacci residues $F_n\bmod N$ for
$n\ge0$ form the complete set
$\{0,1,\ldots,N-1\}$. (See exercise~|fib-complete-3n|.)
\answer The property holds if and only if $N$ has one of the seven forms
$5^k$, \ $2\cdt5^k$, \ $4\cdt5^k$, \ $3^j\cdt5^k$, \ $6\cdt5^k$, \
$7\cdt5^k$, \  $14\cdt5^k$.
\source{"Burr" [|burr-fib|].}

\ex:\exref|fibonomial|%
Let $C_1$, $C_2$, \dots\ be a sequence of nonzero integers such that
\begindisplay
\gcd(C_m,C_n)=C_{\gcd(m,n)}
\enddisplay
for all positive integers $m$ and $n$. Prove that the "generalized
binomial coefficients"
\begindisplay
{n\choose k}_{\scr C}={C_n C_{n-1}\ldots C_{n-k+1}\over
C_k C_{k-1}\ldots C_1}
\enddisplay
are all integers. (In particular, the ``"Fibonomial coefficients"'' formed
in this way from Fibonacci numbers are integers, by \eq(|fib-gcd|).)
\answer For any positive integer $m$, let $r(m)$ be the smallest index~$j$
such that $C_j$ is divisible by~$m$; if no such~$j$ exists, let $r(m)=\infty$.
Then $C_n$ is divisible by~$m$ if and only if
 $\gcd(C_n,C_{r(m)})$ is divisible by~$m$
if and only if $C_{\gcd(n,r(m))}$ is divisible by~$m$
if and only if $\gcd(n,r(m))=r(m)$
if and only if $n$ is divisible by~$r(m)$.\par
(Conversely, the gcd condition is easily seen to be implied by the
condition that the sequence $C_1$,~$C_2$,~\dots\ has a function $r(m)$,
possibly infinite, such that $C_n$ is divisible by~$m$ if and only if $n$ is
divisible by $r(m)$.)\par
Now let $\Pi(n)=C_1C_2\ldots C_n$, so that
\begindisplay
{m+n\choose m}_{{\scr C}}=
{\Pi(m+n)\over\Pi(m)\,\Pi(n)}\,.
\enddisplay
If $p$ is prime, the number of times $p$ divides
$\Pi(n)$ is $f_p(n)=\sum_{k\ge1}\lfloor n/r(p^k)\rfloor$, since $\lfloor n/
p^k\rfloor$ is the number of elements $\{C_1,\ldots,C_n\}$ that are
divisible by~$p^k$. Therefore $f_p(m+n)\ge f_p(m)+f_p(n)$ for all~$p$,
and ${m+n\choose m}_{\scr C}$ is an integer.
\source{[|kw-carry|.}

\ex:\exref|matrices-and-continuants|%
Show that continuant polynomials appear in the matrix product
\begindisplay
\pmatrix{0&1\cr 1&x_1}
\pmatrix{0&1\cr 1&x_2}
\ldots
\pmatrix{0&1\cr 1&x_n}
\enddisplay
and in the determinant
\begindisplay
\det\pmatrix{x_1&1&0&0&\ldots&0\cr
-1&x_2&1&0&&0\cr
0&-1&x_3&1\cr
\smash{\lower4pt\hbox{$\vdots$}}&&-1&&&\smash{\raise6pt\hbox{$\vdots$}}\cr
&&&&\ddots&1\cr
0&0&\ldots&&-1&x_n\cr}\,.
\enddisplay
\answer The matrix product is
\begindisplay
\pmatrix{K_{n-2}(x_2,\ldots,x_{n-1})&
 K_{n-1}(x_2,\ldots,x_{n-1},x_n)\cr
 K_{n-1}(x_1,x_2,\ldots,x_{n-1})&
K_n(x_1,x_2,\ldots,x_{n-1},x_n)\cr}\,.
\enddisplay
This relates to products of $L$ and $R$ as in \equ(6.|LR-gen|), because
we have
\begindisplay
R^a\pmatrix{0&1\cr 1&0\cr}=\pmatrix{0&1\cr 1&a\cr}=
\pmatrix{0&1\cr 1&0\cr}L^a\,.
\enddisplay
The determinant is $K_n(x_1,\ldots,x_n)$; the more general tridiagonal
determinant
\begindisplay
\det\pmatrix{x_1&1&0&\ldots&0\cr
y_2&x_2&1&&0\cr
0&y_3&x_3&1\cr
\smash{\lower4pt\hbox{$\vdots$}}&&&&\smash{\raise6pt\hbox{$\vdots$}}\cr
\noalign{\vskip-4pt}
&&&\ddots&1\cr
0&0&\ldots&y_n&x_n\cr}
\enddisplay
satisfies the recurrence $D_n=x_nD_{n-1}-y_nD_{n-2}$.
\source{[|knuth2|, exercises 4.5.3--2 and 3].}

\ex:
Generalizing \eq(|davison-id'|), find a continued fraction related to the
generating function $\sum_{n\ge1}z^{\lfloor n\alpha\rfloor}$, when
$\alpha$ is any positive irrational number.
\answer Let $\alpha^{-1}=a_0+1\big/\bigl(a_1+1/(a_2+\cdots\,)\bigr)$ be the
continued fraction representation of~$\alpha^{-1}$. Then we have
\begindisplay
{a_0\over z}+{1^{\mathstrut}\over
\displaystyle A_0(z)+{1^{\mathstrut}\over
\displaystyle A_1(z)+{1^{\mathstrut}\over
\displaystyle A_2(z)+{1^{\mathstrut}\over\ddots}}}}
={1-z\over z}\,\sum_{n\ge1}z^{\lfloor n\alpha\rfloor}\,,
\enddisplay
where
\begindisplay
A_m(z)={z^{-q_{m+1}}-z^{-q_{m-1}}\over z^{-q_m}-1}\,,\qquad
q_m=K_m(a_1,\ldots,a_m)\,.
\enddisplay
A proof analogous to the text's proof of \equ(6.|davison-id'|) uses a
generalization of "Zeckendorf"'s theorem ("Fraenkel"~[|fraenkel-wyt|, \S4]).
If $z=1/b$, where $b$ is an integer $\ge2$, this gives the continued
fraction representation of the transcendental number $(b-1)\sum_{n\ge1}
b^{-\lfloor n\alpha\rfloor}$, as in exercise~|davison-cf|.
\source{"Adams" and "Davison" [|adams-davison|].}

\ex:
Let $\alpha$ be an irrational number in $(@0\dts1)$ and let $a_1$, $a_2$,
$a_3$,~\dots\ be the partial quotients in its continued fraction
representation. Show that $\bigl\vert D(\alpha,n)\bigr\vert<2$
when $n=K(a_1,\ldots,a_m)$, where $D$ is the "discrepancy"
defined in Chapter~3.
\answer Let $p=K(0,a_1,a_2,\ldots,a_m)$, so that $p/n$ is the $m$th
convergent to the continued fraction. Then $\alpha=p/n+(-1)^m\!/nq$, where
$q=K(a_1,\ldots,a_m,\beta)$ and $\beta>1$. The points $\{k\alpha\}$
for $0\le k<n$ can therefore be written
\begindisplay
{0\over n},\;{1\over n}+{(-1)^m\pi_1\over nq},\;\ldots,\;
{n-1\over n}+{(-1)^m\pi_{n-1}\over nq}\,,
\enddisplay
where $\pi_{1}\ldots\pi_{n-1}$ is a permutation of $\{1,\ldots,n-1\}$.
Let $f(v)$ be the number of such points $<v$; then $f(v)$ and $vn$
both increase by~$1$ when $v$ increases from $k/n$ to $(k+1)/n$,
except when $k=0$ or $k=n-1$, so they never differ by $2$ or more.

\ex:
Let $Q_n$ be the largest denominator on level $n$ of the Stern--Brocot tree.
(Thus $\<Q_0,Q_1,Q_2,Q_3,Q_4,\ldots\,\>=\<1,2,3,5,8,\ldots\,\>$ according
to the diagram in Chapter~4.) Prove that $Q_n=F_{n+2}$.
\answer By \equ(6.|f-closed|) and \equ(6.|cont-deriv|), we want to maximize
$K(a_1,\ldots,a_m)$ over all sequences of positive integers whose
sum is $\le n+1$. The maximum occurs when all the $a$'s are~$1$, for if
$j\ge1$ and $a\ge1$ we have
\begindisplay
&K_{j+k+1}(1,\ldots,1,a+1,b_1,\ldots,b_k)\cr
&\qquad
 =K_{j+k+1}(1,\ldots,1,a,b_1,\ldots,b_k)+K_j(1,\ldots,1)\,K_k(b_1,\ldots,b_k)\cr
&\qquad
 \le K_{j+k+1}(1,\ldots,1,a,b_1,\ldots,b_k)+K_{j+k}(1,\ldots,1,a,b_1,\ldots,b_k)\cr
&\qquad=K_{j+k+2}(1,\ldots,1,a,b_1,\ldots,b_k)\,.\cr
\enddisplay
("Motzkin" and "Straus" [|motzkin-straus|] show how to
solve more general maximization
problems on continuants.)
\source{"Lehmer" [|lehmer-sp|].}

\subhead Research problems

\ex:
What is the best way to extend the definition of $n\brace k$ to arbitrary
real values of $n$ and~$k$?
\answer A candidate for the case $n\bmod1=\half$ appears in
[|knuth-caching|, \S6], although it may be best to multiply the
integers discussed there by some constant involving $\sqrt\pi$.
Alternatively, Renzo "Sprugnoli" observes that we can define
${n\brace m}=\sum_k{m\choose k}k^n(-1)^(m-k)/m!$ for integer $m\ge0$ and
arbitrary $n\ge0$; then \equ(6.|stirl2-rec|) holds for all $n\ge1$.

\ex:
Let $H_n$ be written in lowest terms as $a_n/b_n$, as in exercise
|basic-esw-levine|.
"!harmonic number divisors"
\itemitem{a}Are there infinitely many~$n$ with $11\divides a_n$?
\itemitem{b}Are there infinitely many~$n$ with $b_n=\lcm(1,2,\ldots,n)$?
(Two such values are $n=250$ and $n=1000$.)
\answer (a) If there are only finitely many solutions, it is natural to
conjecture that the same holds for all primes. (b) The behavior of $b_n$
is quite strange:
We have $b_n=\lcm(1,\ldots,n)$ for $968\le n\le1066$;
\g Another reason to remember 1066?\g
on the other hand, $b_{600}=\lcm(1,\ldots,600)/(3^3\cdt5^2\cdt43)$.
Andrew "Odlyzko" observes that $p$ divides $\lcm(1,\ldots,n)/b_n$
if and only if $kp^m\le n<(k+1)p^m$ for some $m\ge1$ and some~$k<p$
such that $p$ divides the numerator of $H_k$. Therefore infinitely
many such~$n$ exist if it can be shown, for example,
that almost all primes have only one such value of\/~$k$ (namely $k=p-1$).
\source{Part (a) is from "Eswarathasan" and "Levine" [|esw-levine|].}

\ex:
Prove that $\gamma$ and $e^\gamma$ are irrational.
\answer ("Brent" [|brent-gamma|]
found the surprisingly large "partial quotient" $1568705$
in $e^\gamma$, but this seems to be just a coincidence.
For example, "Gosper" has found even larger partial quotients in~$\pi$:
"!pi" The 453,294th is $12996958$ and the 11,504,931st is~$878783625$.)

\ex:
Develop a general theory of the solutions to the two-parameter recurrence
\def\\{\atopwithdelims\vert\vert}
\begindisplay
{n\\k}&=(\alpha n+\beta k+\gamma){n-1\\k}\cr
&\qquad\quad+(\alpha'n+\beta'k+\gamma'){n-1\\k-1}+\[n=k=0]\,,
\quad\hbox{for $n,k\ge0$},
\enddisplay
assuming that ${n\\k}=0$ when $n<0$ or $k<0$. (Binomial coefficients,
Stirling numbers, Eulerian numbers,
and the sequences of exercises |nk-examples|
and~|nk-rise-and-fall| are special cases.)
What special values $(\alpha,\beta,\gamma,\alpha',\beta',\gamma')$ yield
``fundamental solutions'' in terms of which the general solution can
be expressed?
\answer Consider the generating function $\sum_{m,n\ge0}{m+n\atopwithdelims
\vert\vert m}w^mz^n$, which has the form $\sum_n\bigl(wF(a,b,c)+
zF(a',b',c')\bigr){}^n$, where $F(a,b,c)$ is the differential operator
$a+b\vartheta_w+c\vartheta_z$.

\ex:
Find an efficient way to extend the "Gosper-Zeilberger algorithm" from
hypergeometric terms to terms that may involve Stirling numbers.
\answer Complete success might be difficult or impossible, because Stirling
numbers are not ``holonomic'' in the sense of [|zeil-hol|].


